\documentclass[a4paper,12pt]{article}

% ----------------------------
% Pacchetti utili
% ----------------------------
\usepackage[utf8]{inputenc}
\usepackage[T1]{fontenc}
\usepackage[italian]{babel}
\usepackage{graphicx}
\usepackage{xcolor}
\usepackage{geometry}
\usepackage{setspace}
\usepackage{fancyhdr}
\usepackage{tikz}
\usepackage[colorlinks=true, linkcolor=blue, urlcolor=blue, citecolor=blue]{hyperref}
% ----------------------------
% Impostazioni pagina
% ----------------------------
\geometry{
    top=2cm,
    bottom=2cm,
    left=2cm,
    right=2cm
}

\setstretch{1.2}

% ----------------------------
% Dati personalizzabili
% ----------------------------
\newcommand{\Gruppo}{Atlas}
\newcommand{\Email}{\href{mailto:team9.atlas@gmail.com}{\textcolor{blue}{\underline{team9.atlas@gmail.com}}}}
\newcommand{\TitoloVerbale}{Verbale della Riunione}
\newcommand{\DataVerbale}{2025/10/17}
\newcommand{\OraInizio}{10:00}
\newcommand{\OraFine}{12:00}
\newcommand{\LuogoVerbale}{chiamata Discord}
\newcommand{\LogoGruppo}{img/AtlasLogo.png} % Inserisci il file del logo
\newcommand{\AbstractVerbale}{%
In questo verbale vengono riportati i principali argomenti discussi, le decisioni prese e le attività pianificate durante la riunione del gruppo nel giorno \DataVerbale \space dalle \OraInizio \space alle \space \OraFine .
}

% --- Nuove variabili aggiunte ---
\newcommand{\VersioneVerbale}{v1.0} % <-- modifica qui la versione o ID
\newcommand{\VerbaleInterno}{Interno} 

\pagestyle{fancy}
\fancyhf{}
\fancyhead[L]{\Gruppo}
\fancyhead[R]{Verbale: \VerbaleInterno \space - \space \DataVerbale}
\fancyfoot[C]{\thepage}

% ----------------------------
% Inizio documento
% ----------------------------
\begin{document}

% ----------------------------
% Prima pagina
% ----------------------------
\begin{titlepage}
    \centering

    % Logo
    \vspace*{0cm}
    %\includegraphics[width=10cm]{\LogoGruppo}\\[.5cm]
    \begin{tikzpicture}
        \clip (0,-0.1) circle (4.6cm); % raggio = metà della larghezza desiderata
        \node at (0,0) {\includegraphics[width=10cm]{\LogoGruppo}};
    \end{tikzpicture}\\
    [.5cm]
    % Titolo
    {\Huge \textbf{\TitoloVerbale}}\\[0.8cm]
    {\LARGE \Gruppo}\\[0.1cm]
    {\Email}\\[1.2cm]

    % Dati riunione
    \begin{tabular}{rl}
        \textbf{Data:} & \DataVerbale \\
        \textbf{Luogo:} & \LuogoVerbale \\
        \textbf{Versione:} & \VersioneVerbale \\
        \textbf{Tipo:} & \VerbaleInterno \\
    \end{tabular}

    \vspace{1.2cm}

    % Componenti e ruoli
    {\large \textbf{Partecipanti}}\\[0.5cm]
    \begin{tabular}{l|l}
        \textbf{Nome} & \textbf{Presenza} \\
        \hline
        Andrea Difino & SI \\
        Federico Simonetto & SI \\
        Riccardo Valerio & SI \\
        Francesco Marcolongo & SI \\
        Michele Tesser & NO \\
        Giacomo Giora & SI \\
        Bilal Sabic & SI \\
    \end{tabular}

\end{titlepage}

\section*{Registro delle modifiche}{
    \begin{center} 
        \begin{tabular}{|l|l|l|l|l|}
            \hline
            \textbf{Versione} & \textbf{Data} & \textbf{Autore} & \textbf{Verificatore} & \textbf{Descrizione} \\
            \hline
            \VersioneVerbale & 2025/10/23 & Andrea Difino & Riccardo Valerio & Controllato Template\\
            \hline
            v0.1 & \DataVerbale & Andrea Difino & & Prima stesura\\
            \hline
        \end{tabular}
    \end{center}
}

\newpage

\tableofcontents

\newpage
% ----------------------------
% Inizio contenuto verbale
% ----------------------------
\section{Abstract}{
    % Abstract
    \begin{minipage}{0.9\textwidth}
        \small
        \AbstractVerbale
    \end{minipage}
}


\section{Ordine del Giorno}{
    \begin{enumerate}
        \item Modifiche alla struttura del verbale
        \item Creazione repository pubblica su GitHub per i documenti
        \item Confronto su domande da porre alle aziende
        \item Fissati gli incontri con le aziende
    \end{enumerate}
}

\section{Discussione}{
    \subsection{Modifiche alla struttura del verbale}{
        Il team ha apportato diversi miglioramenti strutturali e di contenuto. È stata innanzitutto perfezionata la sezione del Registro delle modifiche, rendendola più chiara e completa nella tracciabilità delle revisioni effettuate. Successivamente, è stato aggiunto un header in ogni pagina, contenente informazioni utili come il nome del team, la data del verbale e il tipo di verbale, così da garantire un'immediata identificazione del documento. Infine, è stata ottimizzata la sezione delle Decisioni prese, migliorandone l'organizzazione e la leggibilità per facilitare la consultazione delle scelte effettuate durante le riunioni.
    }

    \subsection{Creazione repository pubblica su GitHub per i documenti}{
        Durante questa fase di lavoro è stato caricato il primo verbale ufficiale del team, segnando l'avvio della documentazione formale delle attività. Inoltre, il gruppo ha definito e concordato la modalità di gestione delle modifiche ai documenti, stabilendo un flusso di lavoro chiaro e condiviso basato sull'uso di branch dedicati per le modifiche, successiva apertura di pull request, verifica e approvazione da parte del verificatore, e infine merge nel branch principale (main). Questo processo garantisce ordine, tracciabilità e qualità nelle revisioni future dei documenti.
    }

    \subsection{Confronto su domande da porre alle aziende}{
        Abbiamo esaminato le domande per le aziende, proposte da ciascun membro del team nel documento condiviso su Drive, analizzandole insieme per valutarne la pertinenza e selezionare le più significative da approfondire.
    }

    \subsection{Fissati gli incontri con le aziende}{
        Le aziende hanno risposto alle nostre mail e abbiamo fissato gli incontri su google meet / zoom in base alle volontà delle aziende. Gli incontri sono fissati per i seguenti giorni: 

        \begin{itemize}
            \item 2025/10/20 - Ore 16:30
            
            Incontro con l'azienda \textbf{Var Group} per il capitolato C2 \textbf{"Code Guardian"}
            \item 2025/10/21 - Ore 14:30
            
            Incontro con l'azienda \textbf{Zucchetti} per il capitolato C6 \textbf{"Second Brain"}
            \item 2025/10/23 - Ore 16:00
            
            Incontro con l'azienda \textbf{Bluewind} per il capitolato C1 \textbf{"Automated EN18031 Compliance Verification"}
        \end{itemize}
    }
}


\section{Decisioni Prese}{
    \begin{center}
    \begin{tabular}{|c|p{11cm}|}
        \hline
        \textbf{ID} & \textbf{Decisione} \\
        \hline
        D1-17/10/2025 & Iniziare Sito Web con GitHub Pages per i documenti \\
        \hline
        D2-17/10/2025 & Creare Issue su Github per la stesura dei verbali esterni delle riunioni \\
        \hline
        D3-17/10/2025 & Fissato incontro per il 2025/10/20 poco prima dell'inizio della riunione con \textbf{Var Group}. \\
        \hline
    \end{tabular}
    \end{center}
}


\end{document}
