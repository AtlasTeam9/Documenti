\documentclass[a4paper,12pt]{article}

% ----------------------------
% Pacchetti utili
% ----------------------------
\usepackage[utf8]{inputenc}
\usepackage[T1]{fontenc}
\usepackage[italian]{babel}
\usepackage{graphicx}
\usepackage{tabularx}
\usepackage{longtable}
\usepackage{xcolor}
\usepackage{amssymb}
\usepackage[table]{xcolor}
\definecolor{lightblue}{RGB}{225,240,255}
\definecolor{gold}{RGB}{255,215,0}
\usepackage{geometry}
\usepackage{setspace}
\usepackage{calc}
\usepackage{array}
\usepackage{fancyhdr}
\usepackage{tikz}
\usepackage{float}
\usepackage{pgf-pie}
\usepackage[colorlinks=true, linkcolor=blue, urlcolor=blue, citecolor=blue]{hyperref}
\usepackage{caption}

% ----------------------------
% Impostazioni pagina
% ----------------------------
\geometry{
    top=2cm,
    bottom=2cm,
    left=2cm,
    right=2cm
}

\setstretch{1.2}


% ----------------------------
% Dati personalizzabili
% ----------------------------
\newcommand{\Gruppo}{Atlas}
\newcommand{\Email}{\href{mailto:team9.atlas@gmail.com}{\textcolor{blue}{\underline{team9.atlas@gmail.com}}}}
\newcommand{\TitoloDocumento}{Piano di Qualifica}
\newcommand{\DataUltimaModifica}{2026/02/08}
\newcommand{\LogoGruppo}{../../../Assets/AtlasLogo.png} 

% --- Nuove variabili aggiunte ---
\newcommand{\VersioneDocumento}{v1.0.0} % <-- modifica qui la versione o ID
\newcommand{\TipoDocumento}{Esterno} 

\pagestyle{fancy}
\fancyhf{}
\fancyhead[L]{\Gruppo}
\fancyhead[R]{Documento: \TipoDocumento}
\fancyfoot[C]{\thepage}

% larghezza della colonna dei nomi
\newlength{\namecol}
\setlength{\namecol}{4.5cm}

\newlength{\colw}
\setlength{\colw}{1.5cm}

\setcounter{secnumdepth}{4}
\setcounter{tocdepth}{4}

% ----------------------------
% Inizio documento
% ----------------------------
\begin{document}

% ----------------------------
% Prima pagina
% ----------------------------
\begin{titlepage}

    \begin{center}

        % Logo
        \vspace*{0cm}
        \begin{tikzpicture}
            \clip (0,-0.1) circle (5.6cm);
            \node at (0,0) {\includegraphics[width=12cm]{\LogoGruppo}};
        \end{tikzpicture}\\[0.8cm]

        % Barra superiore
        \noindent\rule{\textwidth}{0.4pt}

        % Titolo
        \vspace{1cm}
        {\Huge \textbf{\TitoloDocumento}}\\[0.4cm]
        {\large Progetto di Ingegneria del Software A.A. 2025/2026}\\[0.8cm]
        {\large Versione: \VersioneDocumento}
        \vspace{1cm}

        % Barra inferiore
        \noindent\rule{\textwidth}{0.4pt}

    \end{center}

    % Informazioni in basso
    \vfill
    \noindent
    \begin{minipage}{0.5\textwidth}
        \raggedright
        \textbf{Autore:} \Gruppo\\
        \textbf{Ultima modifica:} \DataUltimaModifica
    \end{minipage}%
    \begin{minipage}{0.5\textwidth}
        \raggedleft
        \textbf{Tipo di documento:} \TipoDocumento
    \end{minipage}

\end{titlepage}


\section*{Registro delle modifiche}
    \begin{center}
    \rowcolors{2}{lightblue}{white} 
        \begin{tabularx}{\textwidth}{|l|l|l|l|X|}
            \hline
            \rowcolor{lightgray}
            \textbf{Versione} & \textbf{Data} & \textbf{Autore} & \textbf{Verificatore} & \textbf{Descrizione} \\
            \hline
            \VersioneDocumento & 2026/02/08 & Francesco Marcolongo & Giacomo Giora & Approvazione \\ 
            \hline
            v0.7.0 & 2026/02/07 & Andrea Difino & Giacomo Giora & Scritta sezione automiglioramento \\ 
            \hline
            v0.6.0 & 2026/02/06 & Andrea Difino & Giacomo Giora & Inseriti grafici nel cruscotto \\ 
            \hline
            v0.5.0 & 2026/02/04 & Andrea Difino & Giacomo Giora & Modificati test di sistema \\ 
            \hline
            v0.4.0 & 2026/01/13 & Riccardo Valerio & Francesco Marcolongo & Definiti test di sistema e di accettazione \\ 
            \hline
            v0.3.0 & 2025/12/31 & Federico Simonetto & Michele Tesser & Aggiunta metriche qualità di prodotto \\ 
            \hline
            v0.2.0 & 2025/12/17 & Federico Simonetto & Michele Tesser & Aggiunta metriche qualità di processo \\
            \hline
            v0.1.0 & 2025/12/17 & Giacomo Giora & & Stesura template \\
            \hline
        \end{tabularx}
    \end{center}


\newpage

\tableofcontents

\newpage

\listoffigures

\newpage

\listoftables

\newpage
% ----------------------------
% Inizio contenuto verbale
% ----------------------------
\section{Introduzione}{
    \subsection{Scopo del documento}{
        Il presente documento descrive nel dettaglio le strategie di verifica e \textit{validazione}\textsubscript{\href{https://atlasteam9.github.io/Atlas/glossario.html\#Validazione}{G}} adottate per garantire la qualità dei processi e del prodotto software. Viene offerta un'analisi esaustiva delle metriche e delle metodologie impiegate per il controllo e la misurazione della qualità in ogni sua componente. Il documento definisce gli obiettivi di qualità, le risorse necessarie e i test pianificati, documentando metodologie e risultati attesi. \newline
        In ottica di miglioramento continuo, il piano sarà oggetto di revisioni periodiche per riflettere l'evoluzione del progetto e i risultati delle verifiche effettuate.
    }

    \subsection{Glossario}{
        All'interno della documentazione prodotta dal team possono comparire termini suscettibili di incomprensioni o ambiguità. Per evitare questo, è disponibile un glossario contenente i termini tecnici e le loro definizioni. Un termine è consultabile nel glossario se è indicato con la notazione \textit{parola\textsubscript{\href{https://atlasteam9.github.io/Atlas/glossario.html}{G}}}.
        Premendo sulla G a pedice, l'utente verrà indirizzato alla pagina web del glossario.
    }

    \subsection{Riferimenti}{
        \subsubsection{Riferimenti normativi}{
            \begin{itemize}
                \item Norme di progetto v1.0.0
                \item Riferimento al \textit{capitolato}\textsubscript{\href{https://atlasteam9.github.io/Atlas/glossario.html\#Capitolato}{G}} 1 dell'azienda proponente:\newline \textbf{Bluewind S.r.l - Automated EN18031 Compliance Verification}\newline \url{https://www.math.unipd.it/~tullio/IS-1/2025/Progetto/C1.pdf}
            \end{itemize}
        }

        \subsubsection{Riferimenti informativi}{
            \begin{itemize}
                \item \textbf{Standard ISO/IEC 9126}\newline \url{https://it.wikipedia.org/wiki/ISO/IEC_9126}
                \item \textbf{Standard ISO/IEC 12207:1995}\newline \url{https://www.math.unipd.it/~tullio/IS-1/2009/Approfondimenti/ISO_12207-1995.pdf}
            \end{itemize}
            
            Riferimenti alle slide del corso di Ingegneria del Software:
            \begin{itemize}
                \item \textbf{Regolamento del progetto didattico}\newline \url{https://www.math.unipd.it/~tullio/IS-1/2025/Dispense/PD1.pdf}
                \item \textbf{T07 - Qualità del software}\newline \url{https://www.math.unipd.it/~tullio/IS-1/2025/Dispense/T07.pdf}
                \item \textbf{T08 - Qualità di processo}\newline \url{https://www.math.unipd.it/~tullio/IS-1/2025/Dispense/T08.pdf}
                \item \textbf{T09 - Verifica e validazione: introduzione}\newline \url{https://www.math.unipd.it/~tullio/IS-1/2025/Dispense/T09.pdf}
                \item \textbf{T10 - Verifica e validazione: analisi statica}\newline \url{https://www.math.unipd.it/~tullio/IS-1/2025/Dispense/T10.pdf}
                \item \textbf{T11 - Verifica e validazione: analisi dinamica}\newline \url{https://www.math.unipd.it/~tullio/IS-1/2025/Dispense/T11.pdf}
            \end{itemize}
        }
    }
}

\newpage
% ----------------------------
% Metriche
% ----------------------------

\section{Metriche per la qualità}{
    In questa sezione vengono delineate le soglie quantitative, composte da valore accettabile e valore ottimo, per valutare l'\textit{efficacia}\textsubscript{\href{https://atlasteam9.github.io/Atlas/glossario.html\#Efficacia}{G}} e l'\textit{efficienza}\textsubscript{\href{https://atlasteam9.github.io/Atlas/glossario.html\#Efficienza}{G}} dei processi e del prodotto. Le definizioni delle metriche applicate sono dettagliate nelle sezioni 6 e 7 del documento \href{https://atlasteam9.github.io/Atlas/docs/RTB/documenti/interni/Norme%20di%20Progetto.pdf}{Norme di Progetto}.
    
    \subsection{Qualità di processo}{
    La qualità di processo rappresenta un’esigenza primaria nello \textit{sviluppo}\textsubscript{\href{https://atlasteam9.github.io/Atlas/glossario.html\#Sviluppo}{G}} software, poiché è grazie alla corretta applicazione di best practice ben definite che è possibile sviluppare un prodotto finale di alta qualità. Queste devono guidare tutte le attività, pratiche e metodologie adottate lungo l’intero ciclo di vita del software. La qualità di processo si fonda sull’idea che il raggiungimento di standard elevati nel prodotto dipenda da controlli regolari e dall’ottimizzazione continua dei processi che lo supportano, garantendo risultati che rispondano pienamente alle aspettative.
    
        \subsubsection{Processi primari}{

            \paragraph{Fornitura}
            \leavevmode
                \begin{table}[H]
                \centering
                \rowcolors{2}{lightblue}{white}
                    \begin{tabular}{|l|l|l|l|l|}
                        \hline
                        \rowcolor{gold}
                        \textbf{Metrica} & \textbf{Nome} & \textbf{Valore accettabile} & \textbf{Valore ottimo} \\
                        \hline
                        MPC01 & Earned Value (EV) & $\ge 0$ & $\ge PV, AC$; $\le EAC$\\
                        \hline
                        MPC02 & Planned Value (PV) & $\ge 0$ & $\approx AC$; $\le BAC$ \\
                        \hline
                        MPC03 & Actual Cost (AC) & $\ge 0$ & $\le EAC$ \\
                        \hline
                        MPC04 & Cost Variance (CV) & $\ge -7,5\%$ & $\ge 0$ \\
                        \hline
                        MPC05 & Schedule Variance (SV) & $\ge -7,5\%$ & $\ge 0$ \\
                        \hline
                        MPC06 & Cost Performance Index (CPI) & $\ge 0,85$ & 1 \\
                        \hline
                        MPC07 & Schedule Performance Index (SPI) & $\ge 0,85$ & 1 \\
                        \hline
                        MPC08 & Estimate At Completion (EAC) & $\pm7\%$ rispetto a BAC & $= BAC$ \\
                        \hline
                        MPC09 & Estimate To Complete (ETC) & $\ge 0$ & $\le BAC-AC$ \\
                        \hline
                    \end{tabular}
                    \caption{Dettagli delle metriche del processo di fornitura} 
                    \label{tab:metriche-fornitura}
                \end{table}
            

            \paragraph{Sviluppo}
            \leavevmode
                \begin{table}[H]
                \centering
                \rowcolors{2}{lightblue}{white}
                    \begin{tabular}{|l|l|l|l|l|}
                        \hline
                        \rowcolor{gold}
                        \textbf{Metrica} & \textbf{Nome} & \textbf{Valore accettabile} & \textbf{Valore ottimo} \\
                        \hline
                        MPC10 & Requirements stability index (RSI) & $\ge 75\%$ & $100\%$ \\
                        \hline
                        MPC11 & Structural Fan-In (SFIN) & - & Relativamente alto \\
                        \hline
                        MPC12 & Structural Fan-Out (SFOUT) & - & Il minimo possibile \\
                        \hline
                    \end{tabular}
                    \caption{Dettagli delle metriche del processo di sviluppo} 
                    \label{tab:metriche-sviluppo}
                \end{table}
            
        }

        \subsubsection{Processi di supporto}{

            \paragraph{Documentazione}
            \leavevmode
                \begin{table}[H]
                \centering
                \rowcolors{2}{lightblue}{white}
                    \begin{tabular}{|l|l|l|l|l|}
                        \hline
                        \rowcolor{gold}
                        \textbf{Metrica} & \textbf{Nome} & \textbf{Valore accettabile} & \textbf{Valore ottimo} \\
                        \hline
                        MPC13 & Indice Gulpease & $\ge 60\%$ & $\ge 80\%$ \\
                        \hline
                        MPC14 & Correttezza ortografica & $\le 2$ & 0 \\
                        \hline
                    \end{tabular}
                    \caption{Dettagli delle metriche del processo di documentazione} 
                    \label{tab:metriche-documentazione}
                \end{table}

            \paragraph{Verifica}
            \leavevmode
                \begin{table}[H]
                \centering
                \rowcolors{2}{lightblue}{white}
                    \begin{tabular}{|l|l|l|l|l|}
                        \hline
                        \rowcolor{gold}
                        \textbf{Metrica} & \textbf{Nome} & \textbf{Valore accettabile} & \textbf{Valore ottimo} \\
                        \hline
                        MPC15 & Code coverage & $\ge 80\%$ & $100\%$ \\
                        \hline
                        MPC16 & Test Success Rate & $100\%$ & $100\%$ \\
                        \hline
                    \end{tabular}
                    \caption{Dettagli delle metriche del processo di verifica} 
                    \label{tab:metriche-verifica}
                \end{table}
            
            \paragraph{Gestione della qualità}
            \leavevmode
                \begin{table}[H]
                \centering
                \rowcolors{2}{lightblue}{white}
                    \begin{tabular}{|l|l|l|l|l|}
                        \hline
                        \rowcolor{gold}
                        \textbf{Metrica} & \textbf{Nome} & \textbf{Valore accettabile} & \textbf{Valore ottimo} \\
                        \hline
                        MPC17 & Quality metrics satisfied & $\ge 85\%$ & $100\%$ \\
                        \hline
                    \end{tabular}
                    \caption{Dettagli delle metriche del processo di gestione della qualità} 
                    \label{tab:metriche-qualità}
                \end{table}
        }

        \subsubsection{Processi organizzativi}{
       
            \paragraph{Gestione dei processi}
            \leavevmode
                \begin{table}[H]
                \centering
                \rowcolors{2}{lightblue}{white}
                    \begin{tabular}{|l|l|l|l|l|}
                        \hline
                        \rowcolor{gold}
                        \textbf{Metrica} & \textbf{Nome} & \textbf{Valore accettabile} & \textbf{Valore ottimo} \\
                        \hline
                        MPC18 & Non-calculated risk & $\le 3$ & 0 \\
                        \hline
                    \end{tabular}
                    \caption{Dettagli delle metriche del processo di gestione dei processi} 
                    \label{tab:metriche-processi}
                \end{table}
        }
    }

    \subsection{Qualità di prodotto}{
        La qualità del prodotto rappresenta l’\textit{obiettivo}\textsubscript{\href{https://atlasteam9.github.io/Atlas/glossario.html\#Obiettivo}{G}} centrale di un progetto software e consiste nella capacità del prodotto finale di soddisfare pienamente i requisiti, le aspettative e le esigenze degli utenti e dei committenti. Essa è il risultato diretto della qualità dei processi adottati durante l'intero ciclo di vita del progetto. Un prodotto software è considerato di alta qualità quando:
        \begin{itemize}
            \item È funzionale, ovvero rispetta i requisiti funzionali e non funzionali definiti nel documento di \href{https://atlasteam9.github.io/Atlas/docs/RTB/documenti/esterni/Analisi%20dei%20Requisiti.pdf}{Analisi dei Requisiti};
            \item È affidabile, ossia garantisce prestazioni consistenti e prive di errori;
            \item È usabile, in quanto rende semplice e intuitiva l'interazione per gli utenti finali;
            \item È efficiente, ovvero ottimizzato per rispondere in modo rapido ed efficace alle richieste;
            \item È manutenibile, ovvero progettato per consentire aggiornamenti, correzioni e modifiche senza compromettere la stabilità.
        \end{itemize}
    
        \subsubsection{Funzionalità}
         \leavevmode
                \begin{table}[H]
                \centering
                \rowcolors{2}{lightblue}{white}
                    \begin{tabular}{|l|l|l|l|l|}
                        \hline
                        \rowcolor{gold}
                        \textbf{Metrica} & \textbf{Nome} & \textbf{Valore accettabile} & \textbf{Valore ottimo} \\
                        \hline
                        MPD01 & Requisiti obbligatori soddisfatti & $100\%$ & $100\%$ \\
                        \hline
                        MPD02 & Requisiti desiderabili soddisfatti & $0\%$ & $100\%$ \\
                        \hline
                        MPD03 & Requisiti opzionali soddisfatti & $0\%$ & $100\%$ \\
                        \hline
                    \end{tabular}
                    \caption{Dettagli delle metriche di funzionalità del prodotto}
                    \label{tab:metriche-funzionalità}
                \end{table}

        \subsubsection{Affidabilità}
        \leavevmode
                \begin{table}[H]
                \centering
                \rowcolors{2}{lightblue}{white}
                    \begin{tabular}{|l|l|l|l|l|}
                        \hline
                        \rowcolor{gold}
                        \textbf{Metrica} & \textbf{Nome} & \textbf{Valore accettabile} & \textbf{Valore ottimo} \\
                        \hline
                        MPD04 & Branch Coverage & $\ge 60\%$ & $\ge 80\%$ \\
                        \hline
                        MPD05 & Statement Coverage & $\ge 70\%$ & $\ge 90\%$ \\
                        \hline
                        MPD06 & Failure Density & $\le 0.5$ & $\le 0.1$ \\
                        \hline
                    \end{tabular}
                    \caption{Dettagli delle metriche di affidabilità del prodotto}
                    \label{tab:metriche-affidabilità}
                \end{table}

        \subsubsection{Usabilità}
        \leavevmode
                \begin{table}[H]
                \centering
                \rowcolors{2}{lightblue}{white}
                    \begin{tabular}{|l|l|l|l|l|}
                        \hline
                        \rowcolor{gold}
                        \textbf{Metrica} & \textbf{Nome} & \textbf{Valore accettabile} & \textbf{Valore ottimo} \\
                        \hline
                        MPD07 & Time on Task & $\le 60$ sec & $\le 30$ sec \\
                        \hline
                        MPD08 & Error Rate & $\le 5\%$ & $\le 2\%$ \\
                        \hline
                    \end{tabular}
                    \caption{Dettagli delle metriche di usabilità del prodotto}
                    \label{tab:metriche-usabilità}
                \end{table}

        \subsubsection{Efficienza}
        \leavevmode
                \begin{table}[H]
                \centering
                \rowcolors{2}{lightblue}{white}
                    \begin{tabular}{|l|l|l|l|l|}
                        \hline
                        \rowcolor{gold}
                        \textbf{Metrica} & \textbf{Nome} & \textbf{Valore accettabile} & \textbf{Valore ottimo} \\
                        \hline
                        MPD09 & Response Time & $\le 2$ sec & $\le 1$ sec \\
                        \hline
                    \end{tabular}
                    \caption{Dettagli delle metriche di efficienza del prodotto}
                    \label{tab:metriche-efficienza}
                \end{table}

        \subsubsection{Manutenibilità}
        \leavevmode
                \begin{table}[H]
                \centering
                \rowcolors{2}{lightblue}{white}
                    \begin{tabular}{|l|l|l|l|l|}
                        \hline
                        \rowcolor{gold}
                        \textbf{Metrica} & \textbf{Nome} & \textbf{Valore accettabile} & \textbf{Valore ottimo} \\
                        \hline
                        MPD10 & Code Smells  & $\le 10$ & $\le 5$ \\
                        \hline
                        MPD11 & Coefficient of Coupling & $\le 0.4$ & $\le 0.2$ \\
                        \hline
                        MPD12 & Cyclomatic complexity & $\le 8$ & $\le 4$ \\
                        \hline
                        MPD13 & Parametri per metodo & $\le 6$ & $\le 5$ \\
                        \hline
                        MPD14 & Linee di codice per metodo & $\le 25$ & $\le 15$ \\
                        \hline
                        MPD15 & Profondità delle gerarchie & $\le 7$ & $\le 4$ \\
                        \hline
                    \end{tabular}
                    \caption{Dettagli delle metriche di manutenibilità del prodotto}
                    \label{tab:metriche-manutenibilità}
                \end{table}
    }
}

\newpage
% ----------------------------
% Testing
% ----------------------------
\section{Strategie di testing}{

    Come stabilito nelle \href{https://atlasteam9.github.io/Atlas/docs/RTB/documenti/interni/Norme%20di%20Progetto.pdf}{Norme di Progetto}, i test effettuati saranno:
    \begin{itemize}
        \item test di Unità
        \item test di Integrazione
        \item test di Sistema
        \item test di Regressione
        \item test di Accettazione
    \end{itemize}

    \noindent Di seguito, sono evidenziati i test di sistema e i test di accettazione. Le altre tipologie verranno definite durante lo svolgimento delle attività per la Product \textit{Baseline}\textsubscript{\href{https://atlasteam9.github.io/Atlas/glossario.html\#Baseline}{G}} (PB).
        
    \subsection{Test di sistema}{
        \rowcolors{2}{lightblue}{white}
            \begin{longtable}{|l|p{0.45\textwidth}|p{0.15\textwidth}|p{0.1\textwidth}|}
                \caption{Tabella dei test di sistema} 
                \label{tab:test-sistema} \\
                \hline
                \rowcolor{gold}
                \textbf{Codice} & \textbf{Descrizione} & \textbf{Requisito di riferimento} & \textbf{Stato del test} \\
                \hline
                \endfirsthead
                
                \multicolumn{4}{c}%
                {\tablename\ \thetable\ -- \textit{continua dalla pagina precedente}} \\
                \hline
                \rowcolor{gold}
                \textbf{Codice} & \textbf{Descrizione} & \textbf{Requisito di riferimento} & \textbf{Stato del test}\\
                \hline
                \endhead
                
                \hline
                \multicolumn{4}{r}{\textit{continua nella pagina successiva}} \\
                \endfoot
                
                \hline
                \endlastfoot
                \hline
                T-1-S & Verificare che l'Utente possa inserire un dispositivo & R-1-F-Ob & NI \\
                \hline
                T-2-S & Verificare che l'Utente, per inserire un dispositivo, possa caricare un dispositivo nel Sistema & R-2-F-Ob & NI \\
                \hline
                T-3-S & Verificare che l'Utente riceva un errore in seguito ad un tentativo di caricamento di un file non valido & R-3-F-Ob & NI \\
                \hline
                T-4-S & Verificare che l’Utente possa visualizzare il dispositivo inserito (caricato o creato) & R-4-F-Ob & NI \\
                \hline
                T-5-S & Verificare che l'Utente, nel visualizzare il dispositivo inserito, possa visualizzare il nome del dispositivo & R-5-F-Ob & NI \\
                \hline
                T-6-S & Verificare che l'Utente possa visualizzare la lista degli assets di un dispositivo & R-6-F-Ob & NI \\
                \hline
                T-7-S & Verificare che l'Utente, nel visualizzare la lista degli assets di un dispositivo, possa visualizzare un singolo asset & R-7-F-Ob & NI \\
                \hline
                T-8-S & Verificare che l'Utente, nel visualizzare un singolo asset nella lista degli assets di un dispositivo, possa visualizzare il nome dell'asset & R-8-F-Ob & NI \\
                \hline
                T-9-S & Verificare che l'Utente, nel visualizzare un singolo asset nella lista degli assets di un dispositivo, possa visualizzare il tipo dell'asset & R-9-F-Ob & NI \\
                \hline
                T-10-S & Verificare che l'Utente, dopo aver inserito un dispositivo, possa avviare e completare l'esecuzione di un test su quel dispositivo & R-10-F-Ob & NI \\
                \hline
                T-11-S & Verificare che l'Utente, durante l'esecuzione di un test su un dispositivo, possa rispondere a una particolare domanda del test & R-11-F-Ob & NI \\
                \hline
                T-12-S & Verificare che l'Utente, durante l'esecuzione di un test su un dispositivo, possa modificare la risposta a una domanda precedente se seleziona la funzionalità corrispondente & R-12-F-Ob & NI \\
                \hline
                T-13-S & Verificare che l'Utente, durante l'esecuzione di un test su un dispositivo, possa modificare la risposta a una domanda successiva se seleziona la funzionalità corrispondente & R-13-F-Ob & NI \\
                \hline
                T-14-S & Verificare che l'Utente possa uscire anticipatamente da un test in fase di esecuzione su un dispositivo & R-14-F-Ob & NI \\
                \hline
                T-15-S & Verificare che l'Utente possa visualizzare i risultati di un test (completato o caricato) & R-15-F-Ob & NI \\
                \hline
                T-16-S & Verificare che l'Utente, nel visualizzare i risultati di un test, possa visualizzare la lista ordinata dei requisiti del test con i risultati & R-16-F-Ob & NI \\
                \hline
                T-17-S & Verificare che l'Utente, nel visualizzare la lista dei requisiti del test con i risultati, possa visualizzare un singolo requisito con risultato & R-17-F-Ob & NI \\
                \hline
                T-18-S & Verificare che l'Utente, nel visualizzare un singolo requisito con risultato nella lista dei requisiti del test con i risultati, possa visualizzare il nome del requisito & R-18-F-Ob & NI \\
                \hline
                T-19-S & Verificare che l'Utente, nel visualizzare un singolo requisito con risultato nella lista dei requisiti del test con i risultati, possa visualizzare il risultato ottenuto per il requisito & R-19-F-Ob & NI \\
                \hline
                T-20-S & Verificare che l'Utente possa esportare un test completato & R-20-F-D & NI \\
                \hline
                T-21-S & Verificare che l'Utente, per inserire un dispositivo, possa creare un dispositivo & R-21-F-Ob & NI \\
                \hline
                T-22-S & Verificare che l'utente, nel processo di creazione di un dispositivo, possa inserire i dati del dispositivo & R-22-F-Ob & NI \\
                \hline
                T-23-S & Verificare che l'Utente, nel processo di inserimento dei dati di un dispositivo, possa inserire il nome del dispositivo & R-23-F-Ob & NI \\
                \hline
                T-24-S & Verificare che l'Utente, nel processo di inserimento dei dati di un dispositivo, possa inserire il sistema operativo del dispositivo & R-24-F-Ob & NI \\
                \hline
                T-25-S & Verificare che l'Utente, nel processo di inserimento dei dati di un dispositivo, possa inserire la versione del firmware del dispositivo & R-25-F-Ob & NI \\
                \hline
                T-26-S & Verificare che l'Utente, nel processo di inserimento dei dati di un dispositivo, possa inserire la funzionalità del dispositivo & R-26-F-Ob & NI \\
                \hline
                T-27-S & Verificare che l'Utente, nel processo di inserimento dei dati di un dispositivo, possa inserire la descrizione del dispositivo & R-27-F-Ob & NI \\
                \hline
                T-28-S & Verificare che l'Utente, nel processo di inserimento dei dati del dispositivo, possa visualizzare un errore se inserisce in uno o più campi dei valori non validi & R-28-F-Ob & NI \\
                \hline
                T-29-S & Verificare che l'Utente, nel processo di creazione di un dispositivo, possa aggiungere almeno un asset & R-29-F-Ob & NI \\
                \hline
                T-30-S & Verificare che l'Utente, nel processo di creazione di un asset, possa inserire il nome dell'asset & R-30-F-Ob & NI \\
                \hline
                T-31-S & Verificare che l'Utente, nel processo di creazione di un asset, possa selezionare il tipo dell'asset & R-31-F-Ob & NI \\
                \hline
                T-32-S & Verificare che l'Utente, nel processo di creazione di un asset, possa inserire la descrizione dell'asset & R-32-F-Ob & NI \\
                \hline
                T-33-S & Verificare che l'Utente, nel processo di creazione di un asset, possa selezionare la sensibilità dell'asset & R-33-F-Ob & NI \\
                \hline
                T-34-S & Verificare che l'Utente, nel processo di creazione di un asset, possa visualizzare un errore se inserisce in uno o più campi dei valori non validi & R-34-F-Ob & NI \\
                \hline
                T-35-S & Verificare che l'Utente possa annullare il processo di creazione di un asset & R-35-F-Ob & NI \\
                \hline
                T-36-S & Verificare che l'Utente possa annullare il processo di creazione di un dispositivo & R-36-F-Ob & NI \\
                \hline
                T-37-S & Verificare che l'Utente possa eliminare un asset presente nella lista degli assets di un dispositivo selezionandone la funzionalità corrispondente & R-37-F-Ob & NI \\
                \hline
                T-38-S & Verificare che l'Utente possa modificare un asset presente nella lista degli assets di un dispositivo selezionandone la funzionalità corrispondente & R-38-F-Op & NI \\
                \hline
                T-39-S & Verificare che l'Utente, nel processo di modifica di un asset, possa modificare il nome del dispositivo & R-39-F-Op & NI \\
                \hline
                T-40-S & Verificare che l'Utente, nel processo di modifica di un asset, possa modificare la selezione del tipo del dispositivo & R-40-F-Op & NI \\
                \hline
                T-41-S & Verificare che l'Utente, nel processo di modifica di un asset, possa modificare la descrizione del dispositivo & R-41-F-Op & NI \\
                \hline
                T-42-S & Verificare che l'Utente, nel processo di modifica di un asset, possa modificare la selezione della sensibilità del dispositivo & R-42-F-Op & NI \\
                \hline
                T-43-S & Verificare che l'Utente, nel processo di modifica di un asset, possa visualizzare un errore se inserisce in uno o più campi dei valori non validi & R-43-F-Op & NI \\
                \hline
                T-44-S & Verificare che l'Utente possa annullare il processo di modifica di un asset & R-44-F-Op & NI \\
                \hline
                T-45-S & Verificare che l'Utente possa visualizzare un avviso che indica la presenza di modifiche non salvate se ha creato un dispositivo o ha apportato modifiche ad un dispositivo esistente & R-45-F-Ob & NI \\
                \hline
                T-46-S & Verificare che l'Utente possa effettuare il salvataggio delle modifiche non salvate quando visualizza l'avviso che indica la presenza di modifiche non salvate & R-46-F-Ob & NI \\
                \hline
                T-47-S & Verificare che l'Utente possa visualizzare le informazioni di un singolo asset nel dettaglio selezionandolo dalla lista degli assets di un dispositivo & R-47-F-Ob & NI \\
                \hline
                T-48-S & Verificare che l'Utente, nel visualizzare le informazioni di dettaglio di un asset, possa visualizzare il nome dell'asset & R-48-F-Ob & NI \\
                \hline
                T-49-S & Verificare che l'Utente, nel visualizzare le informazioni di dettaglio di un asset, possa visualizzare il tipo dell'asset & R-49-F-Ob & NI \\
                \hline
                T-50-S & Verificare che l'Utente, nel visualizzare le informazioni di dettaglio di un asset, possa visualizzare la descrizione dell'asset & R-50-F-Ob & NI \\
                \hline
                T-51-S & Verificare che l'Utente, nel visualizzare le informazioni di dettaglio di un asset, possa visualizzare la sensibilità dell'asset & R-51-F-Ob & NI \\
                \hline
                T-52-S & Verificare che l'Utente possa visualizzare i dati del dispositivo inserito nel dettaglio & R-52-F-Ob & NI \\
                \hline
                T-53-S & Verificare che l'Utente, nel visualizzare i dati del dispositivo inserito nel dettaglio, possa visualizzare il nome del dispositivo & R-53-F-Ob & NI \\
                \hline
                T-54-S & Verificare che l'Utente, nel visualizzare i dati del dispositivo inserito nel dettaglio, possa visualizzare il sistema operativo del dispositivo & R-54-F-Ob & NI \\
                \hline
                T-55-S & Verificare che l'Utente, nel visualizzare i dati del dispositivo inserito nel dettaglio, possa visualizzare la versione del firmware del dispositivo & R-55-F-Ob & NI \\
                \hline
                T-56-S & Verificare che l'Utente, nel visualizzare i dati del dispositivo inserito nel dettaglio, possa visualizzare la funzionalità del dispositivo & R-56-F-Ob & NI \\
                \hline
                T-57-S & Verificare che l'Utente, nel visualizzare i dati del dispositivo inserito nel dettaglio, possa visualizzare la descrizione del dispositivo & R-57-F-Ob & NI \\
                \hline
                T-58-S & Verificare che l'Utente possa caricare un test precedentemente effettuato & R-58-F-Ob & NI \\
                \hline
                T-59-S & Verificare che l'Utente riceva un errore se carica come test precedente un file non valido & R-59-F-Ob & NI \\
                \hline
                T-60-S & Verificare che l'Utente possa modificare i risultati di un test che ha caricato o completato selezionando un particolare asset da cui riprendere l'esecuzione & R-60-F-Ob & NI \\
                \hline
                T-61-S & Verificare che l'Utente, nel modificare i risultati di un test che ha caricato o completato, possa riprendere l'esecuzione del test a partire dall'asset selezionato & R-61-F-Ob & NI \\
                \hline
                T-62-S & Verificare che l'Utente possa completare l'esecuzione di un test caricato non completato & R-62-F-Ob & NI \\
                \hline
                T-63-S & Verificare che l'Utente, nel completare l'esecuzione di un test non completato, possa riprendere l'esecuzione del test a partire dal punto in cui era stato precedentemente fermato & R-63-F-Ob & NI \\
                \hline
                T-64-S & Verificare che l'Utente possa visualizzare la lista delle giustificazioni di un test & R-64-F-Op & NI \\
                \hline
                T-65-S & Verificare che l'Utente, nel visualizzare la lista delle giustificazioni di un test, possa visualizzare una singola giustificazione & R-65-F-Op & NI \\
                \hline
                T-66-S & Verificare che l'Utente, nel visualizzare una singola giustificazione nella lista delle giustificazioni di un test, possa visualizzare il nome del requisito & R-66-F-Op & NI \\
                \hline
                T-67-S & Verificare che l'Utente, nel visualizzare una singola giustificazione nella lista delle giustificazioni di un test, possa visualizzare il testo della giustificazione associata al requisito & R-67-F-Op & NI \\
                \hline
                T-68-S & Verificare che l'Utente possa visualizzare la lista degli assets con risultati di un requisito nel dettaglio selezionando un particolare requisito dalla lista dei requisiti del test con i risultati & R-68-F-Ob & NI \\
                \hline
                T-69-S & Verificare che l'Utente, nel visualizzare la lista degli assets con risultati di un requisito nel dettaglio, possa visualizzare un singolo asset con risultato & R-69-F-Ob & NI \\
                \hline
                T-70-S & Verificare che l'Utente, nel visualizzare un singolo asset con risultato nella lista degli assets con risultati di un requisito nel dettaglio, possa visualizzare il nome dell'asset & R-70-F-Ob & NI \\
                \hline
                T-71-S & Verificare che l'Utente, nel visualizzare un singolo asset con risultato nella lista degli assets con risultati di un requisito nel dettaglio, possa visualizzare il risultato dell'asset & R-71-F-Ob & NI \\
                \hline
                T-72-S & Verificare che l'Utente possa eliminare un dispositivo che ha inserito & R-72-F-Ob & NI \\
                \hline
                T-73-S & Verificare che l'Utente possa modificare i dati del dispositivo che ha inserito & R-73-F-Op & NI \\
                \hline
                T-74-S & Verificare che l'Utente, nel processo di modifica dei dati di un dispositivo, possa modificare il nome del dispositivo & R-74-F-Op & NI \\
                \hline
                T-75-S & Verificare che l'Utente, nel processo di modifica dei dati di un dispositivo, possa modificare il sistema operativo del dispositivo & R-75-F-Op & NI \\
                \hline
                T-76-S & Verificare che l'Utente, nel processo di modifica dei dati di un dispositivo, possa modificare la versione del firmware del dispositivo & R-76-F-Op & NI \\
                \hline
                T-77-S & Verificare che l'Utente, nel processo di modifica dei dati di un dispositivo, possa modificare la funzionalità del dispositivo & R-77-F-Op & NI \\
                \hline
                T-78-S & Verificare che l'Utente, nel processo di modifica dei dati di un dispositivo, possa modificare la descrizione del dispositivo & R-78-F-Op & NI \\
                \hline
                T-79-S & Verificare che l'Utente, nel processo di modifica dei dati di un dispositivo, pssa visualizzare un errore se inserisce in uno o più campi dei valori non validi & R-79-F-Op & NI \\
                \hline
                T-80-S & Verificare che l'Utente possa annullare il processo di modifica dei dati di un dispositivo & R-80-F-Op & NI \\
                \hline
                T-81-S & Verificare che l'Utente possa visualizzare l'elenco dei decision trees selezionando la funzionalità corrispondente & R-81-F-Op & NI \\
                \hline
                T-82-S & Verificare che l'Utente, nel visualizzare l'elenco dei decision trees, possa visualizzare un singolo decision tree & R-82-F-Op & NI \\
                \hline
                T-83-S & Verificare che l'Utente, nel visualizzare un singolo decision tree nell'elenco dei decision trees, possa visualizzare l'id del requisito associato al decision tree & R-83-F-Op & NI \\
                \hline
                T-84-S & Verificare che l'Utente, nel visualizzare un singolo decision tree nell'elenco dei decision trees, possa visualizzare il nome del requisito associato al decision tree & R-84-F-Op & NI \\
                \hline
                T-85-S & Verificare che l'Utente possa visualizzare un singolo decision tree nel dettaglio selezionandolo dalla lista dei decision trees & R-85-F-Op & NI \\
                \hline
                T-86-S & Verificare che l'Utente, nel visualizzare un singolo decision tree nel dettaglio, possa visualizzare l'id del requisito associato al decision tree & R-86-F-Op & NI \\
                \hline
                T-87-S & Verificare che l'Utente, nel visualizzare un singolo decision tree nel dettaglio, possa visualizzare il nome del requisito associato al decision tree & R-87-F-Op & NI \\
                \hline
                T-88-S & Verificare che l'Utente, nel visualizzare un singolo decision tree nel dettaglio, possa visualizzare le dipendenze del requisito associato al decision tree & R-88-F-Op & NI \\
                \hline
                T-89-S & Verificare che l'Utente, nel visualizzare un singolo decision tree nel dettaglio, possa visualizzare il grafo che rappresenta il decision tree & R-89-F-Op & NI \\
                \hline
                T-90-S & Verificare che l'Utente, nel visualizzare il grafo che rappresenta un decision tree, possa visualizzare i nodi interni del grafo & R-90-F-Op & NI \\
                \hline
                T-91-S & Verificare che l'Utente, nel visualizzare un nodo interno del grafo che rappresenta un decision tree, possa visualizzare il codice univoco del nodo & R-91-F-Op & NI \\
                \hline
                T-92-S & Verificare che l'Utente, nel visualizzare un nodo interno del grafo che rappresenta un decision tree, possa visualizzare il testo della domanda del nodo & R-92-F-Op & NI \\
                \hline
                T-93-S & Verificare che l'Utente, nel visualizzare il grafo che rappresenta un decision tree, possa visualizzare i nodi foglia del grafo & R-93-F-Op & NI \\
                \hline
                T-94-S & Verificare che l'Utente, nel visualizzare il grafo che rappresenta un decision tree, possa visualizzare i collegamenti fra i nodi del grafo & R-94-F-Op & NI \\
                \hline
                T-95-S & Verificare che l'Utente possa modificare un decision tree & R-95-F-Op & NI \\
                \hline
                T-96-S & Verificare che l'Utente, per modificare un decision tree, possa aggiungere un nodo al decision tree & R-96-F-Op & NI \\
                \hline
                T-97-S & Verificare che l'Utente, nel processo di aggiunta di un nodo al decision tree, possa inserire il codice univoco del nodo & R-97-F-Op & NI \\
                \hline
                T-98-S & Verificare che l'Utente, nel processo di aggiunta di un nodo al decision tree, possa inserire il testo della domanda del nodo & R-98-F-Op & NI \\
                \hline
                T-99-S & Verificare che l'Utente, per modificare un decision tree, possa eliminare un nodo dal decision tree & R-99-F-Op & NI \\
                \hline
                T-100-S & Verificare che l'Utente riceva un errore se cerca di eliminare il nodo root di un decision tree & R-100-F-Op & NI \\
                \hline
                T-101-S & Verificare che l'Utente, per modificare un decision tree, possa modificare la destinazione di un collegamento del decision tree & R-101-F-Op & NI \\
                \hline
                T-102-S & Verificare che l'Utente riceva un errore se la validazione della modifica del decision tree fallisce & R-102-F-Op & NI \\
                \hline
                T-103-S & Verificare che l'Utente possa annullare il processo di modifica di un decision tree & R-103-F-Op & NI \\
                \hline
                T-104-S & Verificare che l'Utente possa esportare un decision tree & R-104-F-Op & NI \\
            \end{longtable}
         
    }

    
    }

    \subsection{Test di accettazione}{
            \rowcolors{2}{lightblue}{white}
            \begin{longtable}{|l|p{0.5\textwidth}|l|}
                \caption{Tabella dei test di accettazione} 
                \label{tab:test-accettazione} \\
                \hline
                \rowcolor{gold}
                \textbf{Codice} & \textbf{Descrizione} & \textbf{Stato del test} \\
                \hline
                \endfirsthead
                
                \multicolumn{3}{c}%
                {\tablename\ \thetable\ -- \textit{continua dalla pagina precedente}} \\
                \hline
                \rowcolor{gold}
                \textbf{Codice} & \textbf{Descrizione} & \textbf{Stato del test}\\
                \hline
                \endhead
                
                \hline
                \multicolumn{3}{r}{\textit{continua nella pagina successiva}} \\
                \endfoot
                
                \hline
                \endlastfoot
                \hline
                T-1-A & Verificare che il prodotto dia la possibilità di caricare un dispositivo & NI \\
                \hline
                T-2-A & Verificare che il prodotto dia la possibilità di creare un dispositivo & NI \\
                \hline
                T-3-A & Verificare che il prodotto dia la possibilità di inserire i dati di un dispositivo & NI \\
                \hline
                T-4-A & Verificare che il prodotto dia la possibilità di visualizzare il dispositivo inserito & NI \\
                \hline
                T-5-A & Verificare che il prodotto dia la possibilità di aggiungere un asset ad un dispositivo & NI \\
                \hline
                T-6-A & Verificare che il prodotto dia la possibilità di eseguire un test su un dispositivo & NI \\
                \hline
                T-7-A & Verificare che il prodotto dia la possibilità di visualizzare i risultati di un dispositivo & NI \\
            \end{longtable}
    }

    \subsection{Tracciamento dei test di sistema}{
        \rowcolors{2}{lightblue}{white}
               \begin{longtable}{|
                    >{\centering\arraybackslash}p{.5\textwidth}|
                    >{\centering\arraybackslash}p{.5\textwidth}|
                }
                \caption{Tabella di tracciamento dei test di sistema} 
                \label{tab:tracciamento-test-sistema} \\
                \hline
                \rowcolor{gold}
                \textbf{Codice test} & \textbf{Requisito} \\
                \hline
                \endfirsthead
                
                \multicolumn{2}{c}%
                {\tablename\ \thetable\ -- \textit{continua dalla pagina precedente}} \\
                \hline
                \rowcolor{gold}
                \textbf{Codice test} & \textbf{Requisito} \\
                \hline
                \endhead
                
                \hline
                \multicolumn{2}{r}{\textit{continua nella pagina successiva}} \\
                \endfoot
                
                \hline
                \endlastfoot
            
                \hline T-1-S   & R-1-F-Ob   \\ \hline
                T-2-S   & R-2-F-Ob   \\ \hline
                T-3-S   & R-3-F-Ob   \\ \hline
                T-4-S   & R-4-F-Ob   \\ \hline
                T-5-S   & R-5-F-Ob   \\ \hline
                T-6-S   & R-6-F-Ob   \\ \hline
                T-7-S   & R-7-F-Ob   \\ \hline
                T-8-S   & R-8-F-Ob   \\ \hline
                T-9-S   & R-9-F-Ob   \\ \hline
                T-10-S  & R-10-F-Ob  \\ \hline
                T-11-S  & R-11-F-Ob  \\ \hline
                T-12-S  & R-12-F-Ob  \\ \hline
                T-13-S  & R-13-F-Ob  \\ \hline
                T-14-S  & R-14-F-Ob  \\ \hline
                T-15-S  & R-15-F-Ob  \\ \hline
                T-16-S  & R-16-F-Ob  \\ \hline
                T-17-S  & R-17-F-Ob  \\ \hline
                T-18-S  & R-18-F-Ob  \\ \hline
                T-19-S  & R-19-F-Ob  \\ \hline
                T-20-S  & R-20-F-D   \\ \hline
                T-21-S  & R-21-F-Ob  \\ \hline
                T-22-S  & R-22-F-Ob  \\ \hline
                T-23-S  & R-23-F-Ob  \\ \hline
                T-24-S  & R-24-F-Ob  \\ \hline
                T-25-S  & R-25-F-Ob  \\ \hline
                T-26-S  & R-26-F-Ob  \\ \hline
                T-27-S  & R-27-F-Ob  \\ \hline
                T-28-S  & R-28-F-Ob  \\ \hline
                T-29-S  & R-29-F-Ob  \\ \hline
                T-30-S  & R-30-F-Ob  \\ \hline
                T-31-S  & R-31-F-Ob  \\ \hline
                T-32-S  & R-32-F-Ob  \\ \hline
                T-33-S  & R-33-F-Ob  \\ \hline
                T-34-S  & R-34-F-Ob  \\ \hline
                T-35-S  & R-35-F-Ob  \\ \hline
                T-36-S  & R-36-F-Ob  \\ \hline
                T-37-S  & R-37-F-Ob  \\ \hline
                T-38-S  & R-38-F-Op  \\ \hline
                T-39-S  & R-39-F-Op  \\ \hline
                T-40-S  & R-40-F-Op  \\ \hline
                T-41-S  & R-41-F-Op  \\ \hline
                T-42-S  & R-42-F-Op  \\ \hline
                T-43-S  & R-43-F-Op  \\ \hline
                T-44-S  & R-44-F-Op  \\ \hline
                T-45-S  & R-45-F-Ob  \\ \hline
                T-46-S  & R-46-F-Ob  \\ \hline
                T-47-S  & R-47-F-Ob  \\ \hline
                T-48-S  & R-48-F-Ob  \\ \hline
                T-49-S  & R-49-F-Ob  \\ \hline
                T-50-S  & R-50-F-Ob  \\ \hline
                T-51-S  & R-51-F-Ob  \\ \hline
                T-52-S  & R-52-F-Ob  \\ \hline
                T-53-S  & R-53-F-Ob  \\ \hline
                T-54-S  & R-54-F-Ob  \\ \hline
                T-55-S  & R-55-F-Ob  \\ \hline
                T-56-S  & R-56-F-Ob  \\ \hline
                T-57-S  & R-57-F-Ob  \\ \hline
                T-58-S  & R-58-F-Ob  \\ \hline
                T-59-S  & R-59-F-Ob  \\ \hline
                T-60-S  & R-60-F-Ob  \\ \hline
                T-61-S  & R-61-F-Ob  \\ \hline
                T-62-S  & R-62-F-Ob  \\ \hline
                T-63-S  & R-63-F-Ob  \\ \hline
                T-64-S  & R-64-F-Op  \\ \hline
                T-65-S  & R-65-F-Op  \\ \hline
                T-66-S  & R-66-F-Op  \\ \hline
                T-67-S  & R-67-F-Op  \\ \hline
                T-68-S  & R-68-F-Ob  \\ \hline
                T-69-S  & R-69-F-Ob  \\ \hline
                T-70-S  & R-70-F-Ob  \\ \hline
                T-71-S  & R-71-F-Ob  \\ \hline
                T-72-S  & R-72-F-Ob  \\ \hline
                T-73-S  & R-73-F-Op  \\ \hline
                T-74-S  & R-74-F-Op  \\ \hline
                T-75-S  & R-75-F-Op  \\ \hline
                T-76-S  & R-76-F-Op  \\ \hline
                T-77-S  & R-77-F-Op  \\ \hline
                T-78-S  & R-78-F-Op  \\ \hline
                T-79-S  & R-79-F-Op  \\ \hline
                T-80-S  & R-80-F-Op  \\ \hline
                T-81-S  & R-81-F-Op  \\ \hline
                T-82-S  & R-82-F-Op  \\ \hline
                T-83-S  & R-83-F-Op  \\ \hline
                T-84-S  & R-84-F-Op  \\ \hline
                T-85-S  & R-85-F-Op  \\ \hline
                T-86-S  & R-86-F-Op  \\ \hline
                T-87-S  & R-87-F-Op  \\ \hline
                T-88-S  & R-88-F-Op  \\ \hline
                T-89-S  & R-89-F-Op  \\ \hline
                T-90-S  & R-90-F-Op  \\ \hline
                T-91-S  & R-91-F-Op  \\ \hline
                T-92-S  & R-92-F-Op  \\ \hline
                T-93-S  & R-93-F-Op  \\ \hline
                T-94-S  & R-94-F-Op  \\ \hline
                T-95-S  & R-95-F-Op  \\ \hline
                T-96-S  & R-96-F-Op  \\ \hline
                T-97-S  & R-97-F-Op  \\ \hline
                T-98-S  & R-98-F-Op  \\ \hline
                T-99-S  & R-99-F-Op  \\ \hline
                T-100-S & R-100-F-Op \\ \hline
                T-101-S & R-101-F-Op \\ \hline
                T-102-S & R-102-F-Op \\ \hline
                T-103-S & R-103-F-Op \\ \hline
                T-104-S & R-104-F-Op \\ \hline
            \end{longtable}


    }
    


\newpage
% ------------------
% \textit{CRUSCOTTO}\textsubscript{\href{https://atlasteam9.github.io/Atlas/glossario.html\#Cruscotto}{G}}
% ------------------
\section{Cruscotto di valutazione}{

    Vengono ora esposte le misurazioni eseguite durante il periodo intercorso tra l’aggiudicazione del \textit{capitolato}\textsubscript{\href{https://atlasteam9.github.io/Atlas/glossario.html\#Capitolato}{G}} e la Requirements and Technology \textit{Baseline}\textsubscript{\href{https://atlasteam9.github.io/Atlas/glossario.html\#Baseline}{G}} (\textit{RTB}\textsubscript{\href{https://atlasteam9.github.io/Atlas/glossario.html\#RTB}{G}}). Le misurazioni elencate all'inizio ma qui non presenti saranno prese in considerazione durante lo svolgimento delle attività per la Product \textit{Baseline}\textsubscript{\href{https://atlasteam9.github.io/Atlas/glossario.html\#Baseline}{G}} (PB).

    \subsection{MPC - Earned Value, Planned Value e Estimate At Completion}{
        \begin{figure}[H]
            \centering
            \fbox{\includegraphics[scale=0.65]{../../../Assets/Grafici/EV_PV_EAC.png}}
            \caption{Grafico per sprint di MPC01, MPC02 e MPC08}
        \end{figure}

        \noindent Il grafico mostra il rapporto fra le metriche Earned Value e Planned Value e le confronta con la metrica Estimate At Completion. Si può notare che il valore dell'Earned Value in alcuni casi coincide con quello del Planned Value, in altri è superiore. Questo significa che il team ha sempre completato gli obiettivi pianificati e in alcuni casi ha prodotto anche più del necessario. 
    }
    \newpage
    \subsection{MPC - Actual Cost, Estimate At Completion e Estimate To Complete}{
        \begin{figure}[H]
            \centering
            \fbox{\includegraphics[scale=0.65]{../../../Assets/Grafici/AC_EAC_ETC.png}}
            \caption{Grafico per sprint di MPC03, MPC08 e MPC09}
        \end{figure}

        \noindent Il grafico mostra le metriche Actual Cost, Estimate At Completion e Estimate To Complete a confronto. All'aumentare del valore dell'Actual Cost, il valore di Estimate To Complete, (calcolato a partire da Estimate At Completion) diminuisce più o meno della stessa quantità in quasi tutti gli sprint, e questo è un segnale del fatto che ciò che deve essere fatto viene bene compensato da ciò che è stato fatto nei vari periodi. Solo durante lo sprint 5, il valore di Estimate At Completion si mantiene inalterato nonostante un incremento dell'Actual Cost per poi scendere nuovamente in occasione dell'ultimo sprint. Da ciò si comprende che durante il periodo 5 il team non ha fatto realmente dei progressi sostanziali per terminare il lavoro necessario alla candidatura per la \textit{RTB}\textsubscript{\href{https://atlasteam9.github.io/Atlas/glossario.html\#RTB}{G}}, ma li ha anzi relegati quasi totalmente allo sprint finale.
    }
    \newpage
    \subsection{MPC - Budget At Completion e Estimate At Completion}{
        \begin{figure}[H]
            \centering
            \fbox{\includegraphics[scale=0.65]{../../../Assets/Grafici/BAC_EAC.png}}
            \caption{Grafico per sprint di MPC08}
        \end{figure}

        \noindent Il grafico mostra nel dettaglio la metrica Estimate At Completion e la confronta con il valore Budget At Completion. Mentre nel primo sprint rimane pressoché invariato, diminuisce fin sotto il limite accettabile inferiore duranti gli sprint 2 e 4. Ciò evidenzia che durante questo periodo il team ha prodotto più di quanto pianificato. Il valore si alza poi oltre il budget in corrispondenza dello sprint 5, e questo sostiene quanto affermato prima circa i rallentamenti registrati in tale periodo. Il team considera comunque positivo il fatto che Estimate At Completion non abbia mai superato il limite accettabile superiore .
    }
    \newpage
    \subsection{MPC - Cost Variance e Schedule Variance}{
        \begin{figure}[H]
            \centering
            \fbox{\includegraphics[scale=0.65]{../../../Assets/Grafici/CV_SV.png}}
            \caption{Grafico per sprint di MPC04 e MPC05}
        \end{figure}

        \noindent Il grafico evidenzia le metriche Cost Variance e Schedule Variance, che, calcolate in percentuale, misurano lo scostamento di Earned Value rispettivamente da Actual Cost e da Planned Value. Cost Variance rimane quasi sempre positivo, indicando che il progetto è in generale sotto budget. Solo in corrispondenza dello sprint 5, il progetto va leggermente sopra budget. Si nota anche che durante lo sprint 3 il valore prodotto supera di quasi il 15\% in più l'ammontare dei costi. Schedule Variance assume valore positivo. Il progetto in generale è in linea o avanti con i tempi. Durante lo sprint 3, è evidente come il valore prodotto superi quasi del 20\% quello pianificato. In questo periodo, infatti, il team ha avuto la possibilità di rendere le proprie ore molto produttive.
    }
    \newpage
    \subsection{MPC - Cost Performance Index e Schedule Performance Index}{
        \begin{figure}[H]
            \centering
            \fbox{\includegraphics[scale=0.65]{../../../Assets/Grafici/CPI_SPI.png}}
            \caption{Grafico per sprint di MPC06 e MPC07}
        \end{figure}

        \noindent Il grafico mostra le misurazioni di Cost Performance Index e Schedule Performance Index (che sono analoghe a quelle illustrate nel grafico precedente). Si nota chiaramente che i due valori non solo sono sempre sopra il limite accettabile inferiore, ma quasi in ogni sprint coincidono o sono superiori al valore ottimo. Il fatto che Cost Performance Index sia spesso sopra il valore ottimo segnala la quasi assenza del rischio di superamento dei costi (solo lo sprint 5 presenta questa possibilità). Schedule Performance Index rimane sempre maggiore di 1, cioè il progetto non è mai in ritardo con i tempi, e questo è dovuto a una attenta e costante programmazione delle attività da svolgere. 
    }
    \newpage
    \subsection{MPC - Requirements Stability Index}{
        \begin{figure}[H]
            \centering
            \fbox{\includegraphics[scale=0.65]{../../../Assets/Grafici/RSI.png}}
            \caption{Grafico per sprint di MPC10}
        \end{figure}

        \noindent Il grafico mostra che il team ha iniziato a individuare i requisiti a partire dallo sprint 1 (questo è il motivo per cui il valore di Requirements Stability Index è pari allo zero per cento). Si notano inoltre dei picchi al di sotto del limite inferiore di accettabilità in corrispondenza degli sprint 3 e 4. Durante questi periodi, infatti, il team ha effettuato degli ulteriori colloqui con l'azienda proponente per analizzare in profondità il \textit{capitolato}\textsubscript{\href{https://atlasteam9.github.io/Atlas/glossario.html\#Capitolato}{G}} d'appalto e ha di conseguenza aggiunto e modificato diversi requisiti. Il valore diventa infine accettabile a partire dallo sprint 5. 
         
    }
    \newpage
    \subsection{MPC - Indice di Gulpease}{
        \begin{figure}[H]
            \centering
            \fbox{\includegraphics[scale=0.65]{../../../Assets/Grafici/Gulpease.png}}
            \caption{Grafico per sprint di MPC13}
        \end{figure}

        \noindent Il grafico mostra una chiara propensione del team alla cura della leggibilità dei documenti. L'indice di Gulpease si è infatti mantenuto al di sopra del valore di accettabilità per tutti i documenti durante tutti gli sprint. In particolare, i documenti Analisi dei Requisiti, Piano di Progetto e Piano di Qualifica hanno registrato un valore di leggibilità sempre ottimo (e tra l'altro in crescita per i primi due). I documenti Norme di Progetto e \textit{Glossario}\textsubscript{\href{https://atlasteam9.github.io/Atlas/glossario.html\#Glossario}{G}} risultano essere invece quelli meno leggibili nonostante alcuni tentativi di innalzamento dell'indice. Il team riconosce comunque che si tratta di documenti con linguaggio tecnico e elementi non sempre testuali.
    }
    \newpage
    \subsection{MPC - Correttezza ortografica}{
        \begin{figure}[H]
            \centering
            \fbox{\includegraphics[scale=0.65]{../../../Assets/Grafici/Correttezza.png}}
            \caption{Grafico per sprint di MPC14}
        \end{figure}

        \noindent Dal grafico si evince che durante i primi sprint non sono stati commessi errori ortografici. Al termine degli sprint 3 e 4 sono invece stati rilevati degli errori rispettivamente in due e in un documento. Comunque, il team considera positiva questa misurazione in quanto riconosce che il valore si è sempre tenuto sotto il limite superiore di accettabilità e che la scrittura dei documenti si è svolta principalmente proprio negli sprint appena citati.
    }
    \newpage
    \subsection{MPC - Non Calculated Risk}{
        \begin{figure}[H]
            \centering
            \fbox{\includegraphics[scale=0.65]{../../../Assets/Grafici/NonCalculatedRisk.png}}
            \caption{Grafico per sprint di MPC18}
        \end{figure}

        \noindent Il grafico mostra chiaramente che non si sono mai presentati rischi non attesi durante gli sprint. Ciò significa che il team è riuscito a individuare efficacemente e sapientemente i rischi all'inizio della \textit{RTB}\textsubscript{\href{https://atlasteam9.github.io/Atlas/glossario.html\#RTB}{G}} e prima di ogni nuovo sprint e che si è adoperato fin da subito per cercare di mitigarli.
         
    }
    \newpage
    \subsection{MPC - Quality Metrics Satisfied}{
        \begin{figure}[H]
            \centering
            \fbox{\includegraphics[scale=0.65]{../../../Assets/Grafici/QualityMetricsSatisfied.png}}
            \caption{Grafico per sprint di MPC17}
        \end{figure}

        \noindent Questo grafico, che mostra la misurazione Quality Metrics Satisfied, consente di avere una visione generale sul rispetto della qualità del lavoro svolto. Fin dal primo sprint, non tutte le metriche sono risultate soddisfatte. Durante gli sprint 2, 3 e 4, centrali per lo svolgimento dei compiti, il valore è rimasto basso, ma comunque sopra la soglia di accettabilità. Il team giustifica questo con la poca esperienza dei membri nel gestire un tale progetto in modo efficiente. Durante gli sprint 5 e 6, è evidente una autoanalisi e una conseguente risoluzione dei problemi da parte del team.
        
    }
}

\newpage
% ------------------
% AUTOMIGLIORAMENTO
% ------------------
\section{Valutazioni per l'automiglioramento}{

    \subsection{Introduzione}{
        Il miglioramento continuo rappresenta forse il più importante \textit{obiettivo}\textsubscript{\href{https://atlasteam9.github.io/Atlas/glossario.html\#Obiettivo}{G}} per assicurare la resa e la qualità di un progetto software collaborativo. In questa sezione sono illustrate le principali criticità riscontrate e le contromisure adottate per mitigare o eliminare i rallentamenti derivanti, con alcune considerazioni finali.
    }

    \subsection{Valutazione sull'organizzazione}{
        \rowcolors{4}{white}{lightblue}
            \begin{longtable}{|
                >{\centering\arraybackslash}p{.47\textwidth}|
                >{\centering\arraybackslash}p{.47\textwidth}|
            }
            \hline
            \rowcolor{gold}
            \textbf{Descrizione problema} & \textbf{Contromisura adottata} \\
            \hline
            \endfirsthead
            
            \multicolumn{2}{c}%
            {\tablename\ \thetable\ -- \textit{continua dalla pagina precedente}} \\
            \hline
            \rowcolor{gold}
            \textbf{Descrizione problema} & \textbf{Contromisura adottata} \\
            \hline
            \endhead
            
            \hline
            \multicolumn{2}{r}{\textit{continua nella pagina successiva}} \\
            \endfoot
            
            \endlastfoot
            
            La mancanza di tracciabilità delle attività rende complicato l'avanzamento produttivo e la pianificazione del lavoro & Implementazione del sistema di ticketing su GitHub per migliorare la gestione e il monitoraggio delle attività \\ \hline
            L'assenza di controllo sulle modifiche dirette al branch principale può causare problemi di integrazione e qualità del codice & Implementazione della branch protection su GitHub per evitare modifiche non autorizzate e garantire la revisione del codice \\ \hline

            \rowcolor{white}
            \caption{Tabella con le contromisure adottate per migliorare l'organizzazione} 
            \label{tab:contromisure-problemi-organizzazione} 
            \end{longtable}
    }

    \subsection{Valutazione sugli strumenti di lavoro}{
         \rowcolors{2}{white}{lightblue}
            \begin{longtable}{|
                >{\centering\arraybackslash}p{.2\textwidth}|
                >{\centering\arraybackslash}p{.36\textwidth}|
                >{\centering\arraybackslash}p{.36\textwidth}|
            }
            \hline
            \rowcolor{gold}
            \textbf{Strumento} & \textbf{Descrizione problema} & \textbf{Contromisura adottata} \\
            \hline
            \endfirsthead
            
            \multicolumn{3}{c}%
            {\tablename\ \thetable\ -- \textit{continua dalla pagina precedente}} \\
            \hline
            \rowcolor{gold}
            \textbf{Strumento} & \textbf{Descrizione problema} & \textbf{Contromisura adottata} \\
            \hline
            \endhead
            
            \hline
            \multicolumn{3}{r}{\textit{continua nella pagina successiva}} \\
            \endfoot
            
            \endlastfoot
            
            GitHub & Difficoltà nell'organizzazione e nella gestione della documentazione & Uso combinato di Overleaf per i verbali e di un way of warking dettagliato e sempre consultabile per gli altri documenti \\ \hline
            LaTeX, Git, Python, FastAPI & Assenza di familiarità con queste tecnologie da parte di alcuni membri del gruppo con potenziali problemi di ritardo e disallineamento & Studio autonomo dei linguaggi e delle tecnologie non conosciute e confronto sui dubbi riscontrati \\ \hline
            Glossario & Assicurarsi manualmente che ogni occorrenza di una parola del glossario sia opportunamente marcata in ogni documento è un processo molto dispendioso & Implementazione e utilizzo di uno script Python per marcare ogni occorrenza dei termini del glossario su tutti i documenti completati\\ \hline
           
            \rowcolor{white}
            \caption{Tabella con le contromisure adottate per migliorare il rapporto con gli strumenti di lavoro} 
            \label{tab:contromisure-problemi-strumenti}
            \end{longtable}
    }

    \subsection{Valutazione sui ruoli}{
        \rowcolors{2}{white}{lightblue}
            \begin{longtable}{|
                >{\centering\arraybackslash}p{.2\textwidth}|
                >{\centering\arraybackslash}p{.36\textwidth}|
                >{\centering\arraybackslash}p{.36\textwidth}|
            }
            \hline
            \rowcolor{gold}
            \textbf{Ruolo} & \textbf{Descrizione problema} & \textbf{Contromisura adottata} \\
            \hline
            \endfirsthead
            
            \multicolumn{3}{c}%
            {\tablename\ \thetable\ -- \textit{continua dalla pagina precedente}} \\
            \hline
            \rowcolor{gold}
            \textbf{Ruolo} & \textbf{Descrizione problema} & \textbf{Contromisura adottata} \\
            \hline
            \endhead
            
            \hline
            \multicolumn{3}{r}{\textit{continua nella pagina successiva}} \\
            \endfoot
            
            \endlastfoot
            
            Responsabile & È difficile assegnare i compiti in modo equo, causando sovraccarico per alcuni e inattività per altri & Consultazione ad ogni riunione interna per l'individuazione e la suddivisione delle attività da svolgere \\ \hline
            Verificatore & L'attività di verifica è vincolante per garantire una rapida ed efficiente prosecuzione del progetto & Invio di una notifica ai verificatori al fine di svolgere il loro ruolo quanto prima \\ \hline
            Tutti & L'interruzione delle proprie attività per la presa in carico dei compiti del nuovo ruolo a cavallo di ogni sprint comporta un dispendio ulteriore di tempo & Comunicazione reciproca e coordinamento su quanto fatto al termine di ogni sprint \\ \hline

            \rowcolor{white}
            \caption{Tabella con le contromisure adottate per migliorare la gestione dei ruoli} 
            \label{tab:contromisure-problemi-ruoli} 
            \end{longtable}
         
    }

    \subsection{Considerazioni finali}{
         All'inizio del progetto, le valutazioni sul miglioramento erano piuttosto marginali in quanto il gruppo non aveva esperienza con i progetti e non era in grado di determinare se stava operando correttamente. Con il tempo, il numero di autocorrezioni è aumentato grazie a una maggiore presa di consapevolezza sulla complessità di gestione dei progetti software, e le relative contromisure adottate hanno a poco a poco permesso di risolvere o ridurre i problemi individuati, migliorando l'\textit{efficienza}\textsubscript{\href{https://atlasteam9.github.io/Atlas/glossario.html\#Efficienza}{G}} del lavoro svolto. Il team è quindi concorde nell'affermare che il processo di automiglioramento è un'attività fondamentale per garantire la qualità del progetto e si impegna di conseguenza a mantenere un approccio proattivo e collaborativo per individuare e risolvere tempestivamente eventuali criticità.
    }
}


\end{document}
  


