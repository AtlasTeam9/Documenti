\documentclass[a4paper,12pt]{article}

% ----------------------------
% Pacchetti utili
% ----------------------------
\usepackage[utf8]{inputenc}
\usepackage[T1]{fontenc}
\usepackage[italian]{babel}
\usepackage{graphicx}
\usepackage[table]{xcolor}
\definecolor{lightblue}{RGB}{225,240,255}
\usepackage{geometry}
\usepackage{setspace}
\usepackage{calc}
\usepackage{array}
\usepackage{fancyhdr}
\usepackage{tikz}
\usepackage{float}
\usepackage{pgf-pie}
\usepackage[colorlinks=true, linkcolor=blue, urlcolor=blue, citecolor=blue]{hyperref}

% ----------------------------
% Impostazioni pagina
% ----------------------------
\geometry{
    top=2cm,
    bottom=2cm,
    left=2cm,
    right=2cm
}

\setstretch{1.2}

% ----------------------------
% Dati personalizzabili
% ----------------------------
\newcommand{\Gruppo}{Atlas}
\newcommand{\Email}{\href{mailto:team9.atlas@gmail.com}{\textcolor{blue}{\underline{team9.atlas@gmail.com}}}}
\newcommand{\TitoloDocumento}{Piano di progetto}
\newcommand{\DataUltimaModifica}{23/11/2025}
\newcommand{\LogoGruppo}{img/AtlasLogo.png} % Inserisci il file del logo

% --- Nuove variabili aggiunte ---
\newcommand{\VersioneDocumento}{v0.3.0} % <-- modifica qui la versione o ID
\newcommand{\TipoDocumento}{Esterno} 

\pagestyle{fancy}
\fancyhf{}
\fancyhead[L]{\Gruppo}
\fancyhead[R]{Documento: \TipoDocumento}
\fancyfoot[C]{\thepage}

% larghezza della colonna dei nomi
\newlength{\namecol}
\setlength{\namecol}{4.5cm}

\newlength{\colw}
\setlength{\colw}{1.5cm}

\definecolor{mioverde}{RGB}{20,150,60}

% ----------------------------
% Inizio documento
% ----------------------------
\begin{document}

% ----------------------------
% Prima pagina
% ----------------------------
\begin{titlepage}

    \begin{center}

        % Logo
        \vspace*{0cm}
        \begin{tikzpicture}
            \clip (0,-0.1) circle (5.6cm);
            \node at (0,0) {\includegraphics[width=12cm]{\LogoGruppo}};
        \end{tikzpicture}\\[0.8cm]

        % Barra superiore
        \noindent\rule{\textwidth}{0.4pt}

        % Titolo
        \vspace{1cm}
        {\Huge \textbf{\TitoloDocumento}}\\[0.4cm]
        {\large Progetto di Ingegneria del Software A.A. 2025/2026}\\[0.8cm]
        {\large Versione: \VersioneDocumento}
        \vspace{1cm}

        % Barra inferiore
        \noindent\rule{\textwidth}{0.4pt}

    \end{center}

    % Informazioni in basso
    \vfill
    \noindent
    \begin{minipage}{0.5\textwidth}
        \raggedright
        \textbf{Autore:} \Gruppo\\
        \textbf{Ultima modifica:} \DataUltimaModifica
    \end{minipage}%
    \begin{minipage}{0.5\textwidth}
        \raggedleft
        \textbf{Tipo di documento:} \TipoDocumento
    \end{minipage}

\end{titlepage}



\section*{Registro delle modifiche}{
    \begin{center}
    \rowcolors{2}{lightblue}{white} % Alternanza automatica dal secondo row
        \begin{tabular}{|l|l|l|l|l|}
            \hline
            \textbf{Versione} & \textbf{Data} & \textbf{Autore} & \textbf{Verificatore} & \textbf{Descrizione} \\
            \hline
            \VersioneDocumento & \DataUltimaModifica & Andrea Difino & Federico Simonetto & Scrittura sez 1 e inizio 2\\
            \hline
            v0.2.0 & 17/11/2025 & Andrea Difino & Francesco Marcolongo & Scrittura sez 4.2\\
            \hline
            v0.1.0 & 16/11/2025 & Andrea Difino & Francesco Marcolongo & Prima stesura e sez. 4.1\\
            \hline
        \end{tabular}
    \end{center}
}

\newpage

\tableofcontents

\newpage
% ----------------------------
% Inizio contenuto verbale
% ----------------------------
\section{Introduzione}{
    \subsection{Piano di progetto}{
        Il Piano di progetto è un documento che registra le attività svolte e pianifica quelle da completare durante la realizzazione del progetto. Il suo scopo è fornire un consuntivo 
        periodico, analizzando i rischi incontrati dal team, il loro impatto (economico e non) e le misure adottate per il loro superamento. Vengono inoltre evidenziate le differenze 
        tra l'avanzamento previsto e quello effettivamente conseguito, con il relativo impatto sul preventivo. Il Piano di progetto include relazioni dettagliate per ogni sprint e 
        deve essere aggiornato continuamente dall'inizio alla fine del progetto.
    }
    \subsection{Glossario}{
        All'interno della documentazione prodotta dal team possono comparire termini suscettibili di incomprensioni o ambiguità. Per evitare questo, è disponibile un glossario 
        contenente i termini tecnici e le loro definizioni. Un termine è consultabile nel glossario se è indicato con la notazione \textbf{parola\textsubscript{\href{https://atlasteam9.github.io/Atlas/glossario.html}{G}}}.
        Premendo sulla G a pedice, l'utente verrà indirizzato alla pagina web del glossario.
    }
    \subsection{Capitolato}{
        Il capitolato ha come obiettivo la realizzazione di un'applicazione web che supporti le aziende nel processo di verifica di conformità ai requisiti di
        sicurezza informatica previsti dalla direttiva RED (Radio Equipment Directive). Tale verifica viene condotta tramite un sistema basato su alberi decisionali, 
        nei quali l'utente è guidato attraverso una sequenza di domande e condizioni per determinare il livello di aderenza ai requisiti della normativa EN18031. 
        L'applicazione dovrà fornire un'interfaccia chiara e intuitiva per compilare e navigare le domande, permettere l'esportazione dei risultati e garantire la tracciabilità 
        delle risposte. L'obiettivo finale è rendere il processo di valutazione della conformità più accessibile, ripetibile e automatizzato, riducendo errori umani e tempi di verifica.
    }
    \newpage
    \subsection{Riferimenti utili}{
        \subsubsection{Riferimenti normativi}{
            \begin{itemize}
                \item Riferimento al capitolato 1 dell'azienda proponente:\newline \textbf{Bluewind S.r.l - Automated EN18031 Compliance Verification}\newline \url{https://www.math.unipd.it/~tullio/IS-1/2025/Progetto/C1.pdf}
            \end{itemize}
        }
        \subsubsection{Riferimenti informativi}{
            \begin{itemize}
                \item Riferimento alle slide del corso di Ingegneria del Software:\newline \textbf{Regolamento del progetto didattico}\newline \url{https://www.math.unipd.it/~tullio/IS-1/2025/Dispense/PD1.pdf}
                \item Riferimento alle slide del corso di Ingegneria del Software:\newline \textbf{Modelli di sviluppo software}\newline \url{https://www.math.unipd.it/~tullio/IS-1/2025/Dispense/T03.pdf}
                \item Riferimento alle slide del corso di Ingegneria del Software:\newline \textbf{Gestione di progetto}\newline \url{https://www.math.unipd.it/~tullio/IS-1/2025/Dispense/T04.pdf}
            \end{itemize}
        }
    }
}
\newpage

\section{Informazioni del progetto}{
    \subsection{Introduzione}{
        In questa sezione vengono riportate tutte le informazioni di tipo organizzativo-economico riguardanti il progetto.
    }
    \subsection{Date di consegna del progetto}{
        Il gruppo si impegna ad effettuare le consegne delle due milestone "Requirements and Technology Baseline" e "Product Baseline" entro le date riportate di seguito:
        \begin{center}
        \rowcolors{2}{lightblue}{white} % Alternanza automatica dal secondo row
            \begin{tabular}{|p{10cm}|p{4cm}|}
                \hline
                \textbf{Milestone} & \textbf{Data} \\
                \hline
                Requirements and Technology Baseline & 2026/??/?? \\
                \hline
                Product Baseline & 2026/03/20 \\
                \hline
            \end{tabular}
        \end{center}

    }
    \subsection{Costi del progetto}
        I costi del progetto sono riportati di seguito e sono soggetti a un limite prefissato, comunicato dal team all'azienda proponente prima dell'aggiudicazione degli appalti, e 
        non negoziabile.
        \begin{center}
        \rowcolors{2}{white}{lightblue}
        \begin{tabular}{|l|c|c|c|c|c|}
            \hline
            \textbf{Ruolo} & \textbf{Costo/h}(\texteuro) & \textbf{Ore Totali} & \textbf{Ore/Membro} & \textbf{Costo}(\texteuro) & \textbf{\%} \\
            \hline
            Responsabile & 30 & 56 & 8  & 1680 & 8.8 \\
            \hline
            Amministratore & 20 & 70 & 10 & 1400 & 11.0 \\
            \hline
            Analista & 25 & 84 & 12 & 2100 & 13.2 \\
            \hline
            Progettista & 25 & 112 & 16 & 2800 & 17.5 \\
            \hline
            Programmatore & 15 & 168 & 24 & 2520 & 26.4 \\
            \hline
            Verificatore & 15 & 147 & 21 & 2205 & 23.1 \\
            \hline
            \multicolumn{2}{|c|}{\textbf{Totale}} & \textbf{637} & \textbf{91} & \textbf{12\,705} & \textbf{100} \\
            \hline
        \end{tabular}
        \end{center}   

        \begin{figure}[H]
            \centering
            \begin{tikzpicture}
            \pie{8.8/Responsabile, 11.0/Amministratore, 13.2/Analista, 17.5/Progettista, 26.4/Programmatore, 23.1/Verificatore}
            \end{tikzpicture}
            \caption{Distribuzione delle ore per ruolo}
            \label{fig:piechart}
        \end{figure}  


    % CONTENUTO DELLA SUBSECTION FORSE RIPETUTO NELLA SEZ.4. POTREBBE ESSERE TOLTA

    \subsection{Definizione del modello di sviluppo scelto dal team}
        Il team ha deciso di adottare come modello di sviluppo Scrum, un framework di tipo Agile che consente la suddivisione del lavoro in piccoli incrementi a valore aggiunto.
        Il progetto viene organizzato in sprint bisettimanali, al termine dei quali il team deve sempre presentare un prodotto potenzialmente utilizzabile.

        Le principali motivazioni alla base della scelta di Scrum sono:

        \begin{itemize}
        \item \textbf{Flessibilità}: grazie all'adozione dello stesso framework da parte dell'azienda proponente, il team può condividere con i membri dell'azienda
            le retrospettive degli sprint, permettendo un allineamento costante dei progressi del team con le esigenze dello stakeholder e ricevendo feedback continuo dalla proponente.
        \item \textbf{Produttività e apprendimento}: alla fine di ogni sprint avviene una rotazione dei ruoli, permettendo a ciascun membro di apprendere le diverse funzioni del 
            progetto. Inoltre, il cambio rapido di ruolo favorisce un livello di produttività costante, grazie alle nuove sfide che ogni membro affronta ad ogni sprint.
        \item \textbf{Reattività}: sprint brevi consentono al team di reagire rapidamente a eventuali problemi o imprevisti che possono presentarsi durante lo sviluppo.
        \item \textbf{Feedback continui}: la cadenza bisettimanale degli sprint permette al team di ricevere costantemente feedback sui punti di forza e sulle criticità 
            del lavoro svolto. Le retrospettive degli sprint offrono un momento dedicato al controllo della qualità e al miglioramento dei processi.
        \end{itemize}
    
    \subsection{Introduzione struttura per descrizione periodi}{

    }
    \subsection{Struttura della pianificazione}{
        
    }
    \subsection{Struttura dell'esito effettivo}{
        
    }
}

\section{Gestione dei rischi}{
    \subsection{Introduzione}{

    }
    \subsection{Rischi individuati}{
        
    }
}

\section{Pianificazione nel breve termine}{
    \subsection{Introduzione}{
        Atlas ha scelto di adottare un approccio \textit{Agile} per la gestione del progetto, individuando in due settimane la durata più efficace per ottenere risultati significativi. Per questo motivo il lavoro viene suddiviso in sprint di circa due settimane.

        \

        \noindent All'inizio di ogni sprint il gruppo definisce le attività previste per il periodo successivo e procede alla rotazione dei ruoli. Tale rotazione può comunque avvenire anche durante lo sprint, qualora emergano necessità organizzative. Questa scelta permette ai membri del team di maturare esperienza in ogni ruolo e di individuare un ritmo di rotazione adeguato.

        \ 

        \noindent Il gruppo programma inoltre incontri con il proponente \textbf{Bluewind} in prossimità della conclusione degli sprint, così da raccogliere feedback tempestivi sulle funzionalità sviluppate.

        Le sezioni successive descrivono nel dettaglio i vari periodi di lavoro, riportando:
        \begin{itemize}
            \item Informazioni generali  
            \item Attività pianificate  
            \item Stima dei tempi e dei costi  
            \item Rischi previsti  
            \item Tempi e costi effettivi  
            \item Aggiornamento delle risorse rimanenti  
            \item Retrospettiva, comprensiva dei rischi riscontrati
        \end{itemize}
    }


    \subsection{Sprint 1 - RTB}{
        \subsubsection{Tempo}{
            \begin{itemize}
                \item[]Inizio:  \textbf{04/11/2025} 
                \item[]Fine prevista:  \textbf{17/11/2025} 
                \item[]Fine reale:  \textbf{17/11/2025} 
            \end{itemize}
        }

        \subsubsection{Informazioni generali e attività da svolgere}{
            Questo primo periodo ha l'obiettivo principale di risolvere i piccoli problemi sorti durante la candidatura; successivamente, avverrà la redazione dei documenti necessari per un buon inizio dei lavori. 
            
            \ 
            
            \noindent In particolare, le attività previste sono: • 
            
            \begin{itemize}
                \item Aggiornamento del sito web;
                \item Sistemazione del sistema di versionamento dei documenti con utilizzo della notazione MAJOR.MINOR.PATCH;
                \item Prima stesura del Way of Working interno;
                \item Prima redazione del Glossario;
                \item Prima redazione del Piano di Progetto;
                \item Stabilire un incontro con l'azienda proponente \textbf{Bluewind}.
            \end{itemize}
        }

        \subsubsection{Rischi previsti}{
            //TODO scrivere la sezione "Gestione dei rischi" 
        }

        \subsubsection{Preventivo}{
            Si prospetta l'utilizzo delle seguenti risorse:

            \begin{table}[H]
                \centering

                % intestazioni diagonali fuori dalla tabella
                \hspace*{-14cm}%
                \makebox[0pt][l]{%
                    \hspace*{\namecol}%
                    \rotatebox{60}{Responsabile}%
                    \hspace*{0.3cm}\rotatebox{60}{Amministratore}%
                    \hspace*{0.3cm}\rotatebox{60}{Analista}%
                    \hspace*{0.8cm}\rotatebox{60}{Progettista}%
                    \hspace*{0.6cm}\rotatebox{60}{Programmatore}%
                    \hspace*{0.3cm}\rotatebox{60}{Verificatore}%
                }

                \rowcolors{1}{lightblue}{white}

                \begin{tabular}{
                    |>{\raggedright\arraybackslash}m{\namecol}
                    |>{\centering\arraybackslash}m{\colw}
                    |>{\centering\arraybackslash}m{\colw}
                    |>{\centering\arraybackslash}m{\colw}
                    |>{\centering\arraybackslash}m{\colw}
                    |>{\centering\arraybackslash}m{\colw}
                    |>{\centering\arraybackslash}m{\colw}|}
                    \hline
                    Andrea Difino        & 8 & - & - & - & - & - \\
                    \hline
                    Giacomo Giora        & - & - & 2 & - & - & - \\
                    \hline
                    Francesco Marcolongo & - & - & - & - & - & 7 \\
                    \hline
                    Riccardo Valerio     & - & - & - & 4 & - & - \\
                    \hline
                    Michele Tesser       & - & 6 & - & - & - & - \\
                    \hline
                    Federico Simonetto   & - & - & - & 4 & - & - \\
                    \hline
                    Bilal Sabic          & - & - & 2 & - & - & - \\
                    \hline
                \end{tabular}
                \caption{Preventivo ore per ruolo - Sprint 1}
                \label{tab:preventivo-sprint1}
            \end{table}
        }

        \subsubsection{Consuntivo}{
            \begin{table}[H]
                \centering
                % intestazioni diagonali fuori dalla tabella
                \hspace*{-14cm}%
                \makebox[0pt][l]{%
                    \hspace*{\namecol}%
                    \rotatebox{60}{Responsabile}%
                    \hspace*{0.3cm}\rotatebox{60}{Amministratore}%
                    \hspace*{0.3cm}\rotatebox{60}{Analista}%
                    \hspace*{0.8cm}\rotatebox{60}{Progettista}%
                    \hspace*{0.6cm}\rotatebox{60}{Programmatore}%
                    \hspace*{0.3cm}\rotatebox{60}{Verificatore}%
                }

                \rowcolors{1}{lightblue}{white}

                \begin{tabular}{
                    |>{\raggedright\arraybackslash}m{\namecol}
                    |>{\centering\arraybackslash}m{\colw}
                    |>{\centering\arraybackslash}m{\colw}
                    |>{\centering\arraybackslash}m{\colw}
                    |>{\centering\arraybackslash}m{\colw}
                    |>{\centering\arraybackslash}m{\colw}
                    |>{\centering\arraybackslash}m{\colw}|}
                    \hline
                    Andrea Difino        & 8 & - & - & - & - & - \\
                    \hline
                    Giacomo Giora        & - & - & 3 \textcolor{red}{(+1)} & - & - & - \\
                    \hline
                    Francesco Marcolongo & - & - & - & - & - & 6 \textcolor{mioverde}{(-1)}\\
                    \hline
                    Riccardo Valerio     & - & - & - & 3 \textcolor{mioverde}{(-1)} & - & - \\
                    \hline
                    Michele Tesser       & - & 2 \textcolor{mioverde}{(-4)} & - & - & - & - \\
                    \hline
                    Federico Simonetto   & - & - & - & 3 \textcolor{mioverde}{(-1)} & - & - \\
                    \hline
                    Bilal Sabic          & - & - & 2 & - & - & - \\
                    \hline
                \end{tabular}
                \caption{Consuntivo ore per ruolo - Sprint 1}
                \label{tab:consuntivo-sprint1}
            \end{table}

            \begin{figure}[H]
                \centering
                \includegraphics[width=12cm]{img/OrePerRuolo-Sprint1.png}
                \caption{Confronto ore previste ed effettive per ruolo - Sprint 1}
                \label{fig:confronto-ore-sprint1}
            \end{figure}
        }
        
        \subsubsection{Risorse rimanenti aggiornate}{
            \begin{table}[h!]
                \centering

                \rowcolors{2}{lightblue}{white}

                \begin{tabular}{
                    |>{\raggedright\arraybackslash}p{3.2cm}
                    |>{\centering\arraybackslash}p{\colw}
                    |>{\centering\arraybackslash}p{\colw}
                    |>{\centering\arraybackslash}p{\colw}
                    |>{\centering\arraybackslash}p{2.9cm}
                    |>{\centering\arraybackslash}p{3.2cm}|}
                    \hline 
                    Ruolo & Costo & Ore & Costo & Ore rimanenti & Budget rimanenti \\
                    \hline
                    Responsabile         & 30\texteuro/h & 8 & 240\texteuro & 48 \textcolor{red}{(-8)}  &  1440\texteuro \textcolor{red}{(-240\texteuro)}\\
                    \hline
                    Amministratore       & 20\texteuro/h & 2 & 40\texteuro  & 68 \textcolor{red}{(-2)}  &  1400\texteuro \textcolor{red}{(-40\texteuro)}\\
                    \hline
                    Analista             & 25\texteuro/h & 5 & 125\texteuro & 79 \textcolor{red}{(-5)}  &  2100\texteuro \textcolor{red}{(-125\texteuro)}\\
                    \hline
                    Progettista          & 25\texteuro/h & 6 & 150\texteuro & 106 \textcolor{red}{(-6)} &  2800\texteuro \textcolor{red}{(-150\texteuro)}\\
                    \hline
                    Programmatore        & 15\texteuro/h & - & -            & 168                       &  2520\texteuro \\
                    \hline
                    Verificatore         & 15\texteuro/h & 6 & 90\texteuro  & 141 \textcolor{red}{(-6)} &  2205\texteuro \textcolor{red}{(-90\texteuro)}\\
                    \hline
                    Totale               & -             & 27& 645\texteuro & 610 \textcolor{red}{(-27)}&  12060\texteuro \textcolor{red}{(-645\texteuro)}\\
                    \hline
                \end{tabular}
                \caption{Variazione risorse disponibili - Sprint 1}
                \label{tab:risorse-rimanenti-sprint1}
            \end{table}
        }

        \subsubsection{Rischi incontrati}{
            //TODO necessitá la scrittura della sezione "Gestione dei rischi" 
        }

        \subsubsection{Retrospettiva}{
            Nel primo periodo ci si è concentrati sulla risoluzione di problemi iniziali e sulla preparazione dei documenti fondamentali per l'avvio del progetto, includendo aggiornamenti al sito, 
            organizzazione del versionamento, e prime stesure di linee guida, glossario e piano di progetto, 
            oltre alla pianificazione di un incontro con l'azienda proponente.
        }
    }
}

\end{document}
