\documentclass[a4paper,12pt]{article}

% ----------------------------
% Pacchetti utili
% ----------------------------
\usepackage[utf8]{inputenc}
\usepackage[T1]{fontenc}
\usepackage[italian]{babel}
\usepackage{graphicx}
\usepackage[table]{xcolor}
\definecolor{lightblue}{RGB}{225,240,255}
\usepackage{geometry}
\usepackage{setspace}
\usepackage{calc}
\usepackage{array}
\usepackage{fancyhdr}
\usepackage{tikz}
\usepackage[colorlinks=true, linkcolor=blue, urlcolor=blue, citecolor=blue]{hyperref}

% ----------------------------
% Impostazioni pagina
% ----------------------------
\geometry{
    top=2cm,
    bottom=2cm,
    left=2cm,
    right=2cm
}

\setstretch{1.2}

% ----------------------------
% Dati personalizzabili
% ----------------------------
\newcommand{\Gruppo}{Atlas}
\newcommand{\Email}{\href{mailto:team9.atlas@gmail.com}{\textcolor{blue}{\underline{team9.atlas@gmail.com}}}}
\newcommand{\TitoloDocumento}{Piano di progetto}
\newcommand{\DataUltimaModifica}{16/11/2025}
\newcommand{\LogoGruppo}{img/AtlasLogo.png} % Inserisci il file del logo

% --- Nuove variabili aggiunte ---
\newcommand{\VersioneDocumento}{v0.1.0} % <-- modifica qui la versione o ID
\newcommand{\TipoDocumento}{Esterno} 

\pagestyle{fancy}
\fancyhf{}
\fancyhead[L]{\Gruppo}
\fancyhead[R]{Documento: \TipoDocumento}
\fancyfoot[C]{\thepage}

% larghezza della colonna dei nomi
\newlength{\namecol}
\setlength{\namecol}{4.5cm}

\newlength{\colw}
\setlength{\colw}{1.5cm}

\definecolor{mioverde}{RGB}{20,150,60}

% ----------------------------
% Inizio documento
% ----------------------------
\begin{document}

% ----------------------------
% Prima pagina
% ----------------------------
\begin{titlepage}

    \begin{center}

        % Logo
        \vspace*{0cm}
        \begin{tikzpicture}
            \clip (0,-0.1) circle (5.6cm);
            \node at (0,0) {\includegraphics[width=12cm]{\LogoGruppo}};
        \end{tikzpicture}\\[0.8cm]

        % Barra superiore
        \noindent\rule{\textwidth}{0.4pt}

        % Titolo
        \vspace{1cm}
        {\Huge \textbf{\TitoloDocumento}}\\[0.4cm]
        {\large Progetto di Ingegneria del Software A.A. 2025/2026}\\[0.8cm]
        {\large Versione: \VersioneDocumento}
        \vspace{1cm}

        % Barra inferiore
        \noindent\rule{\textwidth}{0.4pt}

    \end{center}

    % Informazioni in basso
    \vfill
    \noindent
    \begin{minipage}{0.5\textwidth}
        \raggedright
        \textbf{Autore:} \Gruppo\\
        \textbf{Ultima modifica:} \DataUltimaModifica
    \end{minipage}%
    \begin{minipage}{0.5\textwidth}
        \raggedleft
        \textbf{Tipo di documento:} \TipoDocumento
    \end{minipage}

\end{titlepage}



\section*{Registro delle modifiche}{
    \begin{center}
    \rowcolors{2}{lightblue}{white} % Alternanza automatica dal secondo row
        \begin{tabular}{|l|l|l|l|l|}
            \hline
            \textbf{Versione} & \textbf{Data} & \textbf{Autore} & \textbf{Verificatore} & \textbf{Descrizione} \\
            \hline
            \VersioneDocumento & 16/11/2025 & Andrea Difino & Francesco Marcolongo & Prima stesura e sez. 4.1\\
            \hline
        \end{tabular}
    \end{center}
}

\newpage

\tableofcontents

\newpage
% ----------------------------
% Inizio contenuto verbale
% ----------------------------
\section{Introduzione}{
    \subsection{Glossario}{

    }
    \subsection{Capitolato}{
        
    }
    \subsection{Riferimenti utili}{
        
    }
}

\section{Informazioni del progetto}{
    \subsection{Introduzione}{

    }
    \subsection{Date di consegna del progetto}{
        
    }
    \subsection{Costi del progetto}{
        
    }
    \subsection{Introduzione struttura per descrizione periodi}{

    }
    \subsection{Struttura della pianificazione}{
        
    }
    \subsection{Struttura dell'esito effettivo}{
        
    }
}

\section{Gestione dei rischi}{
    \subsection{Introduzione}{

    }
    \subsection{Rischi individuati}{
        
    }
}

\section{Pianificazione nel breve termine}{
    \subsection{Introduzione}{
        Atlas ha scelto di adottare un approccio \textit{Agile} per la gestione del progetto, individuando in due settimane la durata più efficace per ottenere risultati significativi. Per questo motivo il lavoro viene suddiviso in sprint di circa due settimane.

        \

        \noindent All'inizio di ogni sprint il gruppo definisce le attività previste per il periodo successivo e procede alla rotazione dei ruoli. Tale rotazione può comunque avvenire anche durante lo sprint, qualora emergano necessità organizzative. Questa scelta permette ai membri del team di maturare esperienza in ogni ruolo e di individuare un ritmo di rotazione adeguato.

        \ 

        \noindent Il gruppo programma inoltre incontri con il proponente \textbf{Bluewind} in prossimità della conclusione degli sprint, così da raccogliere feedback tempestivi sulle funzionalità sviluppate.

        Le sezioni successive descrivono nel dettaglio i vari periodi di lavoro, riportando:
        \begin{itemize}
            \item Informazioni generali  
            \item Attività pianificate  
            \item Stima dei tempi e dei costi  
            \item Rischi previsti  
            \item Tempi e costi effettivi  
            \item Aggiornamento delle risorse rimanenti  
            \item Retrospettiva, comprensiva dei rischi riscontrati
        \end{itemize}
    }


    \subsection{Sprint 1 - RTB}{
        \subsubsection{Tempo}{
            \begin{itemize}
                \item[]Inizio:  \textbf{04/11/2025} 
                \item[]Fine prevista:  \textbf{17/11/2025} 
                \item[]Fine reale:  \textbf{17/11/2025} 
            \end{itemize}
        }

        \subsubsection{Informazioni generali e attività da svolgere}{
            Questo primo periodo ha l'obiettivo principale di risolvere i piccoli problemi sorti durante la candidatura; successivamente, avverrà la redazione dei documenti necessari per un buon inizio dei lavori. 
            
            \ 
            
            \noindent In particolare, le attività previste sono: • 
            
            \begin{itemize}
                \item Aggiornamento del sito web;
                \item Sistemazione del sistema di versionamento dei documenti con utilizzo della notazione MAJOR.MINOR.PATCH;
                \item Prima stesura del Way of Working interno;
                \item Prima redazione del Glossario;
                \item Prima redazione del Piano di Progetto;
                \item Stabilire un incontro con l'azienda proponente \textbf{Bluewind}.
            \end{itemize}
        }

        \subsubsection{Rischi previsti}{
            //TODO scrivere la sezione "Gestione dei rischi" 
        }

        \subsubsection{Preventivo}{
            Si prospetta l'utilizzo delle seguenti risorse:

            \begin{table}[h!]
                \centering

                % intestazioni diagonali fuori dalla tabella
                \hspace*{-14cm}%
                \makebox[0pt][l]{%
                    \hspace*{\namecol}%
                    \rotatebox{60}{Responsabile}%
                    \hspace*{0.3cm}\rotatebox{60}{Amministratore}%
                    \hspace*{0.3cm}\rotatebox{60}{Analista}%
                    \hspace*{0.8cm}\rotatebox{60}{Progettista}%
                    \hspace*{0.6cm}\rotatebox{60}{Programmatore}%
                    \hspace*{0.3cm}\rotatebox{60}{Verificatore}%
                }

                \rowcolors{2}{lightblue}{white}

                \begin{tabular}{
                    |>{\raggedright\arraybackslash}m{\namecol}
                    |>{\centering\arraybackslash}m{\colw}
                    |>{\centering\arraybackslash}m{\colw}
                    |>{\centering\arraybackslash}m{\colw}
                    |>{\centering\arraybackslash}m{\colw}
                    |>{\centering\arraybackslash}m{\colw}
                    |>{\centering\arraybackslash}m{\colw}|}
                    \hline
                    Andrea Difino        & 8 & - & - & - & - & - \\
                    \hline
                    Giacomo Giora        & - & - & 2 & - & - & - \\
                    \hline
                    Francesco Marcolongo & - & - & - & - & - & 7 \\
                    \hline
                    Riccardo Valerio     & - & - & - & 4 & - & - \\
                    \hline
                    Michele Tesser       & - & 6 & - & - & - & - \\
                    \hline
                    Federico Simonetto   & - & - & - & 4 & - & - \\
                    \hline
                    Bilal Sabic          & - & - & 2 & - & - & - \\
                    \hline
                \end{tabular}
                \caption{Preventivo ore per ruolo - Sprint 1}
                \label{tab:preventivo-sprint1}
            \end{table}
        }

        \subsubsection{Consuntivo}{
            Si prospetta l'utilizzo delle seguenti risorse:

            \begin{table}[h!]
                \centering
                % intestazioni diagonali fuori dalla tabella
                \hspace*{-14cm}%
                \makebox[0pt][l]{%
                    \hspace*{\namecol}%
                    \rotatebox{60}{Responsabile}%
                    \hspace*{0.3cm}\rotatebox{60}{Amministratore}%
                    \hspace*{0.3cm}\rotatebox{60}{Analista}%
                    \hspace*{0.8cm}\rotatebox{60}{Progettista}%
                    \hspace*{0.6cm}\rotatebox{60}{Programmatore}%
                    \hspace*{0.3cm}\rotatebox{60}{Verificatore}%
                }

                \rowcolors{2}{lightblue}{white}

                \begin{tabular}{
                    |>{\raggedright\arraybackslash}m{\namecol}
                    |>{\centering\arraybackslash}m{\colw}
                    |>{\centering\arraybackslash}m{\colw}
                    |>{\centering\arraybackslash}m{\colw}
                    |>{\centering\arraybackslash}m{\colw}
                    |>{\centering\arraybackslash}m{\colw}
                    |>{\centering\arraybackslash}m{\colw}|}
                    \hline
                    Andrea Difino        & 8 & - & - & - & - & - \\
                    \hline
                    Giacomo Giora        & - & - & 3 \textcolor{red}{(+1)} & - & - & - \\
                    \hline
                    Francesco Marcolongo & - & - & - & - & - & 6 \textcolor{mioverde}{(-1)}\\
                    \hline
                    Riccardo Valerio     & - & - & - & 3 \textcolor{mioverde}{(-1)} & - & - \\
                    \hline
                    Michele Tesser       & - & 2 \textcolor{mioverde}{(-4)} & - & - & - & - \\
                    \hline
                    Federico Simonetto   & - & - & - & 3 \textcolor{mioverde}{(-1)} & - & - \\
                    \hline
                    Bilal Sabic          & - & - & 2 & - & - & - \\
                    \hline
                \end{tabular}
                \caption{Consuntivo ore per ruolo - Sprint 1}
                \label{tab:consuntivo-sprint1}
            \end{table}
        }
        
        \begin{figure}[h!]
            \centering
            \includegraphics[width=12cm]{img/OrePerRuolo-Sprint1.png}
            \caption{Confronto ore previste ed effettive per ruolo - Sprint 1}
            \label{fig:confronto-ore-sprint1}
        \end{figure}
    }
}

\end{document}
