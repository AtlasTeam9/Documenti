\documentclass[a4paper,12pt]{article}

% ----------------------------
% Pacchetti utili
% ----------------------------
\usepackage[utf8]{inputenc}
\usepackage[T1]{fontenc}
\usepackage[italian]{babel}
\usepackage{graphicx}
\usepackage[table]{xcolor}
\definecolor{lightblue}{RGB}{225,240,255}
\definecolor{gold}{RGB}{255,215,0}
\usepackage{geometry}
\usepackage{setspace}
\usepackage{calc}
\usepackage{array}
\usepackage{fancyhdr}
\usepackage{tikz}
\usepackage{float}
\usepackage{pgf-pie}
\usepackage[colorlinks=true, linkcolor=blue, urlcolor=blue, citecolor=blue]{hyperref}
\usepackage{caption}

% ----------------------------
% Impostazioni pagina
% ----------------------------
\geometry{
    top=2cm,
    bottom=2cm,
    left=2cm,
    right=2cm
}

\setstretch{1.2}


% ----------------------------
% Dati personalizzabili
% ----------------------------
\newcommand{\Gruppo}{Atlas}
\newcommand{\Email}{\href{mailto:team9.atlas@gmail.com}{\textcolor{blue}{\underline{team9.atlas@gmail.com}}}}
\newcommand{\TitoloDocumento}{Analisi dei Requisiti}
\newcommand{\DataUltimaModifica}{2025/11/27}
\newcommand{\LogoGruppo}{img/AtlasLogo.png} 

% --- Nuove variabili aggiunte ---
\newcommand{\VersioneDocumento}{v0.2.0} % <-- modifica qui la versione o ID
\newcommand{\TipoDocumento}{Esterno} 

\pagestyle{fancy}
\fancyhf{}
\fancyhead[L]{\Gruppo}
\fancyhead[R]{Documento: \TipoDocumento}
\fancyfoot[C]{\thepage}

% larghezza della colonna dei nomi
\newlength{\namecol}
\setlength{\namecol}{4.5cm}

\newlength{\colw}
\setlength{\colw}{1.5cm}

\definecolor{mioverde}{RGB}{20,150,60}

\setcounter{secnumdepth}{4}
\setcounter{tocdepth}{4}

% ----------------------------
% Inizio documento
% ----------------------------
\begin{document}

% ----------------------------
% Prima pagina
% ----------------------------
\begin{titlepage}

    \begin{center}

        % Logo
        \vspace*{0cm}
        \begin{tikzpicture}
            \clip (0,-0.1) circle (5.6cm);
            \node at (0,0) {\includegraphics[width=12cm]{\LogoGruppo}};
        \end{tikzpicture}\\[0.8cm]

        % Barra superiore
        \noindent\rule{\textwidth}{0.4pt}

        % Titolo
        \vspace{1cm}
        {\Huge \textbf{\TitoloDocumento}}\\[0.4cm]
        {\large Progetto di Ingegneria del Software A.A. 2025/2026}\\[0.8cm]
        {\large Versione: \VersioneDocumento}
        \vspace{1cm}

        % Barra inferiore
        \noindent\rule{\textwidth}{0.4pt}

    \end{center}

    % Informazioni in basso
    \vfill
    \noindent
    \begin{minipage}{0.5\textwidth}
        \raggedright
        \textbf{Autore:} \Gruppo\\
        \textbf{Ultima modifica:} \DataUltimaModifica
    \end{minipage}%
    \begin{minipage}{0.5\textwidth}
        \raggedleft
        \textbf{Tipo di documento:} \TipoDocumento
    \end{minipage}

\end{titlepage}


\section*{Registro delle modifiche}{
    \begin{center}
    \rowcolors{2}{lightblue}{white}
        \begin{tabular}{|l|l|l|l|l|}
            \hline
            \rowcolor{lightgray}
            \textbf{Versione} & \textbf{Data} & \textbf{Autore} & \textbf{Verificatore} & \textbf{Descrizione} \\
            \hline
            \VersioneDocumento & \DataUltimaModifica & & Federico Simonetto & Prima stesura prima sezione\\
            \hline
            v0.1.0 & 2025/11/27 & Giacomo Giora &  & Stesura template\\
            \hline
        \end{tabular}
    \end{center}
}

\newpage

\tableofcontents

\newpage

\listoffigures

\newpage

\listoftables

\newpage
% ----------------------------
% Inizio contenuto
% ----------------------------
\section{Introduzione}{
    \subsection{Scopo del documento}{
        Il presente documento di \textit{Analisi dei requisiti} ha l'obiettivo di definire in modo chiaro, completo e verificabile l'insieme dei requisiti funzionali e non funzionali del prodotto software che il team
        di sviluppo realizzerà.
        \newline
        I requisiti presenti nel documento saranno classificati secondo il seguente livello di priorità:
        \begin{itemize}
            \item \textbf{Obbligatorio:} requisito indispensabile, esplicitamente richiesto dallo stakeholder;
            \item \textbf{Desiderabile:} requisito non essenziale, ma in grado di apportare un valore aggiunto riconoscibile;
            \item \textbf{Opzionale:} requisito di importanza secondaria, la cui implementazione può essere rimandata o valutata in base a tempi e risorse disponibili.
        \end{itemize}   
        Lo scopo dell'analisi dei requisiti non è descrivere le soluzioni tecniche adottate per l'implementazione, bensì definire in maniera concettuale le funzionalità del sistema. 
        In particolare, il documento si concentra su cosa il sistema deve fare, procedendo in modo gerarchico dall'esterno del sistema (utente e contesto) verso i suoi componenti interni.
        \newline
        In particolare questo documento cerca di garantire le seguenti qualità:
        \begin{itemize}
            \item \textbf{Completezza:} tutti i requisiti rilevanti sono identificati e descritti in maniera esaustiva;
            \item \textbf{Chiarezza:} ogni requisito è formulato in maniera comprensibile e non ambigua, anche grazie all'uso di strumenti di modellazione semiformali, 
                quali diagrammi e user story.
            \item \textbf{Coerenza:} i requisiti non si contraddicono fra loro e sono uniformi nella terminologia utilizzata;
            \item \textbf{Verificabilità:} ciascun requisito è formulato in modo tale da poter essere verificato tramite test o validazione;
            \item \textbf{Tracciabilità:} ogni requisito è identificato e può essere ricondotto ad una specifica esigenza dello stakeholder;
            \item \textbf{Modificabilità:} la struttura del documento consente di apportare modifiche senza comprometterne la coerenza complessiva.
        \end{itemize}  
    }

    \subsection{Glossario}{
        All'interno della documentazione prodotta dal team possono comparire termini suscettibili di incomprensioni o ambiguità. Per evitare questo, è disponibile un glossario 
        contenente i termini tecnici e le loro definizioni. Un termine è consultabile nel glossario se è indicato con la notazione \textbf{parola\textsubscript{\href{https://atlasteam9.github.io/Atlas/glossario.html}{G}}}.
        Premendo sulla G a pedice, l'utente verrà indirizzato alla pagina web del glossario.
    }

    \subsection{Riferimenti}{
        \subsubsection{Riferimenti normativi}{
            \begin{itemize}
                \item Riferimento al capitolato 1 dell'azienda proponente:\newline \textbf{Bluewind S.r.l - Automated EN18031 Compliance Verification}\newline \url{https://www.math.unipd.it/~tullio/IS-1/2025/Progetto/C1.pdf}
            \end{itemize}
        }

        \subsubsection{Riferimenti informativi}{
            \begin{itemize}
                \item Riferimento alle slide del corso di Ingegneria del Software:\newline \textbf{Regolamento del progetto didattico}\newline \url{https://www.math.unipd.it/~tullio/IS-1/2025/Dispense/PD1.pdf}
                \item Riferimento alle slide del corso di Ingegneria del Software:\newline \textbf{Gestione di progetto}\newline \url{https://www.math.unipd.it/~tullio/IS-1/2025/Dispense/T04.pdf}
                \item Riferimento alle slide del corso di Ingegneria del Software:\newline \textbf{Analisi dei requisiti}\newline \url{https://www.math.unipd.it/~tullio/IS-1/2025/Dispense/T05.pdf}
            \end{itemize}
        }
    }
}

\newpage

\section{Descizione generale}{
    \subsection{Obiettivi del prodotto}{
    }

    \subsection{Funzioni del prodotto}{
    }

    \subsection{Caratteristiche utente}{
    }

    \subsection{Tecnologie e struttura del prodotto}{
    }
}

\newpage
% ----------------------------
% Casi d'uso
% ----------------------------
\section{Casi d'uso}{
    \subsection{Obiettivi}{
    }

    \subsection{Attori}{
    }

    \subsection{Elenco casi d'uso}{

        \subsubsection{UC1 - Titolo caso d'uso}
        \label{uc:uc1}
        Testo del caso d'uso padre...

            \paragraph{UC1.1 - Titolo sottocaso d'uso} \mbox{} \\
            Testo del sottocaso d'uso... 

            \paragraph{UC1.2 - Titolo sottocaso d'uso} \mbox{} \\
            Testo del sottocaso d'uso...

        \subsubsection{UC2 - Titolo caso d'uso}
        \label{uc:uc2}
        Testo del caso d'uso padre...

            \paragraph{UC2.1 - Titolo sottocaso d'uso} \mbox{} \\
            Testo del sottocaso d'uso...

            \paragraph{UC2.2 - Titolo sottocaso d'uso} \mbox{} \\
            Testo del sottocaso d'uso...
    }
}

\newpage
% ----------------------------
% Requisiti
% ----------------------------
\section{Requisiti}{
    \subsection{Requisiti funzionali}{
        Scrivere i requisiti funzionali qui. I requisiti funzionali sono: le funzionalità che il sistema deve offrire agli utenti finali per soddisfare le loro esigenze e aspettative.
        
        \begin{table}[H]
        \centering
        \rowcolors{2}{lightblue}{white}
            \begin{tabular}{|l|l|l|l|}
                \hline
                \rowcolor{gold}
                \textbf{Codice} & \textbf{Rilevanza} & \textbf{Descrizione} & \textbf{Fonti} \\
                \hline
                Id requisito & Obbligatorio/Desiderabile/Opzionale & Descrizione & \hyperref[uc:uc1]{UC1} \\
                \hline
                Id requisito & Obbligatorio/Desiderabile/Opzionale & Descrizione & \hyperref[uc:uc1]{UC1}, \hyperref[uc:uc2]{UC2} \\
                \hline
            \end{tabular}
            \caption{Tabella dei requisiti funzionali} 
            \label{tab:requisiti-funzionali}
        \end{table}
    }

    \subsection{Requisiti di qualità}{
        Scrivere i requisiti di qualità qui. I requisiti di qualità sono: le caratteristiche non funzionali che il sistema deve possedere, come prestazioni, usabilità, affidabilità, sicurezza, ecc.

        \begin{table}[H]
        \centering
        \rowcolors{2}{lightblue}{white}
            \begin{tabular}{|l|l|l|l|}
                \hline
                \rowcolor{gold}
                \textbf{Codice} & \textbf{Rilevanza} & \textbf{Descrizione} & \textbf{Fonti} \\
                \hline
                Id requisito & Obbligatorio/Desiderabile/Opzionale & Descrizione & \hyperref[uc:uc1]{UC1} \\
                \hline
                Id requisito & Obbligatorio/Desiderabile/Opzionale & Descrizione & \hyperref[uc:uc1]{UC1}, \hyperref[uc:uc2]{UC2} \\
                \hline
            \end{tabular}
            \caption{Tabella dei requisiti di qualità} 
            \label{tab:requisiti-qualita}
        \end{table}
    }

    \subsection{Requisiti di vincolo}{
        Scrivere i requisiti di vincolo qui. I requisiti di vincolo sono: le limitazioni o condizioni imposte al sistema, come vincoli tecnologici, normativi, di budget, di tempo, ecc.

        \begin{table}[H]
        \centering
        \rowcolors{2}{lightblue}{white}
            \begin{tabular}{|l|l|l|l|}
                \hline
                \rowcolor{gold}
                \textbf{Codice} & \textbf{Rilevanza} & \textbf{Descrizione} & \textbf{Fonti} \\
                \hline
                Id requisito & Obbligatorio/Desiderabile/Opzionale & Descrizione & \hyperref[uc:uc1]{UC1} \\
                \hline
                Id requisito & Obbligatorio/Desiderabile/Opzionale & Descrizione & \hyperref[uc:uc1]{UC1}, \hyperref[uc:uc2]{UC2} \\
                \hline
            \end{tabular}
            \caption{Tabella dei requisiti di vincolo} 
            \label{tab:requisiti-vincolo}
        \end{table}
    }
}

\newpage

\section{Tracciamento dei Requisiti}{
    \begin{table}[H]
    \centering
    \rowcolors{2}{lightblue}{white}
        \begin{tabular}{|p{5cm}|p{5cm}|}
            \hline
            \rowcolor{gold}
            \textbf{Fonte} & \textbf{Requisito} \\
            \hline
            Capitolato & Id requisito \\
            \hline
            UC1 & Id requisito \\
            \hline
        \end{tabular}
        \caption{Tabella di tracciamento dei requisiti} 
        \label{tab:tracciamento-requisiti}
    \end{table}

    \subsection{Riepilogo}{
        \begin{table}[H]
        \centering
        \rowcolors{2}{lightblue}{white}
            \begin{tabular}{|l|l|l|l|l|}
                \hline
                \rowcolor{gold}
                \textbf{Tipologia} & \textbf{Obbligatori} & \textbf{Desiderabili} & \textbf{Opzionali} & \textbf{Totali} \\
                \hline
                Funzionali & XX & XX & XX & XX \\
                \hline
                Qualità & XX & XX & XX & XX \\
                \hline
                Vincolo & XX & XX & XX & XX \\
                \hline
                \textbf{Totali} & \textbf{XX} & \textbf{XX} & \textbf{XX} & \textbf{XX} \\
                \hline
            \end{tabular}
            \caption{Riepilogo dei requisiti} 
            \label{tab:riepilogo-requisiti}
        \end{table}
    }

    \subsection{Conclusioni}{
        I requisiti individuati sono soggetti a possibili variazioni durante l'evoluzione del progetto, al fine di apportare miglioramenti e aggiornamenti. Nel corso del ciclo di vita del progetto sarà valutata l'opportunità di integrare ulteriori requisiti per elevare la qualità complessiva del prodotto. Tali modifiche verranno considerate progressivamente, seguendo un approccio di miglioramento continuo.
    }
}

\end{document}
  

