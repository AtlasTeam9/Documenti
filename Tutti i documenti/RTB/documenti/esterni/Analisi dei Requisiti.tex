\documentclass[a4paper,12pt]{article}

% ----------------------------
% Pacchetti utili
% ----------------------------
\usepackage[utf8]{inputenc}
\usepackage[T1]{fontenc}
\usepackage[italian]{babel}
\usepackage{graphicx}
\usepackage{tabularx}
\usepackage{longtable}
\usepackage{ltablex}
\usepackage{xcolor}
\usepackage{amssymb}
\usepackage[table]{xcolor}
\definecolor{lightblue}{RGB}{225,240,255}
\definecolor{gold}{RGB}{255,215,0}
\usepackage{geometry}
\usepackage{setspace}
\usepackage{calc}
\usepackage{array}
\usepackage{fancyhdr}
\usepackage{tikz}
\usepackage{float}
\usepackage{pgf-pie}
\usepackage[colorlinks=true, linkcolor=blue, urlcolor=blue, citecolor=blue]{hyperref}
\usepackage{caption}

% ----------------------------
% Impostazioni pagina
% ----------------------------
\geometry{
    top=2cm,
    bottom=2cm,
    left=2cm,
    right=2cm
}

\setstretch{1.2}


% ----------------------------
% Dati personalizzabili
% ----------------------------
\newcommand{\Gruppo}{Atlas}
\newcommand{\Email}{\href{mailto:team9.atlas@gmail.com}{\textcolor{blue}{\underline{team9.atlas@gmail.com}}}}
\newcommand{\TitoloDocumento}{Analisi dei Requisiti}
\newcommand{\DataUltimaModifica}{2026/01/12}
\newcommand{\LogoGruppo}{../../../Assets/AtlasLogo.png} 

% --- Nuove variabili aggiunte ---
\newcommand{\VersioneDocumento}{v0.8.4} % <-- modifica qui la versione o ID
\newcommand{\TipoDocumento}{Esterno} 

\pagestyle{fancy}
\fancyhf{}
\fancyhead[L]{\Gruppo}
\fancyhead[R]{Documento: \TipoDocumento}
\fancyfoot[C]{\thepage}

% larghezza della colonna dei nomi
\newlength{\namecol}
\setlength{\namecol}{4.5cm}

\newlength{\colw}
\setlength{\colw}{1.5cm}

\definecolor{mioverde}{RGB}{20,150,60}

\setcounter{secnumdepth}{4}
\setcounter{tocdepth}{4}

% ----------------------------
% Inizio documento
% ----------------------------
\begin{document}

% ----------------------------
% Prima pagina
% ----------------------------
\begin{titlepage}

    \begin{center}

        % Logo
        \vspace*{0cm}
        \begin{tikzpicture}
            \clip (0,-0.1) circle (5.6cm);
            \node at (0,0) {\includegraphics[width=12cm]{\LogoGruppo}};
        \end{tikzpicture}\\[0.8cm]

        % Barra superiore
        \noindent\rule{\textwidth}{0.4pt}

        % Titolo
        \vspace{1cm}
        {\Huge \textbf{\TitoloDocumento}}\\[0.4cm]
        {\large Progetto di Ingegneria del Software A.A. 2025/2026}\\[0.8cm]
        {\large Versione: \VersioneDocumento}
        \vspace{1cm}

        % Barra inferiore
        \noindent\rule{\textwidth}{0.4pt}

    \end{center}

    % Informazioni in basso
    \vfill
    \noindent
    \begin{minipage}{0.5\textwidth}
        \raggedright
        \textbf{Autore:} \Gruppo\\
        \textbf{Ultima modifica:} \DataUltimaModifica
    \end{minipage}%
    \begin{minipage}{0.5\textwidth}
        \raggedleft
        \textbf{Tipo di documento:} \TipoDocumento
    \end{minipage}

\end{titlepage}


\section*{Registro delle modifiche}
    \begin{center}
    \rowcolors{2}{lightblue}{white} 
        \begin{tabularx}{\textwidth}{|l|l|l|X|X|}
            \hline
            \rowcolor{lightgray}
            \textbf{Versione} & \textbf{Data} & \textbf{Autore} & \textbf{Verificatore} & \textbf{Descrizione} \\
            \hline
            \VersioneDocumento & \DataUltimaModifica & Michele Tesser & Andrea Difino & Continuati requisiti e aggiunti UC \\
            \hline
            v0.8.3 & 2026/01/11 & Riccardo Valerio & Andrea Difino & Iniziati requisiti funzionali da 26 a 41 \\
            \hline
            v0.8.2 & 2026/01/10 & Michele Tesser & Andrea Difino & Iniziati requisiti funzionali da 14 a 25 \\
            \hline
            v0.8.1 & 2026/01/08 & Riccardo Valerio & Andrea Difino & Iniziati requisiti funzionali da 1 a 13 \\
            \hline
            v0.8.0 & 2026/01/07 & Michele Tesser & Andrea Difino & Completati Casi d'Uso \\
            \hline
            v0.7.0 & 2026/01/06 & Giacomo Giora & Andrea Difino & Modificati e sistemati UC da 1 a 12. Aggiunti da UC 13 a 15 \\
            \hline
            v0.6.0 & 2025/12/19 & Andrea Difino & Giacomo Giora & Aggiunti e sistemati UC da 1 a 12 \\
            \hline
            v0.5.2 & 2025/12/15 & Riccardo Valerio & Giacomo Giora, Federico Simonetto & Migliorata coerenza \\
            \hline
            v0.5.1 & 2025/12/07 & Michele Tesser & Riccardo Valerio & Link glossario \\
            \hline
            v0.5.0 & 2025/12/06 & Andrea Difino & Bilal Sabic & Aggiunte estensioni UC1 \\
            \hline
            v0.4.0 & 2025/12/04 & Andrea Difino & Bilal Sabic & Migliorata sez 2 e primo caso d'uso \\
            \hline
            v0.3.0 & 2025/11/28 & Michele Tesser & Federico Simonetto & Stesura sez 2 e inizio 3\\
            \hline
            v0.2.0 & 2025/11/27 & Giacomo Giora & Federico Simonetto & Prima stesura prima sezione\\
            \hline
            v0.1.0 & 2025/11/27 & Giacomo Giora & & Stesura template\\
            \hline
        \end{tabularx}
    \end{center}


\newpage

\tableofcontents

\newpage

\listoffigures

\newpage

\listoftables

\newpage
% ----------------------------
% Inizio contenuto verbale
% ----------------------------
\section{Introduzione}{
    \subsection{Scopo del documento}{
        Il presente documento di \textit{Analisi dei requisiti} ha l'obiettivo di definire in modo chiaro, completo e verificabile l'insieme dei requisiti funzionali e non funzionali che il team \textit{Atlas} ha individuato nel corso dello sviluppo del progetto \textit{Automated EN18031 Compliance Verification}.
        \newline
        A tal fine, il documento include una descrizione approfondita dei Casi d'Uso, da cui derivano i requisiti elencati. Tali Casi d'Uso sono illustrati tramite diagrammi che utilizzano la notazione UML, per formalizzarne la descrizione.
        \newline
        I requisiti presenti nel documento saranno classificati secondo il seguente livello di priorità:
        \begin{itemize}
            \item \textbf{Obbligatorio:} requisito indispensabile, esplicitamente richiesto dallo stakeholder;
            \item \textbf{Desiderabile:} requisito non essenziale, ma in grado di apportare un valore aggiunto riconoscibile;
            \item \textbf{Opzionale:} requisito di importanza secondaria, la cui implementazione può essere rimandata o valutata in base a tempi e risorse disponibili.
        \end{itemize}   
        Lo scopo dell'analisi dei requisiti non è descrivere le soluzioni tecniche adottate per l'implementazione, bensì definire in maniera concettuale le funzionalità del sistema. 
        In particolare, il documento si concentra su cosa il sistema deve fare, procedendo in modo gerarchico dall'esterno del sistema (utente e contesto) verso i suoi componenti interni.
        \newline
        In particolare questo documento cerca di garantire le seguenti qualità:
        \begin{itemize}
            \item \textbf{Completezza:} tutti i requisiti rilevanti sono identificati e descritti in maniera esaustiva;
            \item \textbf{Chiarezza:} ogni requisito è formulato in maniera comprensibile e non ambigua, anche grazie all'uso di strumenti di modellazione semiformali, 
                quali diagrammi e user story.
            \item \textbf{Coerenza:} i requisiti non si contraddicono fra loro e sono uniformi nella terminologia utilizzata;
            \item \textbf{Verificabilità:} ciascun requisito è formulato in modo tale da poter essere verificato tramite test o validazione;
            \item \textbf{Tracciabilità:} ogni requisito è identificato e può essere ricondotto ad una specifica esigenza dello stakeholder;
            \item \textbf{Modificabilità:} la struttura del documento consente di apportare modifiche senza comprometterne la coerenza complessiva.
        \end{itemize}  
    }

    \subsection{Glossario}{
        All'interno della documentazione prodotta dal team possono comparire termini suscettibili di incomprensioni o ambiguità. Per evitare questo, è disponibile un glossario 
        contenente i termini tecnici e le loro definizioni. Un termine è consultabile nel glossario se è indicato con la notazione \textit{parola\textsubscript{\href{https://atlasteam9.github.io/Atlas/glossario.html}{G}}}.
        Premendo sulla G a pedice, l'utente verrà indirizzato alla pagina web del glossario.
    }

    \subsection{Riferimenti}{
        \subsubsection{Riferimenti normativi}{
            \begin{itemize}
                \item Riferimento al capitolato 1 dell'azienda proponente:\newline \textbf{Bluewind S.r.l - Automated EN18031 Compliance Verification}\newline \url{https://www.math.unipd.it/~tullio/IS-1/2025/Progetto/C1.pdf}
            \end{itemize}
        }

        \subsubsection{Riferimenti informativi}{
            \begin{itemize}
                \item Riferimento alle slide del corso di Ingegneria del Software:\newline \textbf{Regolamento del progetto didattico}\newline \url{https://www.math.unipd.it/~tullio/IS-1/2025/Dispense/PD1.pdf}
                \item Riferimento alle slide del corso di Ingegneria del Software:\newline \textbf{Gestione di progetto}\newline \url{https://www.math.unipd.it/~tullio/IS-1/2025/Dispense/T04.pdf}
                \item Riferimento alle slide del corso di Ingegneria del Software:\newline \textbf{Analisi dei requisiti}\newline \url{https://www.math.unipd.it/~tullio/IS-1/2025/Dispense/T05.pdf}
                \item Riferimento alle slide del corso di Ingegneria del Software:\newline \textbf{Analisi e descrizione delle funzionalità: Use Case e relativi diagrammi (UML)}\newline \url{https://www.math.unipd.it/~rcardin/swea/2022/Diagrammi%20Use%20Case.pdf} 
            \end{itemize}
        }
    }
}

\newpage

\section{Descrizione generale}{
    \subsection{Obiettivi del prodotto}{
       Il prodotto ha l'obiettivo di supportare la verifica automatizzata della conformità dei dispositivi radio ai requisiti di sicurezza informatica definiti dalla Direttiva RED mediante l'uso di decision trees interattivi.
    }

    \subsection{Funzioni del prodotto}{
        Il sistema permette all’utente di visualizzare e interagire con i decision trees per la verifica automatizzata della conformità normativa degli asset analizzati.
        \newline
        Le sue principali funzionalità includono:
        \begin{itemize}
            \item \textbf{Importazione da file:} il sistema deve consentire di importare file contenenti gli asset del dispositivo da analizzare, in formati strutturati standard;
            \item \textbf{Creazione di asset:} il sistema permette la creazione diretta degli asset tramite interfaccia web, nel caso non sia disponibile un file di input predefinito;
            \item \textbf{Interazione con i decision trees:} per ciascun asset, il sistema mostra i decision trees pertinenti attraverso una rappresentazione grafica intuitiva, facilitando la comprensione del flusso decisionale;
            \item \textbf{Salvataggio su file:} è possibile salvare l'esito delle analisi in diversi formati (PDF, JSON, ecc.) per consentire l'archiviazione, la condivisione e la consultazione dei risultati;
            \item \textbf{(Opzionale) Possibilità di modifica dei decision trees:} il sistema deve permettere all'utente di intervenire direttamente sui decision trees per adattarli a specifici casi di valutazione o a versioni aggiornate degli standard.
        \end{itemize}    
    }

    \subsection{Caratteristiche utente}{
        Gli utenti di \textit{EN18031 - Automated Compliance Verification} appartengono alla categoria degli utenti esperti. Essi possiedono competenze tecniche in ambito di sicurezza informatica e normativa europea e utilizzano il sistema per verificare la conformità dei dispositivi agli standard di sicurezza vigenti.
    }

    \subsection{Tecnologie e struttura del prodotto}{
        Il progetto si basa sulla realizzazione di un tool interattivo per la verifica della conformità allo standard EN 18031. L'architettura del sistema è di tipo data-driven, dove la logica decisionale è definita dinamicamente attraverso un set strutturato di file organizzati secondo una gerarchia logica che guida l'interazione con l'utente.
        Il prodotto si presenterà sotto forma di applicazione web e sarà consultabile dalla maggior parte dei browser.
    }
}

\newpage
% ----------------------------
% Casi d'uso
% ----------------------------
\section{Casi d'uso}{
    \subsection{Obiettivi}{
        La presente sezione riporta l'elenco dei Casi d'Uso individuati dal team di progetto a seguito di un'accurata analisi del capitolato e di diversi incontri di chiarimento effettuati con 
        l'azienda proponente.
        \newline
        Oltre alla descrizione dei Casi d'Uso, vengono presentati anche i relativi diagrammi, che consentono una migliore comprensione degli attori coinvolti e delle funzionalità 
        offerte dal sistema.  
    }

    \subsection{Attori}{
        L'attore del sistema è uno solo: l'utente finale che potrà usare l'applicazione per effettuare le verifiche di sicurezza. Il team, assieme all'azienda proponente, ha deciso che non saranno necessarie altre tipologie di attori.
        \begin{center}
            \begin{figure}[H]
                \centering
                \includegraphics[scale=0.55]{../../../Assets/Diagrammi/attore.png}
                \caption{Attore}
                \label{fig:attore-principale}
            \end{figure}
        \end{center}
        
    }

    \subsection{Elenco Casi d'Uso}{

        Ogni caso d'uso sarà riportato secondo la seguente struttura:
        \begin{itemize}
            \item \textbf{Caso d'uso:} codice e nome del Caso d'Uso;
            \item \textbf{Attore:} attore dello scenario;
            \item \textbf{Precondizioni:} condizioni che devono essere soddisfatte affinchè l'attore possa accedere al Caso d'Uso;
            \item \textbf{Postcondizioni:} stato del sistema dopo che il caso d'uso si è verificato;
            \item \textbf{Scenario principale:} azioni che l'attore esegue per utilizzare la funzionalità descritta nel Caso d'Uso;
            \item \textbf{Scenario alternativo:} Descrizione ragionevole degli eventi che possono accadere qualora una delle operazioni descritte nello \textbf{Scenario principale} non vada a buon fine
            \item \textbf{Estensioni:} Ulteriori Casi d’Uso che possono manifestarsi nel corso dell’esecuzione delle operazioni previste dal Caso d’Uso principale.
            \item \textbf{Inclusioni:} Ulteriori Casi d’Uso che l’Attore è tenuto a eseguire per completare l’implementazione del Caso d’Uso attualmente in esame.
        \end{itemize}  


        \subsubsection{UC1 - Caricamento del dispositivo}
        \label{uc:uc1}

         \begin{center}
            \begin{figure}[H]
                \centering
                %\includegraphics[scale=0.58]{img/UC1.png}
                \caption{UC1 - Caricamento del dispositivo}
                \label{fig:uc1-sistema}
            \end{figure}
        \end{center}
        
        \begin{itemize}
            \item \textbf{Attore}: Utente
            \item \textbf{Pre-condizioni}: Il Sistema si trova allo stato iniziale
            \item \textbf{Post-condizioni}: Il dispositivo è stato caricato e le sue informazioni sono salvate nel Sistema.
            \item \textbf{Scenario principale}:
            \begin{itemize}
                \item Il Sistema carica i decision Tree
                \item L’Utente seleziona il pulsante per la funzionalità di caricamento del dispositivo
                \item L’Utente carica il dispositivo nel Sistema
                \item Il Sistema porta l’utente nella sezione “Resoconto caricamento”
            \end{itemize}
            \item \textbf{Scenari alternativi}:
            \begin{itemize}
                \item L’Utente carica il dispositivo nel Sistema che rileva che il file del dispositivo non é valido [\hyperref[uc:uc2]{UC2}] 
            \end{itemize}
            \item \textbf{Estensioni}: \hyperref[uc:uc2]{UC2 Rilevamento file non valido}
        \end{itemize}

        \subsubsection{UC2 - Rilevamento file non valido}
        \label{uc:uc2}
                \begin{itemize}
                    \item \textbf{Attore}: Utente
                    \item \textbf{Pre-condizioni}: L’Utente ha selezionato una funzionalitá che prevede un caricamento di file esterno e il file presenta errori di struttura o contenuto.
                    \item \textbf{Post-condizioni}: Il Sistema chiede all’utente se vuole tornare alla home o se vuole riprovare il caricamento del file
                    \item \textbf{Scenario principale}:
                    \begin{itemize}
                        \item Il Sistema rileva un errore di struttura o contenuto nel file caricato.
                        \item Il Sistema interrompe il processo di caricamento.
                        \item Il Sistema mostra un messaggio di errore.
                        \item Il Sistema chiede all’utente se vuole tornare alla home o se vuole riprovare il caricamento del file
                    \end{itemize}
                \end{itemize}  

        \subsubsection{UC3 - Visualizzazione dispositivo}
        \label{uc:uc3}
            \begin{center}
                \begin{figure}[H]
                    \centering
                    %\includegraphics[scale=0.58]{img/UC3.png}
                    \caption{UC3 - Visualizzazione dispositivo}
                    \label{fig:uc3-sistema}
                \end{figure}
            \end{center}
            \begin{itemize}
                \item \textbf{Attore}: Utente
                \item \textbf{Pre-condizioni}: L’Utente ha importato o creato con successo un dispositivo e il Sistema ha salvato le informazioni corrispondenti.
                \item \textbf{Post-condizioni}: Il Sistema mostra all’utente il dispositivo
                \item \textbf{Scenario principale}:
                    \begin{itemize}
                        \item Il Sistema recupera i dati del dispositivo.
                        \item Il Sistema mostra il nome del dispositivo
                    \end{itemize}
                \item \textbf{Scenari alternativi}:
                    \begin{itemize}
                        \item L'Utente seleziona il pulsante per l'espansione delle informazioni del dispositivo [\hyperref[uc:uc4]{UC4}]
                        \item Il Dispositivo é stato creato manualmente e l'Utente seleziona il pulsante per la modifica dello stesso [\hyperref[uc:uc5]{UC5}]
                    \end{itemize}
                \item \textbf{Estensioni}:
                \begin{itemize}
                    \item[-] \hyperref[uc:uc4]{UC4 Espandi informazioni dispositivo}
                    \item[-] \hyperref[uc:uc5]{UC5 Modifica assets dispositivo} 
                \end{itemize}
                \end{itemize}  
                
         \subsubsection{UC4 - Espandi informazioni dispositivo}
            \label{uc:uc4}
                        \begin{itemize}
                            \item \textbf{Attore}: Utente
                            \item \textbf{Pre-condizioni}: Un dispositivo è disponibile e viene visualizzato
                            \item \textbf{Post-condizioni}: Il Sistema mostra la lista degli asset appartenenti al dispositivo.
                            \item \textbf{Scenario principale}:
                            \begin{itemize}
                                \item L’Utente seleziona la funzionalità per espandere le informazioni sul dispositivo
                                \item Il Sistema recupera gli asset associati al dispositivo 
                                \item Il Sistema mostra la lista degli assets [\hyperref[uc:uc4.1]{UC4.1}]
                            \end{itemize}
                            \item \textbf{Inclusioni}: \hyperref[uc:uc4.1]{UC4.1 Visualizzazione lista assets dispositivo}
                        \end{itemize}   

                        Il Caso d’Uso UC4 include un ulteriore Caso d’Uso come raffigurato nella seguente immagine:
                
                        \begin{center}
                            \begin{figure}[H]
                                \centering
                                %\includegraphics[scale=0.58]{img/UC4Inclusioni.png}
                                \caption{UC4 -  Inclusioni Caso d'Uso 4: UC4.1}
                                \label{fig:uc4-inclusioni}
                            \end{figure}
                        \end{center}

                \paragraph{UC4.1 - Visualizzazione lista assets dispositivo} \mbox{}
                \label{uc:uc4.1}
                \begin{itemize}
                    \item \textbf{Attore}: Utente
                    \item \textbf{Post-condizioni}: Il Sistema mostra la lista ordinata dei nomi e dei tipi degli asset appartenenti al dispositivo
                    \item \textbf{Scenario principale}:
                    \begin{itemize}
                        \item Il Sistema ottiene l’elenco degli asset del dispositivo
                        \item Il Sistema ordina la lista secondo l’ordine definito nel file
                        \item Il Sistema visualizza la lista dei nomi degli asset associati al loro tipo 
                    \end{itemize}
                \end{itemize} 

        \subsubsection{UC5 - Modifica assets dispositivo}
        \label{uc:uc5} 
        Inserire commento (se è presente una lista di asset salvata manualmente) UC16
         \begin{center}
                \begin{figure}[H]
                    \centering
                    %\includegraphics[scale=0.58]{img/UC5.png}
                    \caption{UC5 - Modifica assets dispositivo}
                    \label{fig:uc5-sistema}
                \end{figure}
            \end{center}
        \begin{itemize}
                \item \textbf{Attore}: Utente
                \item \textbf{Pre-condizioni}: Il dispositivo viene visualizzato ed esiste una lista di assets creata manualmente
                \item \textbf{Post-condizioni}: Gli assets del dispositivo sono aggiornati e salvati nel sistema
                \item \textbf{Scenario principale}:
                \begin{itemize}
                    \item L’Utente visualizza il dispositivo e la lista degli assets
                    \item L’Utente seleziona un asset da modificare 
                    \item Il Sistema mostra i pulsanti per l’eliminazione degli assets.
                    \item L’Utente decide se aggiungere o eliminare gli assets
                    \item L’Utente conferma le modifiche.
                    \item Il Sistema salva le informazioni aggiornate.
                \end{itemize}
        \end{itemize}

        \subsubsection{UC6 - Esecuzione del test}
        \label{uc:uc6} 
         \begin{center}
                \begin{figure}[H]
                    \centering
                    %\includegraphics[scale=0.58]{img/UC6.png}
                    \caption{UC6 - Esecuzione del test}
                    \label{fig:uc6-sistema}
                \end{figure}
            \end{center}
        \begin{itemize}
                \item \textbf{Attore}: Utente
                \item \textbf{Pre-condizioni}: Il Sistema presenta un dispositivo con i relativi assets caricati. L’Utente seleziona un pulsante per l‘avvio del test. I file dei decision tree interni necessari sono disponibili e leggibili 
                \item \textbf{Post-condizioni}: Il Test è stato completato e sono disponibili i risultati 
                \item \textbf{Scenario principale}:
                \begin{itemize}
                    \item L’Utente seleziona il pulsante per avviare la funzionalitá di avvio del test
                    \item Il Sistema determina la prima domanda in base a (decision tree + asset corrente).
                    \item Il Sistema porta l’Utente nella sezione  “Test” e mostra:
                    \begin{itemize}
                        \item[a)] percentuale avanzamento,
                        \item[b)] info dispositivo e asset,
                        \item[c)] domanda [\hyperref[uc:uc6.1]{UC6.1}] 
                    \end{itemize}
                    \item L’Utente risponde SÌ o NO
                    \item Il Sistema registra la risposta e determina la prossima domanda (o nodo finale) in base a:
                    \begin{itemize}
                        \item[a)] risposta data,
                        \item[b)] decision tree,
                        \item[c)] asset analizzato.
                    \end{itemize}
                    \item Il Sistema aggiorna percentuale e contenuti e torna al passo 5 finché il test termina.
                \end{itemize}
                \item \textbf{Scenari alternativi}:
                \begin{itemize}
                    \item L’Utente seleziona il pulsante per tornare alla domanda precedente [\hyperref[uc:uc7]{UC7}]
                    \item L’Utente seleziona il pulsante per tornare alla domanda successiva [\hyperref[uc:uc8]{UC8}]
                    \item Durante l’esecuzione del test l’Utente seleziona la funzionalità di uscita dal test. [\hyperref[uc:uc9]{UC9}]
                \end{itemize}
                \item \textbf{Inclusioni}: 
                \begin{itemize} 
                    \item[-] \hyperref[uc:uc6.1]{UC6.1 Visualizzazione domanda}
                    \item[-] \hyperref[uc:uc6.2]{UC6.2 Visualizzazione risultati test}
                \end{itemize}
                \item \textbf{Estensioni}: 
                \begin{itemize}
                    \item[-] \hyperref[uc:uc7]{UC7 Caricamento domanda precedente}
                    \item[-] \hyperref[uc:uc9]{UC9 Uscita dal test non concluso} 
                \end{itemize} 
        \end{itemize}   

        Il Caso d’Uso UC6 include due ulteriori Casi d’Uso come raffigurato nella seguente immagine:
                
                        \begin{center}
                            \begin{figure}[H]
                                \centering
                                %\includegraphics[scale=0.58]{img/UC6Inclusioni.png}
                                \caption{UC6 -  Inclusioni Caso d'Uso 6: UC6.1, UC6.2}
                                \label{fig:uc6-inclusioni}
                            \end{figure}
                        \end{center}

                \paragraph{UC6.1 - Visualizzazione domanda} \mbox{}
                \label{uc:uc6.1}
                \begin{itemize}
                    \item \textbf{Attore}: Utente
                    \item \textbf{Post-condizioni}: Il Sistema mostra la domanda relativa al nodo del decision Tree corrente
                    \item \textbf{Scenario principale}:
                    \begin{itemize}
                        \item L’Utente sta eseguendo il test
                        \item Il Sistema mostra all’Utente la domanda corrente del decision Tree
                        \item La domanda é identificata da: 
                        \begin{itemize}
                            \item[a)] Un nome univoco collegato al nodo del decision Tree
                            \item[b)] Il testo della domanda
                        \end{itemize}
                    \end{itemize}
                \end{itemize} 

                \paragraph{UC6.2 - Visualizzazione risultati test} \mbox{}
                \label{uc:uc6.2}
                \begin{itemize}
                    \item \textbf{Attore}: Utente
                    \item \textbf{Pre-condizioni}: L’Utente ha completato un test o ha caricato un test precedente nel sistema.
                    \item \textbf{Post-condizioni}: L’Utente visualizza i risultati del test.
                    \item \textbf{Scenario principale}:
                    \begin{itemize}
                        \item L’Utente ha caricato o completato un test.
                        \item Il Sistema elabora i dati del test
                        \item La domanda é identificata da: 
                        \item L’Utente visualizza in maniera ordinata i risultati del test
                    \end{itemize}
                    \item \textbf{Scenari alternativi}:
                    \begin{itemize}
                        \item L'Utente seleziona il pulsante per la modifica dei risultati del test [\hyperref[uc:uc18]{UC18}]
                        \item L'Utente seleziona il pulsante per il completamento di un test caricato che non era stato finito [\hyperref[uc:uc19]{UC19}]
                        \item L'Utente seleziona il pulsante per ottenere delle giustificazione riguardo al test [\hyperref[uc:uc20]{UC20}]
                    \end{itemize}
                    \item \textbf{Inclusioni}: 
                        \begin{itemize} 
                            \item[-] \hyperref[uc:uc6.2.1]{UC6.2.1 Visualizzazione singolo risultato test}
                            \item[-] \hyperref[uc:uc6.2.1]{UC6.2.2 Esportazione test completato}
                        \end{itemize}
                    \item \textbf{Estensioni}: 
                    \begin{itemize}
                        \item[-] \hyperref[uc:uc18]{UC18 Modifica dei risultati del test}
                        \item[-] \hyperref[uc:uc19]{UC19 Completamento di un test non finito}
                        \item[-] \hyperref[uc:uc20]{UC20 Giustificazioni test} 
                    \end{itemize} 
                \end{itemize}

                \noindent Il Caso d’Uso UC6.2 include due ulteriori Casi d’Uso come raffigurato nella seguente immagine:

                \begin{center}
                    \begin{figure}[H]
                        \centering
                        %\includegraphics[scale=0.58]{img/UC6.2Inclusioni.png}
                        \caption{UC6.2 -  Inclusioni Caso d'Uso 6.2: UC6.2.1, UC6.2.2}
                        \label{fig:uc6.2-inclusioni}
                    \end{figure}
                \end{center}

                \paragraph{UC6.2.1 - Visualizzazione singolo risultato test} \mbox{}
                \label{uc:uc6.2.1}
                \begin{itemize}
                    \item \textbf{Attore}: Utente
                    \item \textbf{Post-condizioni}: É visualizzabile il singolo risultato del test.
                    \item \textbf{Scenario principale}:
                    \begin{itemize}
                        \item L’Utente visualizza la lista di risultati del test 
                        \item L’Utente visualizza il singolo risultato con il nome del requisito e il risultato collegato
                        \item Il risultato puó essere di tre tipi: 
                        \begin{itemize}
                            \item[a)] \textbf{PASS}: Il requisito è stato rispettato
                            \item[b)] \textbf{FAIL}: requisito non rispettato 
                            \item[c)] \textbf{NA}: requisito non applicabile
                        \end{itemize}
                    \end{itemize}
                \end{itemize}

                \paragraph{UC6.2.2 - Esportazione test completato} \mbox{}
                \label{uc:uc6.2.2}
                \begin{itemize}
                    \item \textbf{Attore}: Utente
                    \item \textbf{Post-condizioni}: Il test completato è esportato e reso disponibile all’Utente nei due formati previsti.
                    \item \textbf{Scenario principale}:
                    \begin{itemize}
                        \item L’Utente seleziona la funzionalità di esportazione del test.
                        \item Il Sistema raccoglie i dati del test.
                        \item Il Sistema ordina e formatta i dati.
                        \item Il Sistema genera il file di esportazione.
                        \item Il Sistema rende disponibile il test nei formati:
                        \begin{itemize}
                            \item[a)] PDF
                            \item[b)] JSON
                        \end{itemize}
                    \end{itemize}
                \end{itemize}

        \subsubsection{UC7 - Caricamento domanda precedente}
        \label{uc:uc7}

        \begin{itemize}
            \item \textbf{Attore}: Utente
            \item \textbf{Pre-condizioni}: Il test è in corso ed esiste almeno una domanda precedente nella cronologia (non si è alla prima).
            \item \textbf{Post-condizioni}: Il sistema mostra la domanda precedente e i relativi dati. Viene creata ed aggiornata la storia in avanti, quindi il pulsante avanti diventa utilizzabile.
            \item \textbf{Scenario principale}:
            \begin{itemize}
                \item L’Utente seleziona il pulsante per tornare alla domanda precedente
                \item Il Sistema sposta il puntatore alla domanda precedente nella cronologia 
                \item Il Sistema mostra la domanda precedente (con percentuale e identificativi coerenti con lo stato mostrato)
            \end{itemize} 
            \item \textbf{Scenari alternativi}:
            \begin{itemize}
                    \item L'Utente seleziona il pulsante di navigazione alla domanda successiva [\hyperref[uc:uc8]{UC8}]
            \end{itemize}
            \item \textbf{Estensioni}: \hyperref[uc:uc8]{UC8 Caricamento domanda successiva}
        \end{itemize}   

        \subsubsection{UC8 - Caricamento domanda successiva}
        \label{uc:uc8}

        \begin{itemize}
            \item \textbf{Attore}: Utente
            \item \textbf{Pre-condizioni}: Il test è in corso ed esiste una domanda successiva nella storia in avanti.
            \item \textbf{Post-condizioni}: Il Sistema mostra la domanda successiva nella storia in avanti. Se si raggiunge l’ultima, la freccia avanti torna disabilitata.
            \item \textbf{Scenario principale}:
            \begin{itemize}
                \item L’Utente non ha apportato alcuna modifica alla risposta
                \item L’Utente preme il pulsante per tornare alla domanda successiva
                \item Il Sistema avanza nella storia in avanti
                \item Il Sistema mostra la domanda corrispondente 
            \end{itemize} 
            \item \textbf{Scenari alternativi}:
            \begin{itemize}
                \item L’Utente modifica la risposta della domanda corrente.
                \item Il Sistema invalida la storia in avanti.
                \item Il Sistema disabilita la navigazione in avanti.
                \item Il Sistema determina il nuovo nodo successivo del decision tree.
                \item Il Sistema mostra il nuovo nodo come prossima domanda.
            \end{itemize} 
        \end{itemize}   

        \subsubsection{UC9 - Uscita dal test non concluso}
        \label{uc:uc9}
        
        \begin{itemize}
            \item \textbf{Attore}: Utente
            \item \textbf{Pre-condizioni}: Il test é ancora in corso e quindi non si trova nello stato “completato”
            \item \textbf{Post-condizioni}: Il Sistema marca la sessione come incompleta e salva i progressi.
            \item \textbf{Scenario principale}:
            \begin{itemize}
                \item L’Utente esce dalla sezione di test
                \item Il Sistema esegue [\hyperref[uc:uc9.1]{UC9.1}]
                \item Il Sistema porta l’utente alla pagina principale
            \end{itemize} 
        \end{itemize}   

         Il Caso d’Uso UC9 include un ulteriore Caso d’Uso come raffigurato nella seguente immagine:
                
                        \begin{center}
                            \begin{figure}[H]
                                \centering
                                %\includegraphics[scale=0.58]{img/UC9Inclusioni.png}
                                \caption{UC9 -  Inclusioni Caso d'Uso 9: UC9.1}
                                \label{fig:uc9-inclusioni}
                            \end{figure}
                        \end{center}

                \paragraph{UC9.1 - Salvataggio del test parziale} \mbox{}
                \label{uc:uc9.1}
                \begin{itemize}
                    \item \textbf{Attore}: Utente
                    \item \textbf{Post-condizioni}: Il Sistema salva sul dispositivo dell’Utente lo stato del test non completato
                    \item \textbf{Scenario principale}:
                    \begin{itemize}
                        \item Il Sistema serializza lo stato del test.
                        \item Il Sistema salva lo stato su un file.
                        \item Il Sistema conferma il salvataggio.
                    \end{itemize}
                \end{itemize} 

        \subsubsection{UC10 - Inserimento manuale dei dati del dispositivo}
        \label{uc:uc10}

        \begin{itemize}
            \item \textbf{Attore}: Utente
            \item \textbf{Pre-condizioni}: L’Utente seleziona la funzionalitá per l’inserimento manuale dei dati del dispositivo
            \item \textbf{Post-condizioni}: L'Utente si trova nella sezione dedicata all'inserimento dei dati del dispositivo
            \item \textbf{Scenario principale}:
            \begin{itemize}
                \item L'Utente si trova allo stato iniziale dell'applicazione 
                \item Il Sistema carica i decision Tree
                \item L’Utente seleziona il pulsante per la funzionalità di caricamento manuale dei dati del dispositivo
                \item Il Sistema porta l’utente nella sezione apposita
                \item L’Utente visualizza il form di inserimento dati
            \end{itemize} 
            \item \textbf{Scenari alternativi}: 
            \begin{itemize}
                \item L'Utente salva i dati inseriti nel form [\hyperref[uc:uc11]{UC11}]
            \end{itemize}
            \item \textbf{Inclusioni}: \hyperref[uc:uc10.1]{UC10.1 Visualizzazione form dati dispositivo}
            \item \textbf{Estensioni}: \hyperref[uc:uc11]{UC11 Salvataggio dati inseriti manualmente del dispositivo}
        \end{itemize}     
        
        \noindent Il Caso d’Uso UC10 include un ulteriore Caso d’Uso come raffigurato nella seguente immagine:
                
                        \begin{center}
                            \begin{figure}[H]
                                \centering
                                %\includegraphics[scale=0.58]{img/UC10Inclusioni.png}
                                \caption{UC10 -  Inclusioni Caso d'Uso 10: UC10.1}
                                \label{fig:uc10-inclusioni}
                            \end{figure}
                        \end{center}

                \paragraph{UC10.1 - Visualizzazione form dati dispositivo} \mbox{}
                \label{uc:uc10.1}
                \begin{itemize}
                    \item \textbf{Attore}: Utente
                    \item \textbf{Post-condizioni}: Il Sistema mostra il form di inserimento manuale dei dati
                    \item \textbf{Scenario principale}:
                    \begin{itemize}
                        \item L’Utente si trova nella sezione per l'inserimento dei dati del dispositivo
                        \item L’Utente visualizza il form per l’inserimento dei dati
                    \end{itemize}
                    \item \textbf{Inclusioni}:
                    \begin{itemize}
                        \item[-] \hyperref[uc:uc10.1.1]{UC10.1.1 Inserimento nome }
                        \item[-] \hyperref[uc:uc10.1.2]{UC10.1.2 Inserimento sistema operativo}
                        \item[-] \hyperref[uc:uc10.1.3]{UC10.1.3 Inserimento versione firmware}
                        \item[-] \hyperref[uc:uc10.1.4]{UC10.1.4 Inserimento funzionalità}
                        \item[-] \hyperref[uc:uc10.1.5]{UC10.1.5 Inserimento descrizione}
                    \end{itemize}
                \end{itemize} 

                \noindent Il Caso d’Uso UC10.1 include cinque ulteriori Caso d’Uso come raffigurato nella seguente immagine:
                
                        \begin{center}
                            \begin{figure}[H]
                                \centering
                                %\includegraphics[scale=0.58]{img/UC10.1Inclusioni.png}
                                \caption{UC10.1 -  Inclusioni Caso d'Uso 10.1: UC10.1.1, UC10.1.2, UC10.1.3, UC10.1.4, UC10.1.5}
                                \label{fig:uc10.1-inclusioni}
                            \end{figure}
                        \end{center}

                \paragraph{UC10.1.1 - Inserimento nome} \mbox{}
                \label{uc:uc10.1.1}
                \begin{itemize}
                    \item \textbf{Attore}: Utente
                    \item \textbf{Post-condizioni}: Il nome è stato inserito dall’Utente
                    \item \textbf{Scenario principale}:
                    \begin{itemize}
                        \item L’Utente visualizza il form di inserimento dati del dispositivo
                        \item L’Utente seleziona il campo “Nome”
                        \item L’Utente inserisce il nome del dispositivo
                        \item Il Sistema memorizza il valore inserito
                    \end{itemize}
                \end{itemize} 

                \paragraph{UC10.1.2 - Inserimento sistema operativo} \mbox{}
                \label{uc:uc10.1.2}
                \begin{itemize}
                    \item \textbf{Attore}: Utente
                    \item \textbf{Post-condizioni}: Il sistema operativo è stato inserito dall’Utente
                    \item \textbf{Scenario principale}:
                    \begin{itemize}
                        \item L’Utente visualizza il form di inserimento dati del dispositivo.
                        \item L’Utente seleziona il campo “Sistema operativo”.
                        \item L’Utente inserisce il sistema operativo del dispositivo.
                        \item Il Sistema memorizza il valore inserito.
                    \end{itemize}
                \end{itemize} 

                \paragraph{UC10.1.3 - Inserimento versione firmware} \mbox{}
                \label{uc:uc10.1.3}
                \begin{itemize}
                    \item \textbf{Attore}: Utente
                    \item \textbf{Post-condizioni}: La versione del firmware del dispositivo è stata inserita dall’Utente
                    \item \textbf{Scenario principale}:
                    \begin{itemize}
                        \item L’Utente visualizza il form di inserimento dati del dispositivo.
                        \item L’Utente seleziona il campo “Versione firmware”.
                        \item L’Utente inserisce la versione del firmware del dispositivo.
                        \item Il Sistema memorizza il valore inserito.
                    \end{itemize}
                \end{itemize} 

                \paragraph{UC10.1.4 - Inserimento funzionalità} \mbox{}
                \label{uc:uc10.1.4}
                \begin{itemize}
                    \item \textbf{Attore}: Utente
                    \item \textbf{Post-condizioni}: La funzionalitá del dispositivo é stata inserita dall’Utente
                    \item \textbf{Scenario principale}:
                    \begin{itemize}
                        \item L’Utente visualizza il form di inserimento dati del dispositivo.
                        \item L’Utente seleziona il campo “Funzionalitá”.
                        \item L’Utente inserisce le funzionalità del dispositivo.
                        \item Il Sistema memorizza il valore inserito.
                    \end{itemize}
                \end{itemize} 

                \paragraph{UC10.1.5 - Inserimento descrizione} \mbox{}
                \label{uc:uc10.1.5}
                \begin{itemize}
                    \item \textbf{Attore}: Utente
                    \item \textbf{Post-condizioni}: La descrizione del dispositivo é stata inserita dall’Utente
                    \item \textbf{Scenario principale}:
                    \begin{itemize}
                        \item L’Utente visualizza il form di inserimento dati del dispositivo.
                        \item L’Utente seleziona il campo “Descrizione”.
                        \item L’Utente inserisce la descrizione del dispositivo.
                        \item Il Sistema memorizza il valore inserito.
                    \end{itemize}
                \end{itemize} 
                
        \subsubsection{UC11 - Salvataggio dati del dispositivo inseriti manualmente }
        \label{uc:uc11}

        \begin{itemize}
            \item \textbf{Attore}: Utente
            \item \textbf{Pre-condizioni}: L’Utente ha compilato tutti i campi obbligatori riguardanti il form del dispositivo
            \item \textbf{Post-condizioni}: L’Utente si trova nella sezione per l'aggiunta degli assets del dispositivo
            \item \textbf{Scenario principale}:
            \begin{itemize}
                \item L’Utente ha completato il form dei dati del dispositivo.
                \item L’Utente seleziona la funzionalità di salvataggio
                \item Il Sistema valida i dati inseriti.
                \item Il Sistema salva il dispositivo.
                \item Il Sistema porta l’Utente alla sezione per l'inserimento degli assets del dispositivo.
            \end{itemize} 
        \end{itemize}

        \subsubsection{UC12 - Aggiunta asset dispositivo }
        \label{uc:uc12}

        \begin{itemize}
            \item \textbf{Attore}: Utente
            \item \textbf{Pre-condizioni}: Il Sistema ha salvato i dati del dispositivo inseriti manualmente
            \item \textbf{Post-condizioni}: L’Utente si trova nella sezione per l'inserimento di un asset
            \item \textbf{Scenario principale}:
            \begin{itemize}
                \item L'Utente si trova nella sezione per l'inserimento degli assets del dispositivo
                \item L'Utente seleziona il pulsante per l'inserimento di un asset
                \item L'Utente visualizza il form di Inserimento dati per l'Asset
            \end{itemize} 
            \item \textbf{Scenari alternativi}:
            \begin{itemize}
                \item 
            \end{itemize}
            \item \textbf{Inclusioni}: \hyperref[uc:uc12.1]{UC12.1 Visualizzazione form dati singolo asset }
            \item \textbf{Estensioni}: 
            \begin{itemize}
                \item[-] \hyperref[uc:uc13]{UC13 Salvataggio dell’asset creato}
                \item[-] \hyperref[uc:uc15]{UC15 Annullamento creazione assets}
            \end{itemize}
        \end{itemize}

         Il Caso d’Uso UC12 include un ulteriore Caso d’Uso come raffigurato nella seguente immagine:
                
                        \begin{center}
                            \begin{figure}[H]
                                \centering
                                %\includegraphics[scale=0.58]{img/UC12Inclusioni.png}
                                \caption{UC12 -  Inclusioni Caso d'Uso 12: UC12.1}
                                \label{fig:uc12-inclusioni}
                            \end{figure}
                        \end{center}

                \paragraph{UC12.1 - Visualizzazione form dati singolo asset } \mbox{}
                \label{uc:uc12.1}
                \begin{itemize}
                    \item \textbf{Attore}: Utente
                    \item \textbf{Post-condizioni}: L’Utente visualizza il form per la creazione dell’asset
                    \item \textbf{Scenario principale}:
                    \begin{itemize}
                        \item L’Utente si trova nella sezione contenente il form per l'inserimento dei dati di un asset
                        \item L'Utente visualizza il form per l'inserimento dei dati dell'asset
                    \end{itemize}
                    \item \textbf{Inclusioni}: 
                    \begin{itemize}
                        \item[-] \hyperref[uc:uc12.1.1]{UC12.1.1 Inserimento nome asset }
                        \item[-] \hyperref[uc:uc12.1.2]{UC12.1.2 Scelta tipo dell’asset }
                        \item[-] \hyperref[uc:uc12.1.3]{UC12.1.3 Inserimento descrizione asset }
                        \item[-] \hyperref[uc:uc12.1.4]{UC12.1.4 Scelta sensibilità dell’asset}
                    \end{itemize}
                \end{itemize} 

                \noindent Il Caso d’Uso UC12.1 include quattro ulteriori Casi d’Uso come raffigurato nella seguente immagine:
                
                        \begin{center}
                            \begin{figure}[H]
                                \centering
                                %\includegraphics[scale=0.58]{img/UC12.1Inclusioni.png}
                                \caption{UC12.1 -  Inclusioni Caso d'Uso 12.1: UC12.1.1, UC12.1.2, UC12.1.3, UC12.1.4}
                                \label{fig:uc12.1-inclusioni}
                            \end{figure}
                        \end{center}

                \paragraph{UC12.1.1 - Inserimento nome asset} \mbox{}
                \label{uc:uc12.1.1}
                \begin{itemize}
                    \item \textbf{Attore}: Utente
                    \item \textbf{Post-condizioni}: Il nome dell’asset è stato inserito
                    \item \textbf{Scenario principale}:
                    \begin{itemize}
                        \item L’Utente visualizza il form di creazione dell’asset.
                        \item L’Utente seleziona il campo “Nome asset”.
                        \item L’Utente inserisce il nome.
                        \item Il Sistema memorizza il valore.
                    \end{itemize}
                \end{itemize} 

                \paragraph{UC12.1.2 - Scelta tipo dell’asset } \mbox{}
                \label{uc:uc12.1.2}
                \begin{itemize}
                    \item \textbf{Attore}: Utente
                    \item \textbf{Post-condizioni}: Il tipo dell’asset è stato scelto. 
                    \item \textbf{Scenario principale}:
                    \begin{itemize}
                        \item L’Utente visualizza il form di creazione dell’asset.
                        \item L’Utente seleziona il campo per l’inserimento del tipo.
                        \item L’Utente sceglie il tipo.
                    \end{itemize}
                \end{itemize} 

                \paragraph{UC12.1.3 - Inserimento descrizione asset} \mbox{}
                \label{uc:uc12.1.3}
                \begin{itemize}
                    \item \textbf{Attore}: Utente
                    \item \textbf{Post-condizioni}: La descrizione dell’asset è stata inserita. 
                    \item \textbf{Scenario principale}:
                    \begin{itemize}
                        \item L’Utente visualizza il form di creazione dell’asset.
                        \item L’Utente seleziona il campo per l’inserimento della descrizione
                        \item L’Utente inserisce la descrizione.
                        \item L’Utente completa il campo con l’ informazione richiesta
                    \end{itemize}
                \end{itemize} 

                \paragraph{UC12.1.4 - Scelta sensibilità dell’asset} \mbox{}
                \label{uc:uc12.1.4}
                \begin{itemize}
                    \item \textbf{Attore}: Utente
                    \item \textbf{Post-condizioni}: La sensibilità dell’asset è stata scelta. 
                    \item \textbf{Scenario principale}:
                    \begin{itemize}
                        \item L’Utente visualizza il form di creazione dell’asset.
                        \item L’Utente seleziona il campo per l’inserimento della sensibilitá
                        \item L’Utente sceglie la sensibilitá
                    \end{itemize}
                \end{itemize} 
                
            \subsubsection{UC13 - Salvataggio dell’asset creato}
            \label{uc:uc13}
    
            \begin{itemize}
                \item \textbf{Attore}: Utente
                \item \textbf{Pre-condizioni}: L’Utente ha inserito tutte le informazioni obbligatorie per la creazione dell’asset
                \item \textbf{Post-condizioni}: L’asset é stato salvato nel sistema
                \item \textbf{Scenario principale}:
                \begin{itemize}
                    \item L’Utente ha completato il form dell’asset.
                    \item L’Utente conferma la creazione.
                    \item Il Sistema valida i dati.
                    \item Il Sistema salva l’asset.
                    \item Il Sistema aggiorna la lista degli asset creati.
                \end{itemize} 
                \item \textbf{Inclusioni}: \hyperref[uc:uc13.1]{UC13.1 Visualizzazione lista assets creati}
            \end{itemize}

            \noindent Il Caso d’Uso UC13 include un ulteriore Caso d’Uso come raffigurato nella seguente immagine:
                
                    \begin{center}
                        \begin{figure}[H]
                            \centering
                            %\includegraphics[scale=0.58]{img/UC13Inclusioni.png}
                            \caption{UC13 -  Inclusioni Caso d'Uso 13: UC13.1}
                            \label{fig:uc13-inclusioni}
                        \end{figure}
                    \end{center}

                    \paragraph{UC13.1 - Visualizzazione lista assets creati} \mbox{}
                    \label{uc:uc13.1}
                    \begin{itemize}
                        \item \textbf{Attore}: Utente
                        \item \textbf{Post-condizioni}: L’Utente visualizza la lista degli assets creati fino a quel momento
                        \item \textbf{Scenario principale}:
                        \begin{itemize}
                            \item L’Utente si trova nella sezione di visualizzazione degli assets creati.
                            \item L’Utente visualizza la lista di assets creati.
                        \end{itemize}
                    \end{itemize} 

                    \noindent Il Caso d’Uso UC13.1 include un ulteriore Caso d’Uso come raffigurato nella seguente immagine:
                
                    \begin{center}
                        \begin{figure}[H]
                            \centering
                            %\includegraphics[scale=0.58]{img/UC13.1Inclusioni.png}
                            \caption{UC13.1 -  Inclusioni Caso d'Uso 13.1: UC13.1.1}
                            \label{fig:uc13.1-inclusioni}
                        \end{figure}
                    \end{center}

                    \paragraph{UC13.1.1 - Visualizzazione singolo asset} \mbox{}
                    \label{uc:uc13.1.1}
                    \begin{itemize}
                        \item \textbf{Attore}: Utente
                        \item \textbf{Post-condizioni}: L'Asset é visualizzato
                        \item \textbf{Scenario principale}:
                        \begin{itemize}
                            \item L'Utente visualizza il nome dell'asset dalla lista di asset
                        \end{itemize}
                        \item \textbf{Scenari alternativi}:
                        \begin{itemize}
                            \item L'Utente seleziona il pulsante per l'eliminazione dell'asset [\hyperref[uc:uc14]{UC14}]
                        \end{itemize}
                        \item \textbf{Estensioni}: \hyperref[uc:uc14]{UC14 Eliminazione asset creato}
                    \end{itemize} 

            \subsubsection{UC14 - Eliminazione asset creato}
            \label{uc:uc14}
            \begin{itemize}
                \item \textbf{Attore}: Utente
                \item \textbf{Pre-condizioni}: Almeno un asset è stato creato e salvato nel sistema
                \item \textbf{Post-condizioni}: L’asset é stato eliminato dal sistema 
                \item \textbf{Scenario principale}:
                \begin{itemize}
                    \item L’Utente visualizza la lista degli asset creati.
                    \item L’Utente seleziona un asset.
                    \item L’Utente seleziona la funzionalità di eliminazione.
                    \item Il Sistema elimina l’asset.
                    \item Il Sistema aggiorna la lista degli asset.
                \end{itemize} 
            \end{itemize}

            \subsubsection{UC15 - Annullamento creazione assets}
            \label{uc:uc15}
            \begin{itemize}
                \item \textbf{Attore}: Utente
                \item \textbf{Pre-condizioni}: L’Utente si trova nella sezione di riepilogo degli assets creati
                \item \textbf{Post-condizioni}: Il sistema elimina le informazioni del dispositivo e degli asset. L’Utente si trova allo stato iniziale dell’applicazione
                \item \textbf{Scenario principale}:
                \begin{itemize}
                    \item L’Utente si trova nella sezione di riepilogo degli assets creati.
                    \item L’Utente seleziona la funzionalità di annullamento.
                    \item Il Sistema chiede conferma dell’operazione.
                    \item L’Utente conferma l'operazione.
                    \item Il Sistema elimina i dati del dispositivo e degli asset.
                    \item Il Sistema riporta l’Utente allo stato iniziale.
                \end{itemize} 
            \end{itemize}

            \subsubsection{UC16 - Salvataggio della lista di assets}
            \label{uc:uc16}
            \begin{itemize}
                \item \textbf{Attore}: Utente
                \item \textbf{Pre-condizioni}: La lista di assets creati manualmente nella sezione di riepilogo degli assets creati presenta almeno un asset 
                \item \textbf{Post-condizioni}: Il dispositivo con la sua lista di assets è salvato ed è modificabile
                \item \textbf{Scenario principale}:
                \begin{itemize}
                    \item L’Utente ha completato la creazione della lista di asset.
                    \item L’Utente seleziona la funzionalità di salvataggio.
                    \item Il Sistema valida la lista di asset.
                    \item Il Sistema salva il dispositivo con gli asset associati.
                    \item Il Sistema rende il dispositivo modificabile [\hyperref[uc:uc5]{UC5}].
                \end{itemize} 
            \end{itemize}

            \subsubsection{UC17 - Caricamento test precedente}
            \label{uc:uc17}
            \begin{itemize}
                \item \textbf{Attore}: Utente
                \item \textbf{Pre-condizioni}: L’Utente seleziona la funzionalitá per il caricamento di un test precedente 
                \item \textbf{Post-condizioni}: Il Test é stato caricato nel sistema 
                \item \textbf{Scenario principale}:
                \begin{itemize}
                    \item L’Utente seleziona la funzionalità di caricamento di un test precedente.
                    \item Il Sistema richiede il file del test.
                    \item L’Utente seleziona il file.
                    \item Il Sistema valida il file.
                    \item Il Sistema carica il test nel sistema.
                    \item Il Sistema mostra i risultati del test [\hyperref[uc:uc6.2]{UC6.2}].
                \end{itemize} 
                \item \textbf{Scenari alternativi}:
                \begin{itemize}
                    \item L’Utente carica un file che presenta errori di struttura o di formato [\hyperref[uc:uc2]{UC2}]
                \end{itemize} 
                \item \textbf{Estensioni}: \hyperref[uc:uc2]{UC2 Rilevamento file non valido}
            \end{itemize}

            \subsubsection{UC18 - Modifica dei risultati del test}
            \label{uc:uc18}
            \begin{itemize}
                \item \textbf{Attore}: Utente
                \item \textbf{Pre-condizioni}: É presente nel sistema un test completato caricato dall’Utente o concluso in questa sessione
                \item \textbf{Post-condizioni}: Il test è aggiornato a partire dal requisito selezionato e i nuovi risultati sono salvati.
                \item \textbf{Scenario principale}:
                \begin{itemize}
                    \item L’Utente visualizza i risultati di un test completato [\hyperref[uc:uc6.2]{UC6.2}].
                    \item L’Utente seleziona un requisito da modificare.
                    \item Il Sistema identifica il nodo del decision tree associato al requisito selezionato.
                    \item Il Sistema avvia l’esecuzione del test a partire da quel nodo.
                \end{itemize} 
                \item \textbf{Inclusioni}: \hyperref[uc:uc6]{UC6  Esecuzione del test }
            \end{itemize}

            \subsubsection{UC19 - Completamento di un test non finito}
            \label{uc:uc19}
            \begin{itemize}
                \item \textbf{Attore}: Utente
                \item \textbf{Pre-condizioni}: È stato caricato un test non completato
                \item \textbf{Post-condizioni}: Il test riprende dal nodo successivo a quello risposto dall’Utente
                \item \textbf{Scenario principale}:
                \begin{itemize}
                    \item L’Utente carica un test non completato.
                    \item Il Sistema identifica l’ultimo nodo risposto.
                    \item Il Sistema posiziona il test sul nodo successivo.
                    \item Il Sistema avvia l’esecuzione del test da quel nodo [\hyperref[uc:uc6]{UC6}].
                \end{itemize} 
            \end{itemize}

            \subsubsection{UC20 - Giustificazioni test}
            \label{uc:uc20}
            \begin{itemize}
                \item \textbf{Attore}: Utente
                \item \textbf{Pre-condizioni}: E’ presente un test completato
                \item \textbf{Post-condizioni}: Le giustificazioni sono visualizzate
                \item \textbf{Scenario principale}:
                \begin{itemize}
                    \item L’Utente visualizza i risultati di un test completato.
                    \item L’Utente seleziona la funzionalità “Giustificazioni”.
                    \item Il Sistema recupera le motivazioni associate a ciascun risultato.
                    \item Il Sistema mostra le giustificazioni in modo strutturato. 
                \end{itemize} 
            \end{itemize}

            \subsubsection{UC21 - Accesso all’editor grafico dei decision tree}
            \label{uc:uc21}
            \begin{itemize}
                \item \textbf{Attore}: Utente
                \item \textbf{Pre-condizioni}: Il Sistema è avviato e i file dei decision tree sono disponibili.
                \item \textbf{Post-condizioni}: L’Utente si trova nella sezione per l'editing del decision tree.
                \item \textbf{Scenario principale}:
                \begin{itemize}
                    \item L’Utente seleziona la funzionalità di editing del decision tree.
                    \item Il Sistema carica l’elenco dei decision tree disponibili.
                    \item Il Sistema porta l’Utente nella sezione dedicata alla modifica del decision tree.
                \end{itemize}
                \item \textbf{Scenari alternativi}:
                \begin{itemize}
                    \item L’Utente seleziona un decision tree da modificare [\hyperref[uc:uc22]{UC22}].
                \end{itemize}
                \item \textbf{Inclusioni}: \hyperref[uc:uc21.1]{UC21.1 Visualizzazione elenco decision tree} 
                \item \textbf{Estensioni}: \hyperref[uc:uc22]{UC22 Selezione decision tree da modificare}
            \end{itemize}

            \noindent Il Caso d’Uso UC21 include un ulteriore Caso d’Uso come raffigurato nella seguente immagine:
                
                    \begin{center}
                        \begin{figure}[H]
                            \centering
                            %\includegraphics[scale=0.58]{img/UC21Inclusioni.png}
                            \caption{UC21 -  Inclusioni Caso d'Uso 21: UC21.1}
                            \label{fig:uc21-inclusioni}
                        \end{figure}
                    \end{center}

                    \paragraph{UC21.1 - Visualizzazione elenco decision trees} \mbox{}
                    \label{uc:uc21.1}
                    \begin{itemize}
                        \item \textbf{Attore}: Utente
                        \item \textbf{Post-condizioni}: Il Sistema mostra l’elenco dei decision tree disponibili.
                        \item \textbf{Scenario principale}:
                        \begin{itemize}
                            \item Il Sistema recupera i decision tree dal file di configurazione.
                            \item Il Sistema mostra una lista ordinata dei decision tree
                        \end{itemize}
                        \item \textbf{Inclusioni}: \hyperref[uc:uc21.1.1]{UC21.1.1 Visualizzazione singolo decision tree} 
                    \end{itemize}

                    \noindent Il Caso d’Uso UC21.1 include un ulteriore Caso d’Uso come raffigurato nella seguente immagine:
                
                    \begin{center}
                        \begin{figure}[H]
                            \centering
                            %\includegraphics[scale=0.58]{img/UC21.1Inclusioni.png}
                            \caption{UC21.1 -  Inclusioni Caso d'Uso 21.1: UC21.1.1}
                            \label{fig:uc21.1-inclusioni}
                        \end{figure}
                    \end{center}

                    \paragraph{UC21.1.1 - Visualizzazione singolo decision tree} \mbox{}
                    \label{uc:uc21.1.1}
                    \begin{itemize}
                        \item \textbf{Attore}: Utente
                        \item \textbf{Post-condizioni}: Il singolo decision tree é visualizzato.
                        \item \textbf{Scenario principale}:
                        \begin{itemize}
                            \item L'Utente visualizza il singolo decision tree.
                            \item Il decision tree viene descritto da:
                            \begin{itemize}
                                \item id del requisito [\hyperref[uc:uc21.1.1.1]{UC21.1.1.1}]
                                \item titolo [\hyperref[uc:uc21.1.1.2]{UC21.1.1.2}]
                                \item dipendenze (se presenti) [\hyperref[uc:uc21.1.1.3]{UC21.1.1.3}]
                            \end{itemize}
                        \end{itemize}
                        \item \textbf{Inclusioni}: 
                        \begin{itemize}
                            \item \hyperref[uc:uc21.1.1.1]{UC21.1.1.1 Visualizzazione id del requisito del decision tree}
                            \item \hyperref[uc:uc21.1.1.2]{UC21.1.1.2 Visualizzazione titolo del decision tree}
                            \item \hyperref[uc:uc21.1.1.3]{UC21.1.1.3 Visualizzazione dipendenze del decision tree}
                        \end{itemize} 
                    \end{itemize}

                    \noindent Il Caso d’Uso UC21.1.1 include tre ulteriori Casi d’Uso come raffigurato nella seguente immagine:
                
                    \begin{center}
                        \begin{figure}[H]
                            \centering
                            %\includegraphics[scale=0.58]{img/UC21.1.1Inclusioni.png}
                            \caption{UC21.1.1 -  Inclusioni Caso d'Uso 21.1.1: UC21.1.1.1, UC21.1.1.2, UC21.1.1.3, UC21.1.1.4}
                            \label{fig:uc21.1.1-inclusioni}
                        \end{figure}
                    \end{center}

                    \paragraph{UC21.1.1.1 - Visualizzazione id del requisito del decision tree} \mbox{}
                    \label{uc:uc21.1.1.1}
                    \begin{itemize}
                        \item \textbf{Attore}: Utente
                        \item \textbf{Post-condizioni}: L'id del requisito del decision tree é visualizzato
                        \item \textbf{Scenario principale}:
                        \begin{itemize}
                            \item L'Utente visualizza l'id del requisito rappresentato dal decision tree.
                        \end{itemize}
                    \end{itemize}

                     \paragraph{UC21.1.1.2 - Visualizzazione titolo del decision tree} \mbox{}
                    \label{uc:uc21.1.1.2}
                    \begin{itemize}
                        \item \textbf{Attore}: Utente
                        \item \textbf{Post-condizioni}: Il titolo del decision tree é visualizzato
                        \item \textbf{Scenario principale}:
                        \begin{itemize}
                            \item L'Utente visualizza il titolo del decision tree.
                        \end{itemize}
                    \end{itemize}

                     \paragraph{UC21.1.1.3 - Visualizzazione dipendenze del decision tree} \mbox{}
                    \label{uc:uc21.1.1.3}
                    \begin{itemize}
                        \item \textbf{Attore}: Utente
                        \item \textbf{Pre-condizioni}: Il decision tree presenta delle dipendenze
                        \item \textbf{Post-condizioni}: Le dipendenze del decision tree sono visualizzate
                        \item \textbf{Scenario principale}:
                        \begin{itemize}
                            \item L'Utente visualizza le dipendenze del decision tree.
                        \end{itemize}
                    \end{itemize}
                    

            \subsubsection{UC22 - Selezione decision tree da modificare}
            \label{uc:uc22}
            \begin{itemize}
                \item \textbf{Attore}: Utente
                \item \textbf{Pre-condizioni}: L’Utente visualizza l’elenco dei decision tree.
                \item \textbf{Post-condizioni}: Il decision tree selezionato è caricato.
                \item \textbf{Scenario principale}:
                \begin{itemize}
                    \item L’Utente seleziona un decision tree.
                    \item Il Sistema carica la struttura del tree.
                    \item Il Sistema visualizza il decision tree.
                \end{itemize}
                \item \textbf{Inclusioni}: \hyperref[uc:uc22.1]{UC22.1 Visualizzazione grafica del decision tree} 
            \end{itemize}

            \noindent Il Caso d’Uso UC22 include un ulteriore Caso d’Uso come raffigurato nella seguente immagine:
                
                    \begin{center}
                        \begin{figure}[H]
                            \centering
                            %\includegraphics[scale=0.58]{img/UC22Inclusioni.png}
                            \caption{UC22 -  Inclusioni Caso d'Uso 22: UC22.1}
                            \label{fig:uc22-inclusioni}
                        \end{figure}
                    \end{center}

                    \paragraph{UC22.1 - Visualizzazione grafica del decision tree} \mbox{}
                    \label{uc:uc22.1}
                    \begin{itemize}
                        \item \textbf{Attore}: Utente
                        \item \textbf{Post-condizioni}: Il Sistema mostra il decision tree come grafo orientato.
                        \item \textbf{Scenario principale}:
                        \begin{itemize}
                            \item Il Sistema identifica:
                            \begin{itemize}
                                \item nodo root
                                \item nodi intermedi
                                \item nodi risultato (PASS / FAIL / NA)
                            \end{itemize}
                            \item Il Sistema visualizza:
                            \begin{itemize}
                                \item ogni nodo come elemento grafico
                                \item i collegamenti true/false come archi direzionati
                            \end{itemize}
                            \item Il Sistema evidenzia l’id del requisito associato al tree.
                        \end{itemize}
                    \end{itemize}

            \subsubsection{UC23 - Modifica del decision tree}
            \label{uc:uc23}
            \begin{itemize}
                \item \textbf{Attore}: Utente
                \item \textbf{Pre-condizioni}: Un decision tree è visualizzato nell’editor.
                \item \textbf{Post-condizioni}: Il decision tree è aggiornato.
                \item \textbf{Scenario principale}:
                \begin{itemize}
                    \item L’Utente seleziona un nodo.
                    \item Il Sistema mostra i dati modificabili.
                    \begin{itemize}
                        \item testo della domanda
                        \item collegamenti true / false
                    \end{itemize}
                    \item L’Utente modifica uno o più campi.
                    \item Il Sistema aggiorna il nodo nel tree.
                \end{itemize}
                \item \textbf{Scenari alternativi}:
                \begin{itemize}
                    \item L’Utente seleziona la funzionalitá di aggiunta di un nodo del decision tree \hyperref[uc:uc24]{UC24}.
                    \item L’Utente seleziona la funzionalitá di eliminazione di un nodo del decision tree \hyperref[uc:uc25]{UC25}.
                    \item L’Utente modifica il collegamento tra due nodi del decision tree \hyperref[uc:uc26]{UC26}.
                \end{itemize}
                \item \textbf{Estensioni}: 
                \begin{itemize}
                    \item \hyperref[uc:uc24]{UC24 Aggiunta nodo al decision tree}
                    \item \hyperref[uc:uc25]{UC25 Eliminazione nodo del decision tree} 
                    \item \hyperref[uc:uc26]{UC26 Modifica collegamenti tra nodi} 
                \end{itemize}
            \end{itemize}

            \subsubsection{UC24 - Aggiunta nodo al decision tree}
            \label{uc:uc24}
            \begin{itemize}
                \item \textbf{Attore}: Utente
                \item \textbf{Pre-condizioni}: Un decision tree è visualizzato nell’editor.
                \item \textbf{Post-condizioni}: Un nuovo nodo è stato aggiunto al decision tree.
                \item \textbf{Scenario principale}:
                \begin{itemize}
                    \item L’Utente seleziona la funzionalità di aggiunta di un nodo.
                    \item Il Sistema crea un nuovo nodo vuoto.
                    \item L’Utente inserisce:
                    \begin{itemize}
                        \item testo della domanda
                        \item tipo del nodo
                    \end{itemize}
                    \item Il Sistema aggiunge il nodo alla fine del tree.
                \end{itemize}
            \end{itemize}

            \subsubsection{UC25 - Eliminazione nodo del decision tree}
            \label{uc:uc25}
            \begin{itemize}
                \item \textbf{Attore}: Utente
                \item \textbf{Pre-condizioni}: Il decision tree contiene almeno un nodo non root.
                \item \textbf{Post-condizioni}: Il nodo selezionato è rimosso e i collegamenti aggiornati.
                \item \textbf{Scenario principale}:
                \begin{itemize}
                    \item L’Utente seleziona un nodo.
                    \item L’Utente seleziona la funzionalità di eliminazione.
                    \item Il Sistema verifica che il nodo non sia root.
                    \item Il Sistema rimuove il nodo.
                    \item Il Sistema aggiorna i collegamenti.
                \end{itemize}
            \end{itemize}

            \subsubsection{UC26 - Modifica collegamenti tra nodi}
            \label{uc:uc26}
            \begin{itemize}
                \item \textbf{Attore}: Utente
                \item \textbf{Pre-condizioni}: Un decision tree è visualizzato.
                \item \textbf{Post-condizioni}: I collegamenti true/false sono aggiornati.
                \item \textbf{Scenario principale}:
                \begin{itemize}
                    \item L’Utente seleziona un collegamento.
                    \item L’Utente modifica il nodo di destinazione.
                    \item Il Sistema aggiorna il grafo.
                \end{itemize}
            \end{itemize}

            \subsubsection{UC27 - Validazione del decision tree}
            \label{uc:uc27}
            \begin{itemize}
                \item \textbf{Attore}: Utente
                \item \textbf{Pre-condizioni}: Il decision tree è stato modificato.
                \item \textbf{Post-condizioni}: Il decision tree è validato e salvato nel Sistema
                \item \textbf{Scenario principale}:
                \begin{itemize}
                    \item L’Utente seleziona la funzionalità di validazione del decision tree.
                    \item Il Sistema verifica:
                    \begin{itemize}
                        \item esistenza del nodo root
                        \item assenza di nodi orfani
                        \item raggiungibilità dei nodi risultato
                        \item coerenza degli id
                    \end{itemize}
                    \item Il Sistema mostra l’esito della validazione.
                    \item Il decision tree é salvato nel Sistema.
                \end{itemize}
                \item \textbf{Scenari alternativi}:
                \begin{itemize}
                    \item Il decision tree modificato presenta uno o piú errori  [\hyperref[uc:uc28]{UC28}].
                \end{itemize}
                \item \textbf{Estensioni}: 
                \begin{itemize}
                    \item \hyperref[uc:uc28]{UC28 Segnalazione errori del decision tree}
                \end{itemize} 
            \end{itemize}

            \subsubsection{UC28 - Segnalazione errori del decision tree}
            \label{uc:uc28}
            \begin{itemize}
                \item \textbf{Attore}: Utente
                \item \textbf{Pre-condizioni}: La validazione del decision tree ha rilevato errori.
                \item \textbf{Post-condizioni}: Gli errori sono mostrati all’Utente.
                \item \textbf{Scenario principale}:
                \begin{itemize}
                    \item Il Sistema evidenzia i nodi problematici.
                    \item Il Sistema descrive il tipo di errore.
                    \item L’Utente può tornare alla modifica.
                \end{itemize}
            \end{itemize}

            \subsubsection{UC29 - Esportazione decision tree}
            \label{uc:uc28}
            \begin{itemize}
                \item \textbf{Attore}: Utente
                \item \textbf{Pre-condizioni}: È presente un decision tree valido modificato.
                \item \textbf{Post-condizioni}: Il file del decision tree è esportato.
                \item \textbf{Scenario principale}:
                \begin{itemize}
                    \item L’Utente seleziona la funzionalità esportazione del decision tree.
                    \item Il Sistema serializza il decision tree nel formato previsto.
                    \item Il Sistema salva il file.
                    \item Il Sistema conferma l’esportazione.
                \end{itemize}
            \end{itemize}

            \subsubsection{UC30 - Accesso all’editor del documento del dispositivo}
            \label{uc:uc30}
            \begin{itemize}
                \item \textbf{Attore}: Utente
                \item \textbf{Pre-condizioni}: Un dispositivo con asset associati è visualizzato nel sistema.
                \item \textbf{Post-condizioni}: L’Utente si trova nell’editor del documento dispositivo.
                \item \textbf{Scenario principale}:
                \begin{itemize}
                    \item L’Utente visualizza il resoconto del dispositivo con i suoi assets.
                    \item L’Utente seleziona la funzionalità di modifica del documento.
                    \item Il Sistema carica i dati del dispositivo e degli asset.
                    \item Il Sistema porta l’Utente nell’editor del documento.
                \end{itemize}
                \item \textbf{Scenari alternativi}:
                \begin{itemize}
                    \item L'Utente seleziona il pulsante per il salvataggio delle modifiche riguardo i dati identificativi del dispositivo [\hyperref[uc:uc31]{UC31}]
                \end{itemize}
                \item \textbf{Inclusioni}: \hyperref[uc:uc30.1]{UC30.1 Visualizzazione dati dispositivo in modalitá modifica}
                \item \textbf{Estensioni}: \hyperref[uc:uc31]{UC31 Visualizzazione lista asset in modalità modifica}
            \end{itemize}

            \noindent Il Caso d’Uso UC30 include un ulteriore Caso d’Uso come raffigurato nella seguente immagine:
                
                    \begin{center}
                        \begin{figure}[H]
                            \centering
                            %\includegraphics[scale=0.58]{img/UC30Inclusioni.png}
                            \caption{UC30 -  Inclusioni Caso d'Uso 30: UC30.1}
                            \label{fig:uc30-inclusioni}
                        \end{figure}
                    \end{center}

                    \paragraph{UC30.1 - Visualizzazione dati dispositivo in modalitá modifica} \mbox{}
                    \label{uc:uc30.1}
                    \begin{itemize}
                        \item \textbf{Attore}: Utente
                        \item \textbf{Post-condizioni}: I dati del dispositivo sono visualizzati in forma modificabile.
                        \item \textbf{Scenario principale}:
                        \begin{itemize}
                            \item Il Sistema mostra i campi del dispositivo:
                            \begin{itemize}
                                \item nome
                                \item sistema operativo
                                \item versione firmware
                                \item funzionalità
                                \item descrizione
                            \end{itemize}
                            \item I campi sono modificabili dall’Utente.
                        \end{itemize}
                    \end{itemize}

            \subsubsection{UC31 - Visualizzazione lista asset in modalità modifica}
            \label{uc:uc31}
            \begin{itemize}
                \item \textbf{Attore}: Utente
                \item \textbf{Pre-condizioni}: L'Utente ha salvato le modifiche riguardo i dati identificativi del dispositivo
                \item \textbf{Post-condizioni}: La lista degli asset è visualizzata con funzionalità di selezione e modifica.
                \item \textbf{Scenario principale}:
                \begin{itemize}
                    \item Il Sistema mostra la lista degli asset associati al dispositivo.
                    \item Per ogni asset, il Sistema mostra l’azione di modifica.
                    \item L’Utente può selezionare un asset.
                \end{itemize}
            \end{itemize}

            \subsubsection{UC32 - Selezione asset da modificare}
            \label{uc:uc32}
            \begin{itemize}
                \item \textbf{Attore}: Utente
                \item \textbf{Pre-condizioni}: La lista degli asset è visualizzata.
                \item \textbf{Post-condizioni}: L’asset è selezionato è aperto in modalità modifica.
                \item \textbf{Scenario principale}:
                \begin{itemize}
                    \item L'Utente visualizza la lista degli assets modificabili
                    \item L'Utente seleziona un asset che vuole modificare 
                \end{itemize}
                \item \textbf{Scenari alternativi}:
                \begin{itemize}
                    \item L'Utente seleziona la funzionalità per modificare l'asset selezionato [\hyperref[uc:uc33]{UC33}]
                \end{itemize}
                \item \textbf{Estensioni}: \hyperref[uc:uc33]{UC33 Modifica dati asset}
            \end{itemize}

            \subsubsection{UC33 - Modifica dati asset}
            \label{uc:uc33}
            \begin{itemize}
                \item \textbf{Attore}: Utente
                \item \textbf{Pre-condizioni}: Un asset è stato aperto in modalità modifica.
                \item \textbf{Post-condizioni}: Le modifiche all’asset sono salvate in memoria.
                \item \textbf{Scenario principale}:
                \begin{itemize}
                    \item Il Sistema mostra i campi dell’asset:
                    \begin{itemize}
                        \item nome
                        \item tipo
                        \item descrizione
                        \item sensibilità
                    \end{itemize}
                    \item L’Utente modifica uno o più campi.
                    \item Il Sistema aggiorna i dati dell’asset nello stato corrente.
                \end{itemize}
                \item \textbf{Scenari alternativi}:
                \begin{itemize}
                    \item L'Utente seleziona la funzionalità per annullare la modifica dell'asset [\hyperref[uc:uc34]{UC34}]
                    \item L'Utente seleziona la funzionalità per salvare la modifica dell'asset [\hyperref[uc:uc35]{UC35}]
                \end{itemize}
                \item \textbf{Estensioni}: 
                \begin{itemize}
                    \item \hyperref[uc:uc33]{UC33 Annullamento modifica asset}
                    \item \hyperref[uc:uc33]{UC33 Conferma modifica asset}
                \end{itemize}
            \end{itemize}

            \subsubsection{UC34 - Annullamento modifica asset}
            \label{uc:uc34}
            \begin{itemize}
                \item \textbf{Attore}: Utente
                \item \textbf{Pre-condizioni}: Un asset è in fase di modifica.
                \item \textbf{Post-condizioni}: Le modifiche all'asset sono scartate.
                \item \textbf{Scenario principale}:
                \begin{itemize}
                    \item L’Utente si trova nella sezione di modifica dell'asset.
                    \item L’Utente seleziona il pulsante per l'annullamento della modifica dell'asset.
                    \item Il Sistema ripristina i dati originali dell’asset.
                    \item Il Sistema riporta l’Utente alla lista degli asset.
                \end{itemize}
            \end{itemize}

            \subsubsection{UC35 - Conferma modifica asset}
            \label{uc:uc35}
            \begin{itemize}
                \item \textbf{Attore}: Utente
                \item \textbf{Pre-condizioni}: Un asset è in fase di modifica.
                \item \textbf{Post-condizioni}: Le modifiche all’asset sono salvate in memoria.
                \item \textbf{Scenario principale}:
                \begin{itemize}
                    \item L’Utente seleziona la funzionalità di salvataggio.
                    \item Il Sistema valida i dati inseriti.
                    \item Il Sistema aggiorna l’asset nel documento dispositivo.
                    \item Il Sistema riporta l’Utente alla lista degli asset.
                \end{itemize}
            \end{itemize}

            \subsubsection{UC36 - Salvataggio documento dispositivo modificato}
            \label{uc:uc36}
            \begin{itemize}
                \item \textbf{Attore}: Utente
                \item \textbf{Pre-condizioni}: Il documento dispositivo è stato modificato.
                \item \textbf{Post-condizioni}: Il dispositivo e la lista degli asset sono salvati nel sistema.
                \item \textbf{Scenario principale}:
                \begin{itemize}
                    \item L’Utente seleziona la funzionalità di salvataggio del documento.
                    \item Il Sistema valida il documento complessivo.
                    \item Il Sistema salva il dispositivo e gli asset.
                    \item Il Sistema conferma il salvataggio.
                \end{itemize}
              \item \textbf{Inclusioni}: \hyperref[uc:uc16]{UC16 Salvataggio della lista di assets}
            \end{itemize}

            \subsubsection{UC37 - Uscita dall’editor senza salvataggio}
            \label{uc:uc37}
            \begin{itemize}
                \item \textbf{Attore}: Utente
                \item \textbf{Pre-condizioni}: L’Utente si trova nell’editor del documento.
                \item \textbf{Post-condizioni}: L'Utente si trova allo stato iniziale dell'applicazione.
                \item \textbf{Scenario principale}:
                \begin{itemize}
                    \item L’Utente seleziona il pulsante per l'uscita dall'editor.
                    \item Il Sistema rileva modifiche non salvate.
                    \item Il Sistema chiede conferma.
                    \item L’Utente conferma l’uscita.
                    \item Il Sistema annulla le modifiche 
                    \item Il documento viene cancellato dal sistema 
                    \item Il Sistema porta l'Utente allo stato iniziale dell'applicazione
                \end{itemize}
            \end{itemize}


            
        
    }
}

\newpage
% ----------------------------
% Requisiti
% ----------------------------
\section{Requisiti}{
    Verranno ora descritti i requisiti che ALimitedGroup ha individuato, raggruppati per requisiti:
    \begin{itemize}
        \item \textbf{Funzionali}, ovvero requisiti che rappresentano qualcosa che il Sistema sviluppato deve avere per soddisfare un’aspettativa;
        \item \textbf{Qualità}, ovvero requisiti che devono essere soddisfatti per accertare la qualità di quanto realizzato;
        \item \textbf{Vincolo}, ovvero restrizioni poste al Sistema, quali, a titolo di esempio, sull’uso di alcune tecnologie;
    \end{itemize}
    
    \noindent Per la nomenclatura utilizzata si consiglia di leggere la Sez. 2.2.3.2 delle Norme di Progetto ver. 1.0.0

    \subsection{Requisiti funzionali}{
        \rowcolors{2}{lightblue}{white}
            \begin{longtable}{|l|p{0.6\textwidth}|l|}
                \caption{Tabella dei requisiti funzionali} 
                \label{tab:requisiti-funzionali} \\
                \hline
                \rowcolor{gold}
                \textbf{Codice} & \textbf{Descrizione} & \textbf{Fonti} \\
                \hline
                \endfirsthead
                
                \multicolumn{3}{c}%
                {\tablename\ \thetable\ -- \textit{continua dalla pagina precedente}} \\
                \hline
                \rowcolor{gold}
                \textbf{Codice} & \textbf{Descrizione} & \textbf{Fonti} \\
                \hline
                \endhead
                
                \hline
                \multicolumn{3}{r}{\textit{continua nella pagina successiva}} \\
                \endfoot
                
                \hline
                \endlastfoot
                \hline
                R-1-F-Ob & L'Utente deve poter caricare un dispositivo nel Sistema & \hyperref[uc:uc1]{UC1} \\
                \hline
                R-2-F-Ob & L’Utente deve ricevere un errore in seguito ad un tentativo di caricamento di un file non valido & \hyperref[uc:uc2]{UC2}\\
                \hline
                R-3-F-Ob & L’Utente deve poter visualizzare il dispositivo caricato & \hyperref[uc:uc3]{UC3}\\
                \hline
                R-4-F-Ob & L’Utente deve poter visualizzare il nome del dispositivo caricato & \hyperref[uc:uc3]{UC3}\\
                \hline
                R-5-F-Ob & L'Utente deve poter espandere le informazioni del dispositivo & \hyperref[uc:uc4]{UC4}\\
                \hline
                R-6-F-Ob & L'Utente deve poter visualizzare la lista degli assets del dispositivo & \hyperref[uc:uc4.1]{UC4.1}\\
                \hline
                R-7-F-Ob & L'Utente deve poter visualizzare il nome di ogni assets del dispositivo & \hyperref[uc:uc4.1]{UC4.1}\\
                \hline
                R-8-F-Ob & L'Utente deve poter visualizzare il tipo di ogni assets del dispositivo & \hyperref[uc:uc4.1]{UC4.1}\\
                \hline
                R-9-F-Ob & L'Utente, dopo aver creato manualmente la lista di assets di un dispositivo, deve poter modificare gli assets del dispositivo & \hyperref[uc:uc5]{UC5}, \hyperref[uc:uc16]{UC16}\\
                \hline
                R-10-F-Ob & L'Utente deve poter eliminare un asset del dispositivo & \hyperref[uc:uc5]{UC5}\\
                \hline
                R-11-F-Ob & L'Utente deve poter aggiungere un asset alla lista di assets creata manualmente & \hyperref[uc:uc5]{UC5}\\
                \hline
                R-12-F-Ob & L'Utente deve poter salvare le modifiche apportate alla lista degli asset creata manualmente & \hyperref[uc:uc5]{UC5}\\
                \hline
                R-13-F-Ob & L'Utente deve poter iniziare il test per la verifica dei requisiti della norma per il dispositivo & \hyperref[uc:uc6]{UC6}\\
                \hline
                R-14-F-Ob & L'Utente deve poter visualizzare ogni domanda del test durante la sua esecuzione & \hyperref[uc:uc6.1]{UC6.1}\\
                \hline
                R-15-F-Ob & L'Utente deve poter rispondere ad ogni domanda & \hyperref[uc:uc6]{UC6}\\
                \hline
                R-16-F-Ob & L'Utente deve poter visualizzare i risultati del test una volta completato & \hyperref[uc:uc6.2]{UC6.2}\\
                \hline
                R-17-F-Ob & L'Utente deve poter visualizzare i risultati del test in una lista ordinata & \hyperref[uc:uc6.2]{UC6.2}\\
                \hline
                R-18-F-Ob & L'Utente deve poter visualizzare il singolo risultato con il nome del requisito e l'esito associati & \hyperref[uc:uc6.2.1]{UC6.2.1}\\
                \hline
                R-19-F-Ob & L'Utente deve poter visualizzare PASS se il requisito é stato rispettato & \hyperref[uc:uc6.2.1]{UC6.2.1}\\
                \hline
                R-20-F-Ob & L'Utente deve poter visualizzare FAIL se il requisito non é stato rispettato  & \hyperref[uc:uc6.2.1]{UC6.2.1}\\
                \hline
                R-21-F-Ob & L'Utente deve poter visualizzare NA se il requisito non é applicabile & \hyperref[uc:uc6.2.1]{UC6.2.1}\\
                \hline
                R-22-F-D & L'Utente deve poter visualizzare NA se il requisito non é applicabile & \hyperref[uc:uc6.2.2]{UC6.2.2}\\
                \hline
                R-23-F-Ob & L'Utente deve poter caricare la domanda precedente durante l'esecuzione di un test se non si é alla prima & \hyperref[uc:uc7]{U7}\\
                \hline
                R-24-F-Ob & L'Utente deve poter modificare una domanda a cui ha precedentemente risposto & \hyperref[uc:uc7]{U7}\\
                \hline
                R-25-F-Ob & L'Utente deve poter caricare la domanda successiva durante l'esecuzione di un test se é stato prima selezionato il pulsante per tornare alla domanda precedente & \hyperref[uc:uc8]{UC8}\\
                \hline
                R-26-F-Ob & L'Utente deve poter uscire dal test in qualsiasi momento della sua esecuzione & \hyperref[uc:uc9]{UC9}\\
                \hline
                R-27-F-Ob & L'Utente deve ricevere un salvataggio del test parziale nel caso in cui esca da un test durante la sua esecuzione & \hyperref[uc:uc9.1]{UC9.1}\\
                \hline
                R-28-F-Ob & L'Utente deve poter inserire manualmente i dati di un dispositivo & \hyperref[uc:uc10]{UC10}\\
                \hline
                R-29-F-Ob & L'Utente deve poter visualizzare il form per l'inserimento dei dati del dispositivo & \hyperref[uc:uc10.1]{UC10.1}\\
                \hline
                R-30-F-Ob & L'Utente deve poter inserire il nome del dispositivo nell'apposito form di creazione & \hyperref[uc:uc10.1.1]{UC10.1.1}\\
                \hline
                R-31-F-Ob & L'Utente deve poter inserire il sistema operativo del dispositivo nell'apposito form di creazione & \hyperref[uc:uc10.1.2]{UC10.1.2}\\
                \hline
                R-32-F-Ob & L'Utente deve poter inserire la versione del firmware del dispositivo nell'apposito form di creazione & \hyperref[uc:uc10.1.3]{UC10.1.3}\\
                \hline
                R-33-F-Ob & L'Utente deve poter inserire le funzionalitá del dispositivo nell'apposito form di creazione & \hyperref[uc:uc10.1.4]{UC10.1.4}\\
                \hline
                R-34-F-Ob & L'Utente puó inserire la descrizone del dispositivo nell'apposito form di creazione & \hyperref[uc:uc10.1.5]{UC10.1.5}\\
                \hline
                R-35-F-Ob & L'Utente deve poter salvare il form per l'inserimento manuale dei dati del dispositivo & \hyperref[uc:uc11]{UC11}\\
                \hline
                R-36-F-Ob & L'Utente deve poter inserire manualmente gli asset del dispositivo & \hyperref[uc:uc12]{UC12}\\
                \hline
                R-37-F-Ob & L'Utente deve poter visualizzare il form per l'inserimento dei dati di un singolo asset & \hyperref[uc:uc12.1]{UC12.1}\\
                \hline
                R-38-F-Ob & L'Utente deve poter inserire il nome dell'asset nel form apposito & \hyperref[uc:uc12.1.1]{UC12.1.1}\\
                \hline
                R-39-F-Ob & L'Utente deve poter scegliere il tipo dell'asset nel form apposito & \hyperref[uc:uc12.1.2]{UC12.1.2}\\
                \hline
                R-40-F-Ob & L'Utente puó inserire la descrizione dell'asset nel form apposito & \hyperref[uc:uc12.1.3]{UC12.1.3}\\
                \hline
                R-41-F-Ob & L'Utente deve poter inserire la sensibilitá dell'asset nel form apposito & \hyperref[uc:uc12.1.4]{UC12.1.4}\\
                \hline
                R-42-F-Ob & L'Utente deve poter salvare il form per la creazione manuale dell'asset & \hyperref[uc:uc13]{UC13}\\
                \hline
                R-43-F-Ob & L'Utente deve poter visualizzare la lista di assets creati fino a quel momento & \hyperref[uc:uc13.1]{UC13.1}\\
                \hline
                R-44-F-Ob & L'Utente deve poter visualizzare il nome del singolo asset creato & \hyperref[uc:uc13.1.1]{UC13.1.1}\\
                \hline
                R-45-F-Ob & L'Utente deve poter eliminare il singolo asset creato & \hyperref[uc:uc14]{UC14}\\
                \hline
                R-46-F-Ob & L'Utente deve poter annullare la creazione degli assets & \hyperref[uc:uc15]{UC15}\\
                \hline
                R-47-F-Ob & L'Utente deve poter salvare la lista di assets creati manualmente & \hyperref[uc:uc16]{UC16}\\
                \hline
                R-48-F-Ob & L'Utente deve poter effettuare il caricamento di un test precedente & \hyperref[uc:uc17]{UC17}\\
                \hline
                R-49-F-Ob & L'Utente deve poter annullare la creazione degli assets & \hyperref[uc:uc15]{UC15}\\
                \hline
                R-50-F-Ob & L'Utente deve poter modificare i risultati di un test completato o caricato & \hyperref[uc:uc18]{UC18}\\
                \hline
                R-51-F-Ob & L'Utente deve poter modificare un test a partire da un nodo scelto & \hyperref[uc:uc18]{UC18}\\
                \hline
                R-52-F-Ob & L'Utente deve poter riprendere l'esecuzione di un test non completato & \hyperref[uc:uc19]{UC19}\\
                \hline
                R-53-F-Op & L'Utente deve poter ottenere le giustificazioni riguardo ai risultati di un test completato & \hyperref[uc:uc20]{UC20}\\
                \hline
                R-54-F-Op & L'Utente deve poter accedere all'editor grafico per la modifica del decision tree & \hyperref[uc:uc21]{UC21}\\
                \hline
                R-55-F-Op & L'Utente deve poter visualizzare l'elenco dei decision tree & \hyperref[uc:uc21.1]{UC21.1}\\
                \hline
                R-56-F-Op & L'Utente deve poter visualizzare il singolo decision tree & \hyperref[uc:uc21.1.1]{UC21.1.1}\\
                \hline
                R-57-F-Op & L'Utente deve poter visualizzare l'id del requisito rappresentato dal decision tree & \hyperref[uc:uc21.1.1.1]{UC21.1.1.1}\\
                \hline
                R-58-F-Op & L'Utente deve poter visualizzare il titolo del decision tree & \hyperref[uc:uc21.1.1.2]{UC21.1.1.2}\\
                \hline
                R-59-F-Op & L'Utente deve poter visualizzare le dipendenze del decision tree se presenti & \hyperref[uc:uc21.1.1.3]{UC21.1.1.3}\\
                \hline
            \end{longtable}
    }

    \subsection{Requisiti di qualità}{
        Scrivere i requisiti di qualità qui. I requisiti di qualità sono: le caratteristiche non funzionali che il \textit{sistema}\textsubscript{\href{https://atlasteam9.github.io/Atlas/glossario.html\#Sistema}{G}} deve possedere, come prestazioni, \textit{usabilità}\textsubscript{\href{https://atlasteam9.github.io/Atlas/glossario.html\#Usabilità}{G}}, affidabilità, \textit{sicurezza}\textsubscript{\href{https://atlasteam9.github.io/Atlas/glossario.html\#Sicurezza}{G}}, ecc.

        \begin{table}[H]
        \centering
        \rowcolors{2}{lightblue}{white}
            \begin{tabular}{|l|l|l|l|}
                \hline
                \rowcolor{gold}
                \textbf{Codice} & \textbf{Rilevanza} & \textbf{Descrizione} & \textbf{Fonti} \\
                \hline
                Id requisito & Obbligatorio/Desiderabile/Opzionale & Descrizione & \hyperref[uc:uc1]{UC1} \\
                \hline
                Id requisito & Obbligatorio/Desiderabile/Opzionale & Descrizione & \hyperref[uc:uc1]{UC1}, \hyperref[uc:uc2]{UC2} \\
                \hline
            \end{tabular}
            \caption{Tabella dei requisiti di qualità} 
            \label{tab:requisiti-qualita}
        \end{table}
    }

    \subsection{Requisiti di vincolo}{
        Scrivere i requisiti di vincolo qui. I requisiti di vincolo sono: le limitazioni o condizioni imposte al \textit{sistema}\textsubscript{\href{https://atlasteam9.github.io/Atlas/glossario.html\#Sistema}{G}}, come vincoli tecnologici, normativi, di budget, di tempo, ecc.

        \begin{table}[H]
        \centering
        \rowcolors{2}{lightblue}{white}
            \begin{tabular}{|l|l|l|l|}
                \hline
                \rowcolor{gold}
                \textbf{Codice} & \textbf{Rilevanza} & \textbf{Descrizione} & \textbf{Fonti} \\
                \hline
                Id requisito & Obbligatorio/Desiderabile/Opzionale & Descrizione & \hyperref[uc:uc1]{UC1} \\
                \hline
                Id requisito & Obbligatorio/Desiderabile/Opzionale & Descrizione & \hyperref[uc:uc1]{UC1}, \hyperref[uc:uc2]{UC2} \\
                \hline
            \end{tabular}
            \caption{Tabella dei requisiti di vincolo} 
            \label{tab:requisiti-vincolo}
        \end{table}
    }
}

\newpage

\section{Tracciamento dei Requisiti}{
    \begin{table}[H]
    \centering
    \rowcolors{2}{lightblue}{white}
        \begin{tabular}{|p{5cm}|p{5cm}|}
            \hline
            \rowcolor{gold}
            \textbf{Fonte} & \textbf{Requisito} \\
            \hline
            Capitolato & Id requisito \\
            \hline
            UC1 & Id requisito \\
            \hline
        \end{tabular}
        \caption{Tabella di tracciamento dei requisiti} 
        \label{tab:tracciamento-requisiti}
    \end{table}

    \subsection{Riepilogo}{
        \begin{table}[H]
        \centering
        \rowcolors{2}{lightblue}{white}
            \begin{tabular}{|l|l|l|l|l|}
                \hline
                \rowcolor{gold}
                \textbf{Tipologia} & \textbf{Obbligatori} & \textbf{Desiderabili} & \textbf{Opzionali} & \textbf{Totali} \\
                \hline
                Funzionali & XX & XX & XX & XX \\
                \hline
                Qualità & XX & XX & XX & XX \\
                \hline
                Vincolo & XX & XX & XX & XX \\
                \hline
                \textbf{Totali} & \textbf{XX} & \textbf{XX} & \textbf{XX} & \textbf{XX} \\
                \hline
            \end{tabular}
            \caption{Riepilogo dei requisiti} 
            \label{tab:riepilogo-requisiti}
        \end{table}
    }

    \subsection{Conclusioni}{
        I requisiti individuati sono soggetti a possibili variazioni durante l'evoluzione del progetto, al fine di apportare miglioramenti e aggiornamenti. Nel corso del ciclo di vita del progetto sarà valutata l'opportunità di integrare ulteriori requisiti per elevare qualità complessiva del prodotto. Tali modifiche verranno considerate progressivamente, seguendo un approccio di miglioramento continuo.
    }
}

\end{document}
  


