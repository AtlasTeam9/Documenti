\documentclass[a4paper,12pt]{article}

% ----------------------------
% Pacchetti utili
% ----------------------------
\usepackage[utf8]{inputenc}
\usepackage[T1]{fontenc}
\usepackage[italian]{babel}
\usepackage{graphicx}
\usepackage{tabularx}
\usepackage[table]{xcolor}
\definecolor{lightblue}{RGB}{225,240,255}
\usepackage{geometry}
\usepackage{setspace}
\usepackage{calc}
\usepackage{array}
\usepackage{fancyhdr}
\usepackage{tikz}
\usepackage{float}
\usepackage{pgf-pie}
\usepackage{longtable}
\usepackage[colorlinks=true, linkcolor=blue, urlcolor=blue, citecolor=blue]{hyperref}
\usepackage{pgffor}
\usepackage{textcomp}


% ----------------------------
% Impostazioni pagina
% ----------------------------
\geometry{
    top=2cm,
    bottom=2cm,
    left=2cm,
    right=2cm
}

\setstretch{1.2}

% ----------------------------
% Dati personalizzabili
% ----------------------------
\newcommand{\Gruppo}{Atlas}
\newcommand{\Email}{\href{mailto:team9.atlas@gmail.com}{\textcolor{blue}{\underline{team9.atlas@gmail.com}}}}
\newcommand{\TitoloDocumento}{Glossario}
\newcommand{\DataUltimaModifica}{2026/02/09}
\newcommand{\LogoGruppo}{../../../Assets/AtlasLogo.png} % Inserisci il file del logo

% --- Nuove variabili aggiunte ---
\newcommand{\VersioneDocumento}{v1.0.0} % <-- modifica qui la versione o ID
\newcommand{\TipoDocumento}{Interno} 

\pagestyle{fancy}
\fancyhf{}
\fancyhead[L]{\Gruppo}
\fancyhead[R]{Documento: \TipoDocumento}
\fancyfoot[C]{\thepage}

% larghezza della colonna dei nomi
\newlength{\namecol}
\setlength{\namecol}{4.5cm}
\setlength{\headheight}{15pt} % messo per evitare warning di fancyhdr

\setcounter{secnumdepth}{4}
\setcounter{tocdepth}{4}

\newlength{\colw}
\newlength{\colwS}
\setlength{\colw}{1.5cm}
\setlength{\colwS}{1.2cm}

\definecolor{mioverde}{RGB}{20,150,60}

% ----------------------------
% Inizio documento
% ----------------------------
\begin{document}

% ----------------------------
% Prima pagina
% ----------------------------
\begin{titlepage}

    \begin{center}

        % Logo
        \vspace*{0cm}
        \begin{tikzpicture}
            \clip (0,-0.1) circle (5.6cm);
            \node at (0,0) {\includegraphics[width=12cm]{\LogoGruppo}};
        \end{tikzpicture}\\[0.8cm]

        % Barra superiore
        \noindent\rule{\textwidth}{0.4pt}

        % Titolo
        \vspace{1cm}
        {\Huge \textbf{\TitoloDocumento}}\\[0.4cm]
        {\large Progetto di Ingegneria del Software A.A. 2025/2026}\\[0.8cm]
        {\large Versione: \VersioneDocumento}
        \vspace{1cm}

        % Barra inferiore
        \noindent\rule{\textwidth}{0.4pt}

    \end{center}

    % Informazioni in basso
    \vfill
    \noindent
    \begin{minipage}{0.5\textwidth}
        \raggedright
        \textbf{Autore:} \Gruppo\\
        \textbf{Ultima modifica:} \DataUltimaModifica
    \end{minipage}%
    \begin{minipage}{0.5\textwidth}
        \raggedleft
        \textbf{Tipo di documento:} \TipoDocumento
    \end{minipage}

\end{titlepage}



\section*{Registro delle modifiche}
    \begin{center}
    \rowcolors{2}{lightblue}{white} % Alternanza automatica dal secondo row
        \begin{tabularx}{\textwidth}{|l|l|l|l|X|}
            \hline
            \textbf{Versione} & \textbf{Data} & \textbf{Autore} & \textbf{Verificatore} & \textbf{Descrizione} \\
            \hline
            v1.0.0 & 2025/12/09 & Francesco Marcolongo & Giacomo Giora & Approvazione documento\\
            \hline
            v0.2.3 & 2026/02/06 & Andrea Difino & Giacomo Giora & Rimozione voce 'Sistema' dal documento\\
            \hline
            v0.2.2 & 2025/12/16 & Federico Simonetto & Michele Tesser & Rimozione voce 'Stato' dal documento\\
            \hline
            v0.2.1 & 2025/12/11 & Francesco Marcolongo & Bilal Sabic & Rimozione voce 'Qualità' dal documento\\
            \hline
            v0.2.0 & 2025/11/25 & Andrea Difino & Federico Simonetto & Inserimento termini\\
            \hline
            v0.1.0 & 2025/11/10 & Federico Simonetto & Francesco Marcolongo & Prima stesura documento\\
            \hline
        \end{tabularx}
    \end{center}


\newpage

\tableofcontents


\newpage
% ----------------------------
% Inizio contenuto verbale
% ----------------------------
\section{Introduzione}
    
    \subsection{Scopo del documento}
        All'interno della documentazione prodotta dal team possono comparire termini suscettibili di incomprensioni o ambiguità. Il documento corrente è stato reso disponibile per ovviare a questo problema. Un termine presente nei documenti è consultabile nel Glossario se è indicato con la notazione \textit{parola\textsubscript{\href{https://atlasteam9.github.io/Atlas/glossario.html}{G}}}.
        Premendo sulla G a pedice, l'utente verrà indirizzato alla pagina web del Glossario.
    
    

\newpage

\newpage

        \section{A}
        
          \subsection*{Agile}
			Metodologia di sviluppo basata su iterazioni brevi e feedback continui.
          \subsection*{Ambiente di lavoro}
			Contesto operativo e organizzativo in cui il team svolge le proprie attività progettuali.
          \subsection*{API}
			Insieme di funzioni e protocolli che permettono la comunicazione tra software differenti.
          \subsection*{Architettura}
			Modello concettuale che descrive la struttura, il comportamento e le relazioni tra i componenti di un sistema, garantendo la soddisfazione dei requisiti funzionali e non funzionali. Si può intendere come la progettazione di alto livello e le decisioni fondanti che definiscono come il sistema è organizzato, come i suoi elementi interagiscono e quali principi ne guidano lo sviluppo e l'evoluzione.
          \subsection*{Attore}
			Entità esterna al sistema che interagisce con esso, svolgendo un ruolo specifico e scambiando informazioni.
          \subsection*{Ausilio}
			Aiuto o supporto, sia materiale che morale.
          \subsection*{Automazione}
			Tecnologia che si usa per gestire macchine e processi, riducendo la necessità dell'intervento umano, ovvero per l'esecuzione di operazioni ripetitive o complesse.
        
      
\newpage
        \section{B}
        
          \subsection*{Back-end}
			Parte di un sistema che gestisce logica, dati e comunicazione con il database.
          \subsection*{Baseline}
			Versione approvata di un prodotto o documento che serve come riferimento per successive modifiche.
          \subsection*{Branch}
			Linea di sviluppo separata all'interno di un progetto software.
          \subsection*{Broker}
			Intermediario che facilita la comunicazione e la cooperazione tra sistemi, applicazioni o servizi diversi.
          \subsection*{BT}
			Abbreviazione di Bluetooth, la tecnologia wireless per collegare dispositivi a corto raggio.
        
      

      
\newpage      
        \section{C}
        
          \subsection*{Capitolato}
			Documento stipulato tra il committente (azienda) e il gruppo incaricato dello sviluppo, nel quale viene illustrato il problema che il committente intende risolvere e vengono definiti le norme e i vincoli da seguire durante la progettazione e la realizzazione del prodotto software richiesto. 
          \subsection*{Commit}
			Azione che registra le modifiche apportate ai file in un sistema di versionamento come Git.
          \subsection*{Continuous Delivery}
			Pratica che automatizza il rilascio delle nuove versioni.
          \subsection*{Continuous Integration}
			Pratica che prevede l'integrazione continua del codice in un repository condiviso.
          \subsection*{Cruscotto}
			Strumento che offre una visione sintetica e immediata dello stato delle attività di progetto. Si aggiorna dinamicamente con il progredire dei lavori e consente a chi lo utilizza di ottenere un quadro chiaro e intuitivo della situazione.
      
  \newpage    
        \section{D}
        
          \subsection*{Decision Tree}
			Strumento grafico usato per rappresentare un processo decisionale. Ogni nodo mostra una condizione e ogni ramo mostra le varie alternative o conseguenze.
          \subsection*{Deliverable}
			Prodotto o documento che deve essere consegnato come risultato di una fase di progetto.
          \subsection*{Design Pattern}
			Soluzione riutilizzabile a un problema comune nel design del software.
          \subsection*{Diagramma dei casi d'uso}
			Rappresentazione grafica usata per rappresentare interazioni fra attori e sistema.
          \subsection*{Diagramma delle attività}
			Rappresentazione grafica usata per rappresentare logica procedurale.
          \subsection*{Diagramma delle classi}
			Rappresentazione grafica usata per rappresentare la struttura del progetto.
          \subsection*{Diagramma di Gantt}
			Strumento grafico per la rappresentazione sull'asse temporale delle attività che concorrono al completamento di un progetto, permettendone così la programmazione ed il controllo dell'avanzamento.
          \subsection*{Diario di bordo}
			Documento nel quale vengono mostrate le attività intraprese e segnalate le maggiori difficoltà incontrate in un certo momento del progetto.
          \subsection*{Discord}
			Piattaforma di comunicazione online che permette l'invio di messaggi così come l'avvio di chiamate vocali e video. È organizzato in server e viene utilizzato per collaborare e condividere informazioni.  
      
     \newpage 
        \section{E}
        
          \subsection*{Endpoint}
			URL che rappresenta un punto di accesso a un servizio web o API.
          \subsection*{Entity}
			Elemento persistente del modello dati che rappresenta un oggetto del dominio applicativo.
          \subsection*{Efficacia}
			Capacità del sistema di soddisfare gli obiettivi stabiliti e rispondere correttamente ai bisogni degli stakeholder.
          \subsection*{Efficienza}
			Capacità del sistema o del processo di ottenere il risultato desiderato usando il minor numero possibile di risorse, tempo e sforzo.
          \subsection*{Errore}
			Difetto o imperfezione introdotta nel processo o nel codice che può generare un comportamento inatteso o un malfunzionamento.
          \subsection*{EN 18031}
			Norma europea che definisce requisiti e test di sicurezza per dispositivi IoT, utilizzata per la verifica della loro conformità.
          \subsection*{Esempio}
			Caso illustrativo utilizzato per chiarire il funzionamento di un modello, di una funzione o di un processo.
          \subsection*{Editor}
			Interfaccia che permette la creazione, modifica e gestione strutturata di contenuti, come decision tree o configurazioni.
          \subsection*{Esportazione}
			Funzionalità che consente di generare un output riutilizzabile (documenti, report o dati) in un formato definito.
          \subsection*{Esecuzione automatizzata}
			Processo in cui una serie di controlli, analisi o procedure viene eseguita automaticamente da un sistema senza intervento umano.
          \subsection*{Evolutivo}
			(Di un modello) Approccio allo sviluppo che prevede iterazioni successive nelle quali il sistema viene raffinato e ampliato progressivamente.
          \subsection*{Estensione}
			Meccanismo che permette di aggiungere comportamento opzionale o alternativo a un caso d'uso principale.
          \subsection*{Esecuzione condizionale}
			Esecuzione di un comportamento che avviene solo se una determinata condizione risulta vera.

\newpage    
        \section{F}
        
          \subsection*{Fail}
			Possibile output ottenuto percorrendo un Decision Tree, indica che il requisito non è rispettato.
          \subsection*{Front-end}
			Parte visibile di un'applicazione con cui l'utente interagisce direttamente.
          \subsection*{Framework}
			Struttura software che fornisce strumenti e librerie per lo sviluppo di applicazioni.
        
\newpage 
      
        \section{G}
        
          \subsection*{Git}
			Sistema di controllo versione distribuito per la gestione del codice sorgente.
          \subsection*{GitHub}
			Piattaforma per la collaborazione e la pubblicazione di progetti Git.
          \subsection*{Glossario}
			Elenco dei termini tecnici utilizzati in un progetto, con relative definizioni.
        
\newpage
        \section{H}
        
\newpage
        \section{I}
        
          \subsection*{Interattivo}
			Sistema che richiede la partecipazione attiva dell'utilizzatore, che può rispondere alle azioni per progredire.
          \subsection*{Interfaccia}
			Punto di interazione tra due sistemi o componenti, che consente loro di comunicare e scambiare informazioni.
          \subsection*{Interfaccia grafica}
			(detta anche GUI) è un'interfaccia visiva che permette di interagire con un computer o un'applicazione attraverso elementi grafici come icone, pulsanti, menu e finestre.
          \subsection*{Input}
			Dati, informazioni, istruzioni, risorse, immessi in un sistema.
          \subsection*{IoT}
			Acronimo di "Internet of Things", ovvero "Internet delle cose". Si riferisce a una rete di oggetti fisici, dotati di sensori, software e altre tecnologie, che possono connettersi e scambiare dati tra loro e con altri sistemi su Internet. Questa interconnessione permette il monitoraggio e il controllo remoto dei dispositivi.
        
\newpage
        \section{J}
        
          \subsection*{Just-In-Time}
			Approccio basato sull'idea di produrre o fornire qualcosa solo quando serve, evitando sprechi e riducendo al minimo le risorse inutilizzate.
        
\newpage
        \section{K}
        
\newpage 
        \section{L}
        
          \subsection*{LaTeX}
			Sistema di composizione professionale utilizzato per creare documenti tecnici e scientifici con struttura e formattazione rigorosa.
          \subsection*{Legge di Parkinson}
			Principio secondo cui il lavoro tende ad espandersi fino a occupare tutto il tempo disponibile per completarlo.
          \subsection*{Lettera di presentazione}
			Documento introduttivo che accompagna una consegna formale, spiegandone contenuto, obiettivi e struttura.
          \subsection*{Lezione}
			Unità di attività formativa dedicata a spiegazioni teoriche, chiarimenti o discussioni collegate al progetto.
          \subsection*{Lista}
			Struttura che contiene una serie ordinata di elementi, spesso usata per gestire insiemi di oggetti associati o monitorati.
          \subsection*{Livello di astrazione}
			Gradazione del dettaglio con cui un sistema viene descritto, usata per separare aspetti concettuali da quelli implementativi.
        
\newpage
        \section{M}
        
          \subsection*{Memory Leak}
			Quando un programma usa memoria e non la libera più, anche quando non serve più.
          \subsection*{Merge}
			Operazione Git che integra linee di sviluppo indipendenti in un unico branch.
          \subsection*{Monolite}
			Software costruito tutto insieme, in un unico blocco. Tutte le parti (login, database, API, interfaccia, logica) sono dentro lo stesso programma.
          \subsection*{MQTT}
			Acronimo di Message Queuing Telemetry Transport, è un protocollo di messaggistica leggero, progettato per la comunicazione tra dispositivi con risorse limitate, come quelli dell'IoT.
        
\newpage
        \section{N}
        
          \subsection*{NA}
			Acronimo di Not Applicable: possibile output ottenuto percorrendo un Decision Tree, indica che il requisito non è applicabile.
          \subsection*{Negoziazione}
			Processo di discussione e mediazione tra stakeholder per risolvere conflitti e priorità contrastanti tra diversi requisiti.
          \subsection*{Nodo}
			Unità logica di un decision tree che rappresenta una condizione di decisione o un risultato, a seconda della sua posizione nel processo decisionale.
          \subsection*{Norma}
			Documento che definisce requisiti e procedure per garantire sicurezza, affidabilità, qualità ed efficienza di prodotti, processi e servizi.
        
\newpage
      
        \section{O}
        
          \subsection*{Obiettivo}
			Risultato da raggiungere nell'ambito di un'attività o di un'impresa.
          \subsection*{Operatività}
			Capacità di svolgere determinate attività; efficacia.
        
\newpage
        \section{P}
        
          \subsection*{Pass}
			Possibile output ottenuto percorrendo un Decision Tree, indica che il requisito è stato rispettato.
          \subsection*{PDF}
			Acronimo di Portable Document Format, che significa "formato di documento portatile". È un formato di file che permette di presentare e scambiare documenti in modo affidabile e indipendente.
          \subsection*{Pull}
			Operazione Git che consente di aggiornare il branch locale recuperando e integrando le modifiche dal corrispondente branch remoto.
          \subsection*{Push}
			Operazione Git che trasferisce (propaga) i riferimenti e gli oggetti del repository locale verso un repository remoto.
          \subsection*{Python Packaging}
			Processo di creazione e distribuzione di progetti Python in un formato standardizzato (chiamato package o pacchetto), in modo che altri possano facilmente installarli e utilizzarli. Il file pyproject.toml è un file di configurazione moderno e unificato che gestisce questo processo.
        
\newpage
        \section{Q}
        
\newpage
      
        \section{R}
        
          \subsection*{RED}
			Normativa europea che stabilisce requisiti per la sicurezza, la compatibilità elettromagnetica e l'uso dello spettro radio per le apparecchiature che trasmettono o ricevono onde radio.
          \subsection*{Repository}
			Spazio di archiviazione digitale usato per memorizzare codice, documenti o risorse di progetto. Consente il versionamento dei file e facilità la collaborazione tra i membri del team.
          \subsection*{Requisito}
			Capacità necessaria a un utente per raggiungere un obiettivo (lato bisogno) così come capacità necessaria a un sistema per rispondere a un'aspettativa.
          \subsection*{Responsabile}
			Membro di progetto che coordina le attività di gruppo e approva il rilascio di prodotti parziali o finali.
          \subsection*{Retrospective}
			Attività collocata alla fine di uno sprint. Il team riflette su cosa è andato bene, cosa può essere migliorato e come ottimizzare il lavoro futuro.
          \subsection*{Revisione}
			Attività di controllo che consiste nello scovare eventuali errori prima dell'approvazione.
          \subsection*{Router}
			Dispositivo di rete che indirizza i pacchetti di dati fra diverse reti.
          \subsection*{RTB}
			Acronimo di "Requirements and Technology Baseline": baseline di progetto dove vengono definiti i requisiti funzionali e tecnici insieme alle tecnologie utilizzate per soddisfare tali requisiti.
        
\newpage
      
        \section{S}
        
          \subsection*{Scelte tecnologiche}
			Decisioni relative a tecnologie, librerie e architetture adottate per realizzare il sistema.
          \subsection*{Semplicità}
			Proprietà di un sistema o di un requisito che favorisce chiarezza, facilità di comprensione e riduzione della complessità.
          \subsection*{Server}
			Componente o macchina che gestisce logiche, servizi e funzionalità centralizzate accessibili da client o dispositivi.
          \subsection*{Servizi}
			Funzionalità accessibili dall'esterno o da altri componenti del sistema attraverso interfacce definite.
          \subsection*{Sicurezza}
			Insieme delle proprietà che garantiscono protezione da malfunzionamenti, accessi non autorizzati e rischi operativi.
          \subsection*{Sicurezza informatica}
			Disciplina che si occupa della protezione dei sistemi digitali da attacchi, intrusioni e vulnerabilità.
          \subsection*{Soluzione}
			Insieme delle scelte tecniche e organizzative proposte per risolvere un problema o soddisfare un requisito.
          \subsection*{Stakeholder}
			Persona o entità che ha interesse, influenza o responsabilità nei confronti del progetto o del prodotto.
          \subsection*{Standard di processo}
			Insieme di regole, procedure e convenzioni stabilite per garantire qualità, uniformità e controllo nello sviluppo.
          \subsection*{Stima}
			Valutazione quantitativa preliminare di costi, tempi o risorse necessarie per realizzare un'attività.
          \subsection*{Strumento}
			Tool o applicazione che supporta il team nello svolgimento delle attività di progettazione, sviluppo o verifica.
          \subsection*{Sviluppo}
			Fase del progetto in cui il sistema viene implementato, integrato e verificato sulla base dei requisiti.
        
\newpage
        \section{T}
        
          \subsection*{Ticketing}
			Sistema usato per registrare, organizzare e seguire le richieste di lavoro, problemi o domande.
        
\newpage
        \section{U}
        
          \subsection*{UAT}
			Acronimo di User Acceptance Testing: fase del testing in cui gli utenti finali verificano che il software soddisfi i requisiti contrattuali o di business.
          \subsection*{UI}
			Acronimo di User Interface: insieme di elementi grafici o testuali con cui l'utente interagisce con il software. La progettazione UI impatta sulla gestione del progetto e sulle priorità di sviluppo.
          \subsection*{UML}
			Acronimo di Unified Modeling Language: linguaggio standard di modellazione utilizzato nell'ingegneria del software per rappresentare visivamente sistemi complessi, processi e architetture software. Organizza i modelli in due grandi categorie: diagrammi strutturali, che rappresentano la struttura statica del sistema, e diagrammi comportamentali che rappresentano il comportamento dinamico del sistema.
          \subsection*{Unicode}
			Standard di codifica dei caratteri che assegna in modo univoco un numero (chiamato code point) a ogni carattere usato nei sistemi di scrittura di tutto il mondo.
          \subsection*{Unit test}
			Test automatizzato che verifica il corretto funzionamento di un singolo modulo o componente del software. È parte essenziale della gestione della qualità.
          \subsection*{Upgrade}
			Processo di rilascio di nuove versioni del software, che può includere miglioramenti, correzioni di bug o nuove funzionalità.
          \subsection*{Usabilità}
			Misura di quanto il software è facile da usare, efficiente e comprensibile per l'utente finale.
          \subsection*{Use case}
			Descrizione di come un utente o un sistema interagisce con il software per raggiungere un obiettivo specifico. Fondamentale per raccogliere requisiti funzionali.
          \subsection*{User story}
			Breve descrizione di una funzionalità dal punto di vista dell'utente finale. Molto usata in metodologie Agile/Scrum.
          \subsection*{UX}
			Acronimo di User Experience: esperienza complessiva dell'utente nell'usare il software. Il miglioramento della UX può essere parte degli obiettivi del progetto.
        
\newpage
      
        \section{V}
        
          \subsection*{Validazione}
			Attività che verifica che il prodotto rispetti i requisiti e le aspettative del committente.
          \subsection*{Verbale interno}
			Documento che riporta il resoconto di una riunione interna al team di progetto. Riporta l'ordine del giorno, le questioni discusse e le decisioni prese.
          \subsection*{Verbale esterno}
			Documento che riporta il resoconto di una riunione con l'azienda proponente, uno stakeholder o altre persone esterne al team di progetto.
          \subsection*{Verificatore}
			Membro del team di progetto che si occupa di controllare la correttezza, la completezza e la qualità del lavoro svolto.
          \subsection*{Versionamento}
			Processo di gestione e controllo delle diverse versioni di un progetto.
        
\newpage
        \section{W}
        
          \subsection*{Waterfall model}
			Modello di sviluppo sequenziale in cui ogni fase deve essere completata prima di passare alla successiva.
          \subsection*{WBS}
			Acronimo di Work Breakdown Structure: scomposizione gerarchica del progetto in attività sempre più piccole e gestibili. 
          \subsection*{Web-based}
			Caratteristica di applicazioni accessibili tramite browser senza necessità di installazione locale.
          \subsection*{Wi-Fi}
			Tecnologia di comunicazione wireless utilizzata da molti dispositivi per trasmettere dati tramite rete locale.
          \subsection*{Wireless}
			Comunicazione tra dispositivi elettronici che non fa uso di cavi, ma utilizza radioonde o raggi infrarossi.
          \subsection*{Work}
			Attività che tende ad adattarsi al tempo assegnato, influenzando la pianificazione e la gestione del progetto.
          \subsection*{Work environment}
			Ambiente operativo in cui il team collabora, composto da strumenti, procedure e spazi di lavoro.
          \subsection*{WoW}
			Acronimo di Way of Working: insieme delle pratiche, processi e regole operative adottate dal team per gestire in modo efficace il progetto.          
        
\newpage
      
        \section{X}
        
          \subsection*{XML}
			Linguaggio di markup pensato per rappresentare e scambiare dati in modo strutturato. Grazie ai tag personalizzabili, permette di descrivere le informazioni in modo chiaro e portabile tra sistemi diversi.
        
\newpage
        \section{Y}
        
        
\newpage
        \section{Z}
        
\end{document}