\documentclass[a4paper,12pt]{article}

% ----------------------------
% Pacchetti utili
% ----------------------------
\usepackage[utf8]{inputenc}
\usepackage[T1]{fontenc}
\usepackage[italian]{babel}
\usepackage{graphicx}
\usepackage{xcolor}
\usepackage{geometry}
\usepackage{setspace}
\usepackage{fancyhdr}
\usepackage{tikz}
\usepackage[colorlinks=true, linkcolor=blue, urlcolor=blue, citecolor=blue]{hyperref}
% ----------------------------
% Impostazioni pagina
% ----------------------------
\geometry{
    top=2cm,
    bottom=2cm,
    left=2cm,
    right=2cm
}

\setstretch{1.2}

% ----------------------------
% Dati personalizzabili
% ----------------------------
\newcommand{\Gruppo}{Atlas}
\newcommand{\Email}{\href{mailto:team9.atlas@gmail.com}{\textcolor{blue}{\underline{team9.atlas@gmail.com}}}}
\newcommand{\TitoloVerbale}{Verbale della Riunione}
\newcommand{\DataVerbale}{2025/11/11}
\newcommand{\NumeroVerbaleEsterno}{secondo}
\newcommand{\OraInizio}{15:00}
\newcommand{\OraFine}{16:00}
\newcommand{\LuogoVerbale}{Chiamata Google Meet}
\newcommand{\LogoGruppo}{img/AtlasLogo.png} % Inserisci il file del logo
\newcommand{\AbstractVerbale}{%
In questo verbale vengono riportati gli argomenti discussi, le domande poste e le risposte ricevute durante il \NumeroVerbaleEsterno \space meeting effettuato dal team \Gruppo \space con l'azienda \textbf{Bluewind S.r.l.} nel giorno \DataVerbale \space dalle \OraInizio \space alle \OraFine.
}

% --- Nuove variabili aggiunte ---
\newcommand{\VersioneVerbale}{v0.1.3} % <-- modifica qui la versione o ID
\newcommand{\VerbaleEsterno}{Esterno} 

\pagestyle{fancy}
\fancyhf{}
\fancyhead[L]{\Gruppo}
\fancyhead[R]{Verbale: \VerbaleEsterno \space - \space \DataVerbale}
\fancyfoot[C]{\thepage}

% ----------------------------
% Inizio documento
% ----------------------------
\begin{document}

% ----------------------------
% Prima pagina
% ----------------------------
\begin{titlepage}
    \centering

    % Logo
    \vspace*{0cm}
    %\includegraphics[width=10cm]{\LogoGruppo}\\[.5cm]
    \begin{tikzpicture}
        \clip (0,-0.1) circle (4.6cm); % raggio = metà della larghezza desiderata
        \node at (0,0) {\includegraphics[width=10cm]{\LogoGruppo}};
    \end{tikzpicture}\\
    [.5cm]
    % Titolo
    {\Huge \textbf{\TitoloVerbale}}\\[0.8cm]
    {\LARGE \Gruppo}\\[0.1cm]
    {\Email}\\[1.2cm]

    % Dati riunione
    \begin{tabular}{rl}
        \textbf{Data:} & \DataVerbale \\
        \textbf{Luogo:} & \LuogoVerbale \\
        \textbf{Versione:} & \VersioneVerbale \\
        \textbf{Tipo:} & \VerbaleEsterno \\
    \end{tabular}

    \vspace{1.2cm}

    % Componenti e ruoli
    {\large \textbf{Partecipanti}}\\[0.5cm]
    \begin{tabular}{l|l|l}
        \textbf{Nome} & \textbf{Presenza} & \textbf{Ruolo}\\
        \hline
        \textbf{Alessandro Zappia} & SI (\textbf{Bluewind S.r.l.}) & \textbf{Rappresentante} \\
        \textbf{Tobia Fiorese} & SI (\textbf{Bluewind S.r.l.}) & \textbf{Rappresentante} \\
        Andrea Difino & SI & Responsabile\\
        Federico Simonetto & SI & Progettista\\
        Riccardo Valerio & SI & Progettista\\
        Francesco Marcolongo & SI & Verificatore\\
        Michele Tesser & SI & Amministratore\\
        Giacomo Giora & SI & Analista\\
        Bilal Sabic & SI & Analista\\
    \end{tabular}

\end{titlepage}

\section*{Registro delle modifiche}{
    \begin{center} 
        \begin{tabular}{|l|l|l|l|l|}
            \hline
            \textbf{Versione} & \textbf{Data} & \textbf{Autore} & \textbf{Verificatore} & \textbf{Descrizione} \\
            \hline
            \VersioneVerbale & 2025/11/12 & Giacomo Giora & Francesco Marcolongo & Sistemazione errori \\
            \hline
            v0.1.2 & 2025/11/11 & Andrea Difino & Francesco Marcolongo & Modificato Abstract \\
            \hline
            v0.1.1 & 2025/11/11 & Andrea Difino & Francesco Marcolongo & Piccole modifiche \\
            \hline
            v0.1.0 & 2025/11/11 & Federico Simonetto & Francesco Marcolongo & Prima stesura \\
            \hline
        \end{tabular}
    \end{center}
}

\newpage

\tableofcontents

\newpage
% ----------------------------
% Inizio contenuto verbale
% ----------------------------
\section{Abstract}{
    % Abstract
    \begin{minipage}{0.9\textwidth}
        \small
        \AbstractVerbale
    \end{minipage}
}


\section{Ordine del giorno}{
    \begin{itemize}
        \item Esposizione di domande riguardanti il progetto e le interazioni fra il team Atlas e l'azienda proponente
    \end{itemize}
}


\section{Discussione}{
    \subsection{Tecnologie da utilizzare nel progetto}
    \textbf{Domanda:} Ci sono delle tecnologie precise che la proponente desidera vengano impiegate nello sviluppo del progetto?

    \textbf{Risposta:} Viene lasciata molta libertà nella scelta delle tecnologie. Le principali caratteristiche che il prodotto software dovrà possedere sono modularità, comprensibilità e documentabilità. Viene richiesta la persistenza dei dati: in caso di riavvio dell'applicazione, l'utente deve poter ricaricare il progetto su cui stava lavorando. Questo 
    può essere implementato tramite l'uso di un database oppure, più semplicemente, mediante il salvataggio su file.

    % ----------------------------------------------------

    \subsection{Condivisione casi d'uso e documentazione}
    \label{subsec:docu}
    \textbf{Domanda:} È possibile ricevere la documentazione dello standard così come il caso d'uso nominato nel capitolato?

    \textbf{Risposta:} Viene fornito un link al cloud aziendale dove sono presenti la documentazione del RED act e dello standard EN18031, nonché il caso d'uso della Macchina del Caffè menzionato nel capitolato di presentazione. Inoltre, viene condivisa una repository Github contentente altra documentazione tecnica e dei templates di test utili per lo sviluppo del progetto.

    % ----------------------------------------------------

    \subsection{Frequenza degli incontri con l'azienda proponente}
    \textbf{Domanda:} Quanto spesso verranno svolti gli incontri tra il team Atlas e l'azienda proponente? Secondo quali modalità?

    \textbf{Risposta:} Le riunioni possono essere svolte comodamente da remoto tramite piattaforme come Google Meet e Zoom. Per quanto riguarda la frequenza, nella prima parte del progetto le riunioni
    potranno avere luogo ogni due settimane, aumentando successivamente durante la fase di codifica del prodotto. È stato inoltre deciso di utilizzare Telegram come canale di comunicazione aggiuntivo per poter ricevere feedback
    immediati su questioni di minore entità. In seguito alla condivisione dei documenti citati nella sezione \ref{subsec:docu} viene fissata una riunione per il 18 novembre 2025 per chiarire eventuali dubbi a riguardo.

    % ----------------------------------------------------

    \subsection{Percorso di vita dell'utente nell'applicazione}
    \textbf{Domanda:} Quale sarà il percorso di vita dell'utente all'interno dell'applicazione? È necessario fornire un pannello di autenticazione?

    \textbf{Risposta:} Inizialmente, l'utente fornisce le informazioni sul dispositivo tramite file. L'applicazione presenta le domande dei decision tree, l'utente interagisce inserendo le risposte che verranno poi salvate e alla fine analizzate per generare un output preciso. 
    Questo dovrà poi essere trascritto su file e salvato all'esterno. L'autenticazione non viene richiesta, ma potrebbe essere considerata un requisito opzionale. 

    % ----------------------------------------------------

    \subsection{Decision tree}
    \textbf{Domanda:} Si possono avere più informazioni riguardo ai decision tree?

    \textbf{Risposta:} I decision tree dovranno essere implementati a partire dalle descrizioni presenti all'interno della documentazione.\newline
    I decision tree sono gli stessi per ogni dispositivo. \newline 
    L'utente deve sapere come rispondere ai decision tree; non è necessario fornire ulteriori informazioni aggiuntive.\newline 
    Tutti i decision tree devono essere eseguibili per ogni dispositivo. \newline
    Le domande all'interno dei decision tree devono essere presentate con una sequenza logica ben definita poiché la risposta ad alcuni requisiti abilita domande relative ad altri requisiti.

    % ----------------------------------------------------

    \subsection{Interfaccia grafica}
    \textbf{Domanda:} Ci sono idee di design per quanto riguarda la User Interface?

    \textbf{Risposta:} Viene lasciata totale libertà al team di progetto per quanto riguarda il design della UI.

}
\newpage

\section{Decisioni prese}{
    \begin{center}
    \begin{tabular}{|c|p{12cm}|}
        \hline
        \textbf{ID} & \textbf{Decisione} \\
        \hline
            D1-\DataVerbale\_ve & Condivisione da parte dell'azienda proponente di documenti riguardanti il capitolato, finalizzata ad attività di studio. \\
        \hline
            D2-\DataVerbale\_ve & Fissata nuova riunione tramite Google Meet con l'azienda Bluewind S.r.l. in data 2025/11/18. \\
        \hline 
            D3-\DataVerbale\_ve & Decisa la creazione di un gruppo Telegram per poter contattare in modo istantaneo l'azienda proponente.\\    
        \hline
    \end{tabular}
    \end{center}
}

\section{Attivitá da svolgere}{
    \begin{center}
    \begin{tabular}{|c|p{4.5cm}|c|p{3cm}|} 
        \hline
        \textbf{ID} & \textbf{Descrizione} & \textbf{Id Github Issue} & \textbf{Assegnatario}\\
        \hline
            A1-\DataVerbale\_ve &  Iniziare lo studio della direttiva RED, dello standard EN18031 e dei decision tree. & - & Team \\
        \hline
            A2-\DataVerbale\_ve & Brainstorming per iniziare a decidere le tecnologie da utilizzare. & - & Team \\
        \hline
            A3-\DataVerbale\_ve & Creare il gruppo Telegram per effettuare contatto istantaneo con l'azienda proponente. & - & Andrea Difino \\
        \hline
    \end{tabular}
    \end{center}
}

\vspace{2cm}
\begin{flushright}
    \textbf{Approvazione dell'azienda} \\
    Il proponente,\\[0.5cm]
    \rule{6cm}{0.4pt}\\
\end{flushright}



\end{document}