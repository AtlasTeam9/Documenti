\documentclass[a4paper,12pt]{article}

% ----------------------------
% Pacchetti utili
% ----------------------------
\usepackage[utf8]{inputenc}
\usepackage[T1]{fontenc}
\usepackage[italian]{babel}
\usepackage{graphicx}
\usepackage{xcolor}
\usepackage{geometry}
\usepackage{setspace}
\usepackage{fancyhdr}
\usepackage{tikz}
\usepackage[colorlinks=true, linkcolor=blue, urlcolor=blue, citecolor=blue]{hyperref}
% ----------------------------
% Impostazioni pagina
% ----------------------------
\geometry{
    top=2cm,
    bottom=2cm,
    left=2cm,
    right=2cm
}

\setstretch{1.2}

% ----------------------------
% Dati personalizzabili
% ----------------------------
\newcommand{\Gruppo}{Atlas}
\newcommand{\Email}{\href{mailto:team9.atlas@gmail.com}{\textcolor{blue}{\underline{team9.atlas@gmail.com}}}}
\newcommand{\TitoloVerbale}{Verbale della Riunione}
\newcommand{\DataVerbale}{2025/12/02}
\newcommand{\NumeroVerbaleEsterno}{quinto}
\newcommand{\OraInizio}{15:00}
\newcommand{\OraFine}{15:45}
\newcommand{\LuogoVerbale}{Chiamata Google Meet}
\newcommand{\LogoGruppo}{img/AtlasLogo.png} % Inserisci il file del logo
\newcommand{\AbstractVerbale}{%
In questo verbale vengono riportati gli argomenti discussi, le domande poste e le risposte ricevute durante il \NumeroVerbaleEsterno \space meeting effettuato dal team \Gruppo \space con l'azienda \textbf{Bluewind S.r.l.} nel giorno \DataVerbale \space dalle \OraInizio \space alle \OraFine.
}

% --- Nuove variabili aggiunte ---
\newcommand{\VersioneVerbale}{v1.0.0} % <-- modifica qui la versione o ID
\newcommand{\VerbaleEsterno}{Esterno} 

\pagestyle{fancy}
\fancyhf{}
\fancyhead[L]{\Gruppo}
\fancyhead[R]{Verbale: \VerbaleEsterno \space - \space \DataVerbale}
\fancyfoot[C]{\thepage}

% ----------------------------
% Inizio documento
% ----------------------------
\begin{document}

% ----------------------------
% Prima pagina
% ----------------------------
\begin{titlepage}
    \centering

    % Logo
    \vspace*{0cm}
    %\includegraphics[width=10cm]{\LogoGruppo}\\[.5cm]
    \begin{tikzpicture}
        \clip (0,-0.1) circle (4.6cm); % raggio = metà della larghezza desiderata
        \node at (0,0) {\includegraphics[width=10cm]{\LogoGruppo}};
    \end{tikzpicture}\\
    [.5cm]
    % Titolo
    {\Huge \textbf{\TitoloVerbale}}\\[0.8cm]
    {\LARGE \Gruppo}\\[0.1cm]
    {\Email}\\[1.2cm]

    % Dati riunione
    \begin{tabular}{rl}
        \textbf{Data:} & \DataVerbale \\
        \textbf{Luogo:} & \LuogoVerbale \\
        \textbf{Versione:} & \VersioneVerbale \\
        \textbf{Tipo:} & \VerbaleEsterno \\
    \end{tabular}

    \vspace{1.2cm}

    % Componenti e ruoli
    {\large \textbf{Partecipanti}}\\[0.5cm]
    \begin{tabular}{l|l|l}
        \textbf{Nome} & \textbf{Presenza} & \textbf{Ruolo}\\
        \hline
        \textbf{Alessandro Zappia} & SI (\textbf{Bluewind S.r.l.}) & \textbf{Rappresentante} \\
        \textbf{Tobia Fiorese} & SI (\textbf{Bluewind S.r.l.}) & \textbf{Rappresentante} \\
        Andrea Difino & SI & Analista\\
        Federico Simonetto & SI & Responsabile\\
        Riccardo Valerio & SI & Verificatore\\
        Francesco Marcolongo & SI & Amministratore\\
        Michele Tesser & SI & Progettista\\
        Giacomo Giora & SI & Programmatore\\
        Bilal Sabic & SI & Verificatore\\
    \end{tabular}

\end{titlepage}

\section*{Registro delle modifiche}{
    \begin{center} 
        \begin{tabular}{|l|l|l|l|l|}
        \hline
            \textbf{Versione} & \textbf{Data} & \textbf{Autore} & \textbf{Verificatore} & \textbf{Descrizione} \\
            \hline
            \VersioneVerbale & 2025/12/02 & Michele Tesser & Bilal Sabic, Riccardo Valerio & Firma del documento \\
            \hline
            v0.1.0 & 2025/12/02 & Michele Tesser &  & Prima stesura \\
            \hline
        \end{tabular}
    \end{center}
}

\newpage

\tableofcontents

\newpage
% ----------------------------
% Inizio contenuto verbale
% ----------------------------
\section{Abstract}{
    % Abstract
    \begin{minipage}{0.9\textwidth}
        \small
        \AbstractVerbale
    \end{minipage}
}


\section{Ordine del giorno}{
    \begin{itemize}
        \item Esposizione di domande riguardanti il progetto, i prototipi dei casi d'uso e dei requisiti funzionali e non.
    \end{itemize}
}


\section{Discussione}{

    % ----------------------------------------------------
    \subsection{Superivisione casi d'uso}
    \textbf{Domanda:} Abbiamo effettuato una prima stesura dei casi d'uso. Sono corretti? Ci consigliate alcuni cambiamenti?
    
    \textbf{Risposta:} I casi d'uso risultano nel complesso corretti; tuttavia, l'azienda propone alcune modifiche e integrazioni.
    In particolare, considerando la possibilità di aggiungere asset direttamente tramite la web app, è necessario introdurre un'opzione che consenta di salvare gli asset inseriti all'interno di un file.\newline
    Inoltre, durante l'esecuzione dei decision tree, dovrebbe essere previsto un pulsante per tornare al passo precedente, così da permettere la correzione di eventuali risposte inserite in modo errato.\newline
    Infine, nel caso in cui l'utente apra, tramite la web app, un file relativo a un test precedente, questo non dovrà presentare i risultati in modalità sola lettura, bensì dovranno essere pienamente modificabili.
    Qualora venga apportata una modifica a uno specifico requisito, anche i requisiti successivi che dipendono da quello modificato dovranno essere nuovamente proposti all'utente, così da garantire la coerenza dell'intero processo di valutazione.

    % ----------------------------------------------------
    \subsection{Supervisione requisiti}
    \textbf{Domanda:} Abbiamo effettuato una prima stesura dei requisiti. Sono corretti? Ci consigliate alcuni cambiamenti?

    \textbf{Risposta:}  I requisiti presentati sono in buona parte corretti, ci sono alcune modifiche richieste dall'azienda. In particolare: sono sufficienti metodi molto semplicistici per il salvataggio dello stato di avanzamento, in contrasto con i metodi proposti più vicini ai DB.
    % ----------------------------------------------------
    \subsection{PoC}
    \textbf{Domanda:} Per il Proof of Concept, in riferimento alla parte grafica, che quantità di funzionalità vi aspettate?
    
    \textbf{Risposta:} Per il Proof of Concept, nella parte grafica è richiesta solamente una dimostrazione del funzionamento della tecnologia scelta, tutte le funzionalità proprie dell'applicazione potranno essere verificabili attraverso una CLI. 
    % ----------------------------------------------------
    \subsection{Deployment}
    \textbf{Domanda:} La vostra azienda offrirà metodi di hosting per la nostra web app oppure preferite utilizzare altri metodi di deployment?
    
    \textbf{Risposta:} L'azienda potrebbe mettere a disposizione una macchina per l'hosting dell'applicazione; tuttavia si preferisce l'adozione di soluzioni basate sulla containerizzazione come Docker o Podman.
    % ----------------------------------------------------


}
\newpage

\section{Decisioni prese}{
    \begin{center}
    \begin{tabular}{|c|p{12cm}|}
        \hline
        \textbf{ID} & \textbf{Decisione} \\
        \hline
            D1-\DataVerbale\_ve & Fissata la successiva riunione in data 2025/12/16 alle ore 15.00 \\
        \hline
            D2-\DataVerbale\_ve & Deciso utilizzo di GitHub come sistema di versionamento per il prodotto software \\
        \hline
    \end{tabular}
    \end{center}
}

\section{Attivitá da svolgere}{
    \begin{center}
    \begin{tabular}{|c|p{4.5cm}|c|p{3cm}|} 
        \hline
        \textbf{ID} & \textbf{Descrizione} & \textbf{Id Github Issue} & \textbf{Assegnatario}\\
        \hline
            A1-\DataVerbale\_ve & Creare alcuni Decision Tree in TOML & \#57 & Team \\
        \hline
            A2-\DataVerbale\_ve & Creare un template in JSON / CSV per gli assets & \#58 & Team \\
        \hline
            A3-\DataVerbale\_ve & Creare il supporto al versionamento & \#59 & Team \\
        \hline
    \end{tabular}
    \end{center}
}

\vspace{2cm}
\begin{flushright}
    \textbf{Approvazione dell'azienda} \\
    Il proponente,\\[0.5cm]
    \rule{6cm}{0.4pt}\\
\end{flushright}


\end{document}
