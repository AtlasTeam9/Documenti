\documentclass[a4paper,12pt]{article}

% ----------------------------
% Pacchetti utili
% ----------------------------
\usepackage[utf8]{inputenc}
\usepackage[T1]{fontenc}
\usepackage[italian]{babel}
\usepackage{graphicx}
\usepackage{xcolor}
\usepackage{geometry}
\usepackage{setspace}
\usepackage{fancyhdr}
\usepackage{tikz}
\usepackage[colorlinks=true, linkcolor=blue, urlcolor=blue, citecolor=blue]{hyperref}
% ----------------------------
% Impostazioni pagina
% ----------------------------
\geometry{
    top=2cm,
    bottom=2cm,
    left=2cm,
    right=2cm
}

\setstretch{1.2}

% ----------------------------
% Dati personalizzabili
% ----------------------------
\newcommand{\Gruppo}{Atlas}
\newcommand{\Email}{\href{mailto:team9.atlas@gmail.com}{\textcolor{blue}{\underline{Email}}}}
\newcommand{\TitoloVerbale}{Verbale della Riunione}
\newcommand{\DataVerbale}{2025/11/12}
\newcommand{\OraInizio}{16:00}
\newcommand{\OraFine}{17:00}
\newcommand{\LuogoVerbale}{Chiamata Discord}
\newcommand{\LogoGruppo}{img/AtlasLogo.png} % Inserisci il file del logo
\newcommand{\AbstractVerbale}{%
In questo verbale vengono riportati i principali argomenti discussi, le decisioni prese e le attività pianificate durante la riunione del gruppo nel giorno \DataVerbale \space dalle \OraInizio \space alle \space \OraFine .
}

% --- Nuove variabili aggiunte ---
\newcommand{\VersioneVerbale}{v0.1.0} % <-- modifica qui la versione o ID
\newcommand{\VerbaleInterno}{Interno} 

\pagestyle{fancy}
\fancyhf{}
\fancyhead[L]{\Gruppo}
\fancyhead[R]{Verbale: \VerbaleInterno \space - \space \DataVerbale}
\fancyfoot[C]{\thepage}


% ----------------------------
% Inizio documento
% ----------------------------
\begin{document}

% ----------------------------
% Prima pagina
% ----------------------------
\begin{titlepage}
    \centering

    % Logo
    \vspace*{0cm}
    %\includegraphics[width=10cm]{\LogoGruppo}\\[.5cm]
    \begin{tikzpicture}
        \clip (0,-0.1) circle (4.6cm); % raggio = metà della larghezza desiderata
        \node at (0,0) {\includegraphics[width=10cm]{\LogoGruppo}};
    \end{tikzpicture}\\
    [.5cm]
    % Titolo
    {\Huge \textbf{\TitoloVerbale}}\\[0.8cm]
    {\LARGE \Gruppo}\\[0.1cm]
    {\Email}\\[1.2cm]

    % Dati riunione
    \begin{tabular}{rl}
        \textbf{Data:} & \DataVerbale \\
        \textbf{Luogo:} & \LuogoVerbale \\
        \textbf{Versione:} & \VersioneVerbale \\
        \textbf{Tipo:} & \VerbaleInterno \\
    \end{tabular}

    \vspace{1.2cm}

    % Componenti e ruoli
    {\large \textbf{Partecipanti}}\\[0.5cm]
    \begin{tabular}{l|l|l}
        \textbf{Nome} & \textbf{Presenza} & \textbf{Ruolo}\\
        \hline
        Andrea Difino & SI & Responsabile\\
        Federico Simonetto & SI & Progettista\\
        Riccardo Valerio & SI & Progettista\\
        Francesco Marcolongo & SI & Verificatore\\
        Michele Tesser & NO & Amministratore\\
        Giacomo Giora & SI & Analista\\
        Bilal Sabic & SI & Analista\\
    \end{tabular}

\end{titlepage}

\section*{Registro delle modifiche}{
    \begin{center} 
        \begin{tabular}{|l|l|l|l|l|}
            \hline
            \textbf{Versione} & \textbf{Data} & \textbf{Autore} & \textbf{Verificatore} & \textbf{Descrizione} \\
            \hline
              \VersioneVerbale&\DataVerbale &Bilal Sabic & Francesco Marcolongo&Prima stesura \\
            \hline
        \end{tabular}
    \end{center}
}

\newpage

\tableofcontents

\newpage
% ----------------------------
% Inizio contenuto verbale
% ----------------------------
\section{Abstract}{
    % Abstract
    \begin{minipage}{0.9\textwidth}
        \small
        \AbstractVerbale
    \end{minipage}
}


\section{Ordine del Giorno}{
    \begin{enumerate}
        \item Discussione sul way of working
        \item Discussione sulle tecnologie da utilizzare
        \item Creazione di un canale di comunicazione con l'azienda Bluewind
    \end{enumerate}
}

\section{Discussione}{
 	\subsection{Discussione sul way of working} {
	Il gruppo ha discusso del Way of Working e delle possibili modalità per ottimizzarlo, insieme a proposte di miglioramento riguardanti la struttura della repository GitHub. Infatti si è discusso in particolare sul problema della concorrenza. 
	}
	\subsection{Discussione sulle tecnologie da utilizzare} {
	Il gruppo ha discusso sulle varie tecnologie da utilizzare per lo sviluppo del progetto, tenendo conto delle competenze attuali dei membri del team.
	}
	\subsection{Discussione sul canale di comunicazione con l'azienda Bluewind} {
	Il gruppo ha creato un canale Telegram per mantenere una comunicazione più diretta e immediata con l’azienda.
	}
	
}

\section{Decisioni Prese}{
    \begin{center}
    \begin{tabular}{|c|p{12cm}|}
        \hline
        \textbf{ID} & \textbf{Decisione} \\
        \hline
            D\{1\}-\DataVerbale\_vi & Ridefinita la struttura della repo per risolvere il problema della concorrenza  \\
        \hline
	D\{2\}-\DataVerbale\_vi & Deciso Telegram come canale di comuncazione con l'azienda \\
        \hline
    \end{tabular}
    \end{center}
}

\section{Attivitá da svolgere}{
    \begin{center}
    \begin{tabular}{|c|p{4.5cm}|c|p{3cm}|} 
        \hline
        \textbf{ID} & \textbf{Descrizione} & \textbf{Id Github Issue} & \textbf{Assegnatario}\\
        \hline
            A\{1\}-\DataVerbale\_vi & Definire che tecnologie da utilizzare per  il \newline progetto  & -  & Team \\
        \hline
	 A\{2\}-\DataVerbale\_vi & Continuare a scrivere il glossario  & -  & Team \\
        \hline
	A\{3\}-\DataVerbale\_vi & Studiare la documentazione condivisa \newline dall'azienda  & -  & Team \\
        \hline
    \end{tabular}
    \end{center}
}



\end{document}
