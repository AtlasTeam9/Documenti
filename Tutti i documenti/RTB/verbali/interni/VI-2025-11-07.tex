\documentclass[a4paper,12pt]{article}

% ----------------------------
% Pacchetti utili
% ----------------------------
\usepackage[utf8]{inputenc}
\usepackage[T1]{fontenc}
\usepackage[italian]{babel}
\usepackage{graphicx}
\usepackage{xcolor}
\usepackage{geometry}
\usepackage{setspace}
\usepackage{fancyhdr}
\usepackage{tikz}
\usepackage[colorlinks=true, linkcolor=blue, urlcolor=blue, citecolor=blue]{hyperref}
% ----------------------------
% Impostazioni pagina
% ----------------------------
\geometry{
    top=2cm,
    bottom=2cm,
    left=2cm,
    right=2cm
}

\setstretch{1.2}

% ----------------------------
% Dati personalizzabili
% ----------------------------
\newcommand{\Gruppo}{Atlas}
\newcommand{\Email}{\href{mailto:team9.atlas@gmail.com}{\textcolor{blue}{\underline{team9.atlas@gmail.com}}}}
\newcommand{\TitoloVerbale}{Verbale della Riunione}
\newcommand{\DataVerbale}{2025/11/07}
\newcommand{\OraInizio}{9:30}
\newcommand{\OraFine}{11:45}
\newcommand{\LuogoVerbale}{Chiamata Discord}
\newcommand{\LogoGruppo}{../../../Assets/AtlasLogo.png} % Inserisci il file del logo
\newcommand{\AbstractVerbale}{%
In questo verbale vengono riportati i principali argomenti discussi, le decisioni prese e le attività pianificate durante la riunione del gruppo nel giorno \DataVerbale \space dalle \OraInizio \space alle \space \OraFine .
}

% --- Nuove variabili aggiunte ---
\newcommand{\VersioneVerbale}{v1.0.1} % <-- modifica qui la versione o ID
\newcommand{\VerbaleInterno}{Interno} 

\pagestyle{fancy}
\fancyhf{}
\fancyhead[L]{\Gruppo}
\fancyhead[R]{Verbale: \VerbaleInterno \space - \space \DataVerbale}
\fancyfoot[C]{\thepage}


% ----------------------------
% Inizio documento
% ----------------------------
\begin{document}

% ----------------------------
% Prima pagina
% ----------------------------
\begin{titlepage}
    \centering

    % Logo
    \vspace*{0cm}
    %\includegraphics[width=10cm]{\LogoGruppo}\\[.5cm]
    \begin{tikzpicture}
        \clip (0,-0.1) circle (4.6cm); % raggio = metà della larghezza desiderata
        \node at (0,0) {\includegraphics[width=10cm]{\LogoGruppo}};
    \end{tikzpicture}\\
    [.5cm]
    % Titolo
    {\Huge \textbf{\TitoloVerbale}}\\[0.8cm]
    {\LARGE \Gruppo}\\[0.1cm]
    {\Email}\\[1.2cm]

    % Dati riunione
    \begin{tabular}{rl}
        \textbf{Data:} & \DataVerbale \\
        \textbf{Luogo:} & \LuogoVerbale \\
        \textbf{Versione:} & \VersioneVerbale \\
        \textbf{Tipo:} & \VerbaleInterno \\
    \end{tabular}

    \vspace{1.2cm}

    % Componenti e ruoli
    {\large \textbf{Partecipanti}}\\[0.5cm]
    \begin{tabular}{l|l|l}
        \textbf{Nome} & \textbf{Presenza} & \textbf{Ruolo}\\
        \hline
        Andrea Difino & SI & Responsabile\\
        Federico Simonetto & SI & Amministratore\\
        Riccardo Valerio & SI & Verificatore\\
        Francesco Marcolongo & SI & Verificatore\\
        Michele Tesser & SI & Amministratore\\
        Giacomo Giora & SI & Analista\\
        Bilal Sabic & SI & Analista\\
    \end{tabular}

\end{titlepage}

\section*{Registro delle modifiche}{
    \begin{center}
        \begin{tabular}{|l|l|l|l|l|}
            \hline
            \textbf{Versione} & \textbf{Data} & \textbf{Autore} & \textbf{Verificatore} & \textbf{Descrizione} \\
            \hline
            \VersioneVerbale & 2025/11/14 & Federico Simonetto & Riccardo Valerio & Piccole correzioni \\
            \hline
            v1.0.0 & \DataVerbale & Federico Simonetto & Francesco Marcolongo & Aggiunta ruoli e \\
            & & & & tabella azioni \\
            \hline
            v0.2.0 & \DataVerbale & Giacomo Giora & Francesco Marcolongo & Sistemazione contenuto \\
            \hline
            v0.1.0 & \DataVerbale & Giacomo Giora &  & Prima stesura \\
            \hline
        \end{tabular}
    \end{center}
}

\newpage

\tableofcontents

\newpage
% ----------------------------
% Inizio contenuto verbale
% ----------------------------
\section{Abstract}{
    % Abstract
    \begin{minipage}{0.9\textwidth}
        \small
        \AbstractVerbale
    \end{minipage}
}


\section{Ordine del giorno}{
    \begin{enumerate}
        \item Discussione su aggiudicazione appalti
        \item Assegnazione ruoli per il primo sprint
        \item Modifica al sistema di versionamento dei documenti
        \item Discussione sulla gestione dei branch sulla repo Github
        \item Creazione project board su Github
        \item Discussione sull'utilizzo di un tool per la visualizzazione del cruscotto di progetto
        \item Confronto su documento "Glossario" da redigere
    \end{enumerate}
}

\section{Discussione}{
	\subsection{Discussione su aggiudicazione appalti}{
		Il gruppo ha discusso riguardo l'appalto aggiudicato, \textbf{C1 - Automated EN18031 Compliance Verification}, e ha cercato di capire le richieste di integrazione, concentrandosi in particolare sulla gestione delle sezioni del sito web del team.
    }
    \subsection{Assegnazione ruoli per il primo sprint}{
        Sono stati assegnati i ruoli per il primo sprint, con particolare attenzione ai ruoli necessari nella prima fase del progetto, come l'Analista e il Progettista che disporranno di due membri ciascuno. Il ruolo di Programmatore invece non è stato assegnato in quanto non necessario in questa fase iniziale.
    }
	\subsection{Modifica al sistema di versionamento dei documenti}{
		Il gruppo ha discusso le modifiche proposte al sistema di versionamento dei documenti, con particolare attenzione all'adozione di nuove convenzioni.
    }
	\subsection{Discussione sulla gestione dei branch sulla repo Github}{
		Il gruppo ha esaminato le attuali pratiche di gestione dei branch sulla repo Github e ha proposto miglioramenti per facilitare la collaborazione.
    }
    \subsection{Creazione project board su Github}{
        Verrà creata una project board su Github per migliorare la gestione delle attività e il monitoraggio del progresso del progetto, discutendo su quante e quali fasi creare, e su come gestire le dipendenze tra di esse.
    }
    \subsection{Discussione sull'utilizzo di un tool per la visualizzazione del cruscotto di progetto}{
        Il gruppo ha iniziato una discussione sull'adozione di un nuovo strumento per la visualizzazione del cruscotto di progetto, Grafana, valutandone i pro e i contro.
    }
    \subsection{Confronto su documento "Glossario" da redigere}{
        Il gruppo ha discusso la struttura e i contenuti del documento "Glossario" da redigere, definendo le linee guida per la sua compilazione.
    }
}


\section{Decisioni prese}{
    \begin{center}
    \begin{tabular}{|c|p{11cm}|}
        \hline
        \textbf{ID} & \textbf{Decisione} \\
        \hline
            D1-2025/11/07\_vi &  Definiti i seguenti ruoli per il primo sprint: Responsabile - Andrea Difino, Amministratore - Michele Tesser, Analista - Giacomo Giora e Bilal Sabic, Progettista - Riccardo Valerio e Federico Simonetto, Verificatore - Francesco Marcolongo.\\
        \hline
            D2-2025/11/07\_vi & Deciso che il nuovo sistema di versionamento sarà nella forma X.Y.Z. \\
		\hline
			D3-2025/11/07\_vi & Deciso l'utilizzo di una project board su Github per migliorare la gestione delle attività. \\
		\hline
    \end{tabular}
    \end{center}
}

\section{Attivitá da svolgere}{
    \begin{center}
    \begin{tabular}{|c|p{4.5cm}|c|p{3cm}|} 
        \hline
        \textbf{ID} & \textbf{Descrizione} & \textbf{Id Github Issue} & \textbf{Assegnatario}\\
        \hline
            A1-2025/11/07\_vi &  Creare la project board su GitHub. & - & Andrea Difino \\
        \hline
            A2-2025/11/07\_vi & Modificare il way of \newline working. & - & Francesco \newline Marcolongo \\
        \hline
            A3-2025/11/07\_vi & Stesura iniziale del \newline glossario. & - & Team \\
        \hline
    \end{tabular}
    \end{center}
}



\end{document}
