\documentclass[a4paper,12pt]{article}

% ----------------------------
% Pacchetti utili
% ----------------------------
\usepackage[utf8]{inputenc}
\usepackage[T1]{fontenc}
\usepackage[italian]{babel}
\usepackage{graphicx}
\usepackage{xcolor}
\usepackage{geometry}
\usepackage{setspace}
\usepackage{fancyhdr}
\usepackage{tikz}
\usepackage[colorlinks=true, linkcolor=blue, urlcolor=blue, citecolor=blue]{hyperref}
% ----------------------------
% Impostazioni pagina
% ----------------------------
\geometry{
    top=2cm,
    bottom=2cm,
    left=2cm,
    right=2cm
}

\setstretch{1.2}

% ----------------------------
% Dati personalizzabili
% ----------------------------
\newcommand{\Gruppo}{Atlas}
\newcommand{\Email}{\href{mailto:team9.atlas@gmail.com}{\textcolor{blue}{\underline{team9.atlas@gmail.com}}}}
\newcommand{\TitoloVerbale}{Verbale della Riunione}
\newcommand{\DataVerbale}{2026/01/12}
\newcommand{\OraInizio}{15:00}
\newcommand{\OraFine}{16:15}
\newcommand{\LuogoVerbale}{Chiamata Discord}
\newcommand{\LogoGruppo}{../../../Assets/AtlasLogo.png} % Inserisci il file del logo
\newcommand{\AbstractVerbale}{%
In questo verbale vengono riportati i principali argomenti discussi, le decisioni prese e le attività pianificate durante la riunione del gruppo nel giorno \DataVerbale \space dalle \OraInizio \space alle \OraFine.
}

% --- Nuove variabili aggiunte ---
\newcommand{\VersioneVerbale}{v1.0.0} % <-- modifica qui la versione o ID
\newcommand{\TipoVerbale}{Interno} 

\pagestyle{fancy}
\fancyhf{}
\fancyhead[L]{\Gruppo}
\fancyhead[R]{Verbale: \TipoVerbale \space - \space \DataVerbale}
\fancyfoot[C]{\thepage}

% ----------------------------
% Inizio documento
% ----------------------------
\begin{document}

% ----------------------------
% Prima pagina
% ----------------------------
\begin{titlepage}
    \centering

    % Logo
    \vspace*{0cm}
    %\includegraphics[width=10cm]{\LogoGruppo}\\[.5cm]
    \begin{tikzpicture}
        \clip (0,-0.1) circle (4.6cm); % raggio = metà della larghezza desiderata
        \node at (0,0) {\includegraphics[width=10cm]{\LogoGruppo}};
    \end{tikzpicture}\\
    [.5cm]
    % Titolo
    {\Huge \textbf{\TitoloVerbale}}\\[0.8cm]
    {\LARGE \Gruppo}\\[0.1cm]
    {\Email}\\[1.2cm]

    % Dati riunione
    \begin{tabular}{rl}
        \textbf{Data:} & \DataVerbale \\
        \textbf{Luogo:} & \LuogoVerbale \\
        \textbf{Versione:} & \VersioneVerbale \\
        \textbf{Tipo:} & \TipoVerbale \\
    \end{tabular}

    \vspace{1.2cm}

    % Componenti e ruoli
    {\large \textbf{Partecipanti}}\\[0.5cm]
    \begin{tabular}{l|l|l}
        \textbf{Nome} & \textbf{Presenza} & \textbf{Ruolo}\\
        \hline
        Andrea Difino & SI & Programmatore\\
        Federico Simonetto & SI & Analista\\
        Riccardo Valerio & SI & Amministratore\\
        Francesco Marcolongo & SI & Verificatore\\
        Michele Tesser & SI & Analista\\
        Giacomo Giora & SI & Responsabile\\
        Bilal Sabic & SI & Programmatore\\
    \end{tabular}

\end{titlepage}

\section*{Registro delle modifiche}{
    \begin{center}
        \begin{tabular}{|l|l|l|l|l|}
        \hline
            \textbf{Versione} & \textbf{Data} & \textbf{Autore} & \textbf{Verificatore} & \textbf{Descrizione} \\
            \hline
            \VersioneVerbale & \DataVerbale & Michele Tesser & Francesco Marcolongo & Correzioni varie \\
            \hline
            v0.1.0 & \DataVerbale & Federico Simonetto & Francesco Marcolongo & Prima stesura \\
            \hline
        \end{tabular}
    \end{center}
}

\newpage

\tableofcontents

\newpage
% ----------------------------
% Inizio contenuto verbale
% ----------------------------
\section{Abstract}{
    % Abstract
    \begin{minipage}{0.9\textwidth}
        \small
        \AbstractVerbale
    \end{minipage}
}


\section{Ordine del giorno}{
    \begin{itemize}
        \item Aggiornamento sullo stato dei documenti
        \item Discussione sulle metriche aggiornate allo sprint precedente
    \end{itemize}
}
\

\section{Discussione}{
    \subsection{Aggiornamento sullo stato dei documenti} {
    I membri del team condividono lo stato di avanzamento dei documenti di cui sono responsabili, aggiornando il gruppo sul lavoro già svolto. Viene inoltre pianificata l'attività da svolgere per il prosieguo dello sprint attuale.
	}
    \subsection{Discussione sulle metriche aggiornate allo sprint precedente} {
    I membri del team si allineano sui valori delle metriche visualizzati nel cruscotto e viene deciso di semplificare i documenti già scritti al fine di aumentare l'indice di Gulpease.
	}
    
	
}

\section{Decisioni prese}{
    \begin{center}
    \begin{tabular}{|c|p{12cm}|}
        \hline
        \textbf{ID} & \textbf{Decisione} \\
        \hline
            D1-\DataVerbale\_vi & Effettuare una revisione del documento "Norme di Progetto" al fine di renderlo più leggibile e meno tecnico \\
        \hline
    \end{tabular}
    \end{center}
}

\section{Attività da svolgere}{
    \begin{center}
    \begin{tabular}{|c|p{4.5cm}|c|p{3cm}|} 
        \hline
        \textbf{ID} & \textbf{Descrizione} & \textbf{Id Github Issue} & \textbf{Assegnatario}\\
        \hline
            A1-\DataVerbale\_vi & Continuare la stesura dei PdQ e dell'AdR & - & Team \\
        \hline
    \end{tabular}
    \end{center}
}


\end{document}