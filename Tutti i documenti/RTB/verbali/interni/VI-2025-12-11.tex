\documentclass[a4paper,12pt]{article}

% ----------------------------
% Pacchetti utili
% ----------------------------
\usepackage[utf8]{inputenc}
\usepackage[T1]{fontenc}
\usepackage[italian]{babel}
\usepackage{graphicx}
\usepackage{xcolor}
\usepackage{geometry}
\usepackage{setspace}
\usepackage{fancyhdr}
\usepackage{tikz}
\usepackage[colorlinks=true, linkcolor=blue, urlcolor=blue, citecolor=blue]{hyperref}
% ----------------------------
% Impostazioni pagina
% ----------------------------
\geometry{
    top=2cm,
    bottom=2cm,
    left=2cm,
    right=2cm
}

\setstretch{1.2}

% ----------------------------
% Dati personalizzabili
% ----------------------------
\newcommand{\Gruppo}{Atlas}
\newcommand{\Email}{\href{mailto:team9.atlas@gmail.com}{\textcolor{blue}{\underline{team9.atlas@gmail.com}}}}
\newcommand{\TitoloVerbale}{Verbale della Riunione}
\newcommand{\DataVerbale}{2025/12/12}
\newcommand{\OraInizio}{15:00}
\newcommand{\OraFine}{17:00}
\newcommand{\LuogoVerbale}{Chiamata Discord}
\newcommand{\LogoGruppo}{../../../Assets/AtlasLogo.png} % Inserisci il file del logo
\newcommand{\AbstractVerbale}{%
In questo verbale vengono riportati i principali argomenti discussi, le decisioni prese e le attività pianificate durante la riunione del gruppo nel giorno \DataVerbale \space dalle \OraInizio \space alle \OraFine.
}

% --- Nuove variabili aggiunte ---
\newcommand{\VersioneVerbale}{v1.0.0} % <-- modifica qui la versione o ID
\newcommand{\TipoVerbale}{Interno} 

\pagestyle{fancy}
\fancyhf{}
\fancyhead[L]{\Gruppo}
\fancyhead[R]{Verbale: \TipoVerbale \space - \space \DataVerbale}
\fancyfoot[C]{\thepage}

% ----------------------------
% Inizio documento
% ----------------------------
\begin{document}

% ----------------------------
% Prima pagina
% ----------------------------
\begin{titlepage}
    \centering

    % Logo
    \vspace*{0cm}
    %\includegraphics[width=10cm]{\LogoGruppo}\\[.5cm]
    \begin{tikzpicture}
        \clip (0,-0.1) circle (4.6cm); % raggio = metà della larghezza desiderata
        \node at (0,0) {\includegraphics[width=10cm]{\LogoGruppo}};
    \end{tikzpicture}\\
    [.5cm]
    % Titolo
    {\Huge \textbf{\TitoloVerbale}}\\[0.8cm]
    {\LARGE \Gruppo}\\[0.1cm]
    {\Email}\\[1.2cm]

    % Dati riunione
    \begin{tabular}{rl}
        \textbf{Data:} & \DataVerbale \\
        \textbf{Luogo:} & \LuogoVerbale \\
        \textbf{Versione:} & \VersioneVerbale \\
        \textbf{Tipo:} & \TipoVerbale \\
    \end{tabular}

    \vspace{1.2cm}

    % Componenti e ruoli
    {\large \textbf{Partecipanti}}\\[0.5cm]
    \begin{tabular}{l|l|l}
        \textbf{Nome} & \textbf{Presenza} & \textbf{Ruolo}\\
        \hline
        Andrea Difino & SI & Analista\\
        Federico Simonetto & SI & Responsabile\\
        Riccardo Valerio & SI & Verificatore\\
        Francesco Marcolongo & SI & Amministratore\\
        Michele Tesser & NO & Progettista\\
        Giacomo Giora & SI & Programmatore\\
        Bilal Sabic & SI & Verificatore\\
    \end{tabular}

\end{titlepage}

\section*{Registro delle modifiche}{
    \begin{center} 
        \begin{tabular}{|l|l|l|l|l|}
        \hline
            \textbf{Versione} & \textbf{Data} & \textbf{Autore} & \textbf{Verificatore} & \textbf{Descrizione} \\
            \hline
            \VersioneVerbale & \DataVerbale & Federico Simonetto & Bilal Sabic & Sistemazione contenuto \\
            \hline
            v0.1.0 & 2025/12/11 & Team &  & Prima stesura \\
            \hline
        \end{tabular}
    \end{center}
}

\newpage

\tableofcontents

\newpage
% ----------------------------
% Inizio contenuto verbale
% ----------------------------
\section{Abstract}{
    % Abstract
    \begin{minipage}{0.9\textwidth}
        \small
        \AbstractVerbale
    \end{minipage}
}


\section{Ordine del giorno}{
    \begin{itemize}
        \item Chiarimenti relativi ai documenti che il gruppo sta stilando
        \item Cambiamenti al glossario
        \item Chiarimenti relativi al modello grafico
        \item Domande da porre all'azienda proponente
        \item Ridistribuzione delle ore per sprint
    \end{itemize}
}
\

\section{Discussione}{
 	\subsection{Chiarimenti relativi ai documenti che il gruppo sta stilando} {
        Il team ha richiesto chiarimenti su alcune sezioni dell'\textit{Analisi dei requisiti}. Gli analisti di progetto, responsabili della stesura del documento, hanno fornito le spiegazioni necessarie.
	}
    \subsection{Cambiamenti al glossario} {
        Il team ha deciso di modificare il formato delle parole nei documenti che fanno riferimento a voci del glossario. È stata inoltre stabilita l'eliminazione di alcune voci precedentemente inserite, ritenute non necessarie.
	}
    \subsection{Chiarimenti relativi al modello grafico} {
        Dopo la realizzazione di un primo modello grafico dell'applicazione su Figma, utile per individuare i casi d'uso, gli autori hanno chiarito i dubbi sollevati dagli altri membri del team. Sono state inoltre annotate indicazioni relative alla gestione della persistenza dei dati, da implementare nel prodotto software.
	}
    \subsection{Domande da porre all'azienda proponente} {
        Al termine della riunione, il team ha annotato ulteriori dubbi relativi al progetto da sottoporre all'azienda proponente. Considerata la brevità delle domande, si è deciso di inviarle in modalità asincrona tramite Telegram.
	}
    \subsection{Ridistribuzione delle ore per sprint}{
        A seguito del ridotto utilizzo di alcuni ruoli e dell'elevato impiego di altri, le ore assegnate ai vari ruoli vengono modificate per il resto dello sprint corrente e per quelli successivi.
    }
	
}

\section{Decisioni prese}{
    \begin{center}
    \begin{tabular}{|c|p{12cm}|}
        \hline
        \textbf{ID} & \textbf{Decisione} \\
        \hline
            D1-\DataVerbale\_vi & Modificata la distribuzione delle ore per gli sprint seguenti \\
        \hline
            D2-\DataVerbale\_vi & Modificato il formato delle parole che fanno riferimento al glossario \\
        \hline
    \end{tabular}
    \end{center}
}

\section{Attività da svolgere}{
    \begin{center}
    \begin{tabular}{|c|p{4.5cm}|c|p{3cm}|} 
        \hline
        \textbf{ID} & \textbf{Descrizione} & \textbf{Id Github Issue} & \textbf{Assegnatario}\\
        \hline
            A1-\DataVerbale\_vi & Scrivere all'azienda proponente domande su Telegram  & - & Team \\
        \hline
            A2-\DataVerbale\_vi & Proseguire la stesura dei documenti & - & Team \\
        \hline
    \end{tabular}
    \end{center}
}


\end{document}