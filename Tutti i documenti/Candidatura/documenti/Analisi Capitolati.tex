\documentclass[a4paper,12pt]{article}

% ----------------------------
% Pacchetti utili
% ----------------------------
\usepackage[utf8]{inputenc}
\usepackage[T1]{fontenc}
\usepackage[italian]{babel}
\usepackage{graphicx}
\usepackage{xcolor}
\usepackage{geometry}
\usepackage{setspace}
\usepackage{fancyhdr}
\usepackage{tikz}
\usepackage[colorlinks=true, linkcolor=blue, urlcolor=blue, citecolor=blue]{hyperref}
% ----------------------------
% Impostazioni pagina
% ----------------------------
\geometry{
    top=2cm,
    bottom=2cm,
    left=2cm,
    right=2cm
}

\setstretch{1.2}

% ----------------------------
% Dati personalizzabili
% ----------------------------
\newcommand{\Gruppo}{Atlas}
\newcommand{\Email}{\href{mailto:team9.atlas@gmail.com}{\textcolor{blue}{\underline{team9.atlas@gmail.com}}}}
\newcommand{\Titolo}{Analisi Capitolati}
\newcommand{\DataModifica}{2025/10/29}
\newcommand{\LogoGruppo}{img/AtlasLogo.png} % Inserisci il file del logo

% --- Nuove variabili aggiunte ---
\newcommand{\VersioneVerbale}{v1.0} % <-- modifica qui la versione o ID
\newcommand{\Interno}{Interno} 

\pagestyle{fancy}
\fancyhf{}
\fancyhead[L]{\Gruppo}
\fancyhead[R]{Documento: \Interno \space - \space \DataModifica}
\fancyfoot[C]{\thepage}

% ----------------------------
% Inizio documento
% ----------------------------
\begin{document}

% ----------------------------
% Prima pagina
% ----------------------------
\begin{titlepage}
    \centering

    % Logo in alto
    \vspace*{0cm}
    \begin{tikzpicture}
        \clip (0,-0.1) circle (4.6cm); % raggio = metà della larghezza desiderata
        \node at (0,0) {\includegraphics[width=10cm]{\LogoGruppo}};
    \end{tikzpicture}\\[0.8cm]

    % Nome gruppo ed email sotto al logo
    {\LARGE \textbf{\Gruppo}}\\[0.1cm]
    {\large \Email}\\[1.2cm]

    % Sottotitolo del progetto
    {\Large \textit{Progetto di ingegneria del software A.A. 2025/2026}}\\[1.5cm]

    % Titolo principale
    {\Huge \textbf{\Titolo}}\\[1.5cm]

    % Dati riunione
    \begin{tabular}{rl}
        \textbf{Data:} & \DataModifica \\
        \textbf{Versione:} & \VersioneVerbale \\
        \textbf{Tipo:} & \Interno \\
    \end{tabular}

\end{titlepage}


\section*{Tabella delle revisioni}{
    \begin{center} 
        \begin{tabular}{|l|l|l|l|l|}
            \hline
            \textbf{Versione} & \textbf{Data} & \textbf{Autore} & \textbf{Verificatore} & \textbf{Descrizione} \\
            \hline
            v1.0& 2025/10/29 & Riccardo Valerio & & Modifiche sulle analisi \\
            \hline
            v0.1 & 2025/10/28 & Andrea Difino & Team & Prima Stesura \\
            && Federico Simonetto &&\\
            \hline
            
        \end{tabular}
    \end{center}
}

\newpage

\tableofcontents

\newpage

\section{Capitolato Scelto: C1 - Automated EN18031 Compliance Verification}
    Proponente: \textbf{Bluewind S.r.l.}

    \subsection{Descrizione Capitolato}{
        Il capitolato proposto da Bluewind S.r.l. ha come obiettivo la realizzazione di un'applicazione, web o desktop, che supporti le aziende nel processo di verifica di conformità ai requisiti di sicurezza informatica previsti dalla direttiva \textbf{RED} (Radio Equipment Directive).
        Tale verifica viene condotta tramite un sistema basato su \textbf{alberi decisionali}, nei quali l'utente è guidato attraverso una sequenza di domande e condizioni per determinare il livello di aderenza ai requisiti della normativa \textbf{EN18031}.
        L'applicazione dovrà fornire un'interfaccia chiara e intuitiva per compilare e navigare le domande, permettere l'esportazione dei risultati e garantire la tracciabilità delle risposte.
        L’obiettivo finale è rendere il processo di valutazione della conformità più accessibile, ripetibile e automatizzato, riducendo errori umani e tempi di verifica.
    

    \subsection{Obiettivo}
        Il progetto mira a realizzare un software capace di:
        \begin{itemize}
            \item Gestire la logica degli alberi decisionali definiti dallo standard EN18031 e dai requisiti della direttiva RED;
            \item Guidare l’utente nella compilazione delle domande attraverso un’interfaccia interattiva e user-friendly;
            \item Permettere il salvataggio, l’aggiornamento e la consultazione delle sessioni di verifica completate;
            \item Generare \textbf{report} riepilogativi che sintetizzino i risultati ottenuti e i punti di non conformità;
            \item Consentire la manutenzione e l’aggiornamento dei contenuti in modo modulare, così da recepire facilmente future revisioni della normativa.
        \end{itemize}
    

    \subsection{Tecnologie richieste}
        Le tecnologie consigliate dal proponente includono:
        \begin{itemize}
            \item \textbf{Python 3} per la realizzazione del backend, nel caso in cui venga sviluppata un’applicazione web;
            \item Librerie o framework per la creazione dell’interfaccia grafica, a discrezione del gruppo;
            \item Possibilità di integrazione con strumenti di \textbf{data export} per generare i report finali in formati standard.
        \end{itemize}
        Il proponente lascia piena libertà nella scelta degli strumenti e dei framework, purché siano adeguatamente documentati e rispettino buone pratiche di progettazione software.
    

    \subsection{Motivazione Scelta}
        La decisione di selezionare il capitolato proposto da Bluewind è stata motivata da diversi fattori:
        \begin{itemize}
            \item L’interesse del gruppo verso tematiche di \textbf{sicurezza informatica} e \textbf{verifica di conformità}, ambiti molto attuali e con forte valore industriale;
            \item L’approccio metodologico basato su \textbf{alberi decisionali}, che unisce aspetti di logica, modellazione e automazione;
            \item La \textbf{disponibilità e collaborazione} dimostrata dall’azienda, che si è resa parte attiva sin dalle fasi iniziali offrendo supporto tecnico e organizzativo tramite incontri periodici;
            \item La possibilità di lavorare su un progetto con un chiaro risvolto pratico, ma al contempo flessibile, che permette di sperimentare scelte architetturali e tecnologiche in autonomia;
            \item La presenza di un \textbf{caso di studio reale} fornito da Bluewind, utile per comprendere concretamente gli standard e la struttura degli alberi decisionali.
        \end{itemize}
    

    \subsection{Conclusioni}
        Il capitolato \textbf{C1 - Automated EN18031 Compliance Verification} è stato ritenuto dal gruppo Atlas il più equilibrato tra complessità tecnica e valore applicativo. 
        La possibilità di collaborare con un’azienda competente nel settore embedded e della sicurezza, unita alla libertà di scelta tecnologica, ha reso questa proposta ideale per raggiungere gli obiettivi formativi del progetto didattico.
        La combinazione tra contenuti tecnici di rilievo, autonomia progettuale e disponibilità del proponente ha portato alla \textbf{scelta definitiva del C1} come progetto finale del gruppo.
    


\newpage

\section{C2 - Code Guardian}
    Proponente: \textbf{Var Group S.p.A.}

    \subsection{Descrizione Capitolato}{
        Il capitolato proposto da Var Group ha come obiettivo la progettazione e lo sviluppo di una piattaforma in grado di effettuare \textbf{analisi automatizzate di repository GitHub} per individuare eventuali \textbf{lacune di sicurezza}, \textbf{debiti tecnici} o violazioni di \textbf{best practice} di sviluppo.
        L’applicazione deve operare come un sistema di \textbf{agenti software cooperanti}, ognuno dei quali specializzato in una fase di analisi (test, sicurezza, documentazione, qualità del codice).
        Tali agenti comunicano e si coordinano per fornire un \textbf{report complessivo} sullo stato della repository analizzata, segnalando i punti critici e proponendo raccomandazioni per il miglioramento della qualità e della sicurezza del codice.
        Il progetto si inserisce nel contesto dell’\textbf{Application Security} e della \textbf{Code Quality Automation}, temi di crescente interesse nel settore dello sviluppo software professionale.
    

    \subsection{Obiettivo}
        L’obiettivo principale del progetto è realizzare una piattaforma web che, attraverso un’architettura modulare ad agenti, permetta di:
        \begin{itemize}
            \item Analizzare automaticamente le repository GitHub per valutarne la \textbf{qualità}, la \textbf{manutenibilità} e la \textbf{sicurezza};
            \item Fornire \textbf{report dettagliati} contenenti risultati delle analisi, vulnerabilità rilevate e suggerimenti per la loro risoluzione;
            \item Identificare carenze in termini di \textbf{test automatici}, copertura, gestione delle dipendenze e conformità a standard di codifica;
            \item Applicare un sistema di \textbf{remediation} che suggerisca miglioramenti secondo le principali linee guida OWASP e le best practice di sviluppo sicuro;
            \item Integrare l’analisi nei flussi di sviluppo tramite \textbf{GitHub Actions}, favorendo la \textbf{Continuous Integration} e la \textbf{Continuous Delivery}.
        \end{itemize}
    

    \subsection{Tecnologie richieste}
        Le tecnologie consigliate nel capitolato comprendono:
        \begin{itemize}
            \item \textbf{Node.js} e \textbf{Python} per la realizzazione del backend e dell’orchestratore degli agenti;
            \item \textbf{React.js} per lo sviluppo del frontend web;
            \item \textbf{MongoDB} o \textbf{PostgreSQL} come database per la gestione dei dati di analisi e dei risultati;
            \item \textbf{GitHub Actions} per l’integrazione della pipeline CI/CD e l’automatizzazione dei test;
            \item \textbf{AWS} come infrastruttura cloud per il deployment e l’orchestrazione dei servizi.
        \end{itemize}
        L’architettura è concepita per essere \textbf{scalabile}, \textbf{estendibile} e facilmente integrabile con strumenti di sviluppo già esistenti.
    

    \subsection{Pro e Contro}
        \begin{center}
            \begin{tabular}{|p{9cm}|p{5cm}|}
                \hline
                \textbf{Pro} & \textbf{Contro} \\
                \hline
                Progetto innovativo e attuale, incentrato su \textbf{analisi automatica del codice}, \textbf{sicurezza} e \textbf{miglioramento continuo della qualità del software}. & Utilizzo di componenti cloud su \textbf{AWS} inizialmente poco familiari al gruppo. \\
                \hline
                Buon livello di supporto e disponibilità da parte dell’azienda proponente. & Necessità di gestire un’architettura a microservizi distribuita. \\
                \hline
            \end{tabular}
        \end{center}
    

    \subsection{Conclusioni}
        Il capitolato \textbf{C2 - Code Guardian} è stato valutato positivamente per la sua struttura tecnica chiara, la rilevanza del tema e l’elevata applicabilità nel contesto aziendale moderno.
        Tuttavia, la complessità legata all’orchestrazione degli agenti e all’integrazione con infrastrutture cloud avanzate ha portato il gruppo a considerare questo progetto meno adatto come prima scelta.
        Il gruppo ha pertanto deciso di orientarsi verso il \textbf{capitolato C1}, giudicato più coerente con le competenze di partenza e con la disponibilità temporale prevista.
    

\newpage

\section{C3 - DIPReader}
    Proponente: \textbf{Sanmarco Informatica S.p.A.}

    \subsection{Descrizione Capitolato}
        Il capitolato proposto da Sanmarco Informatica prevede la realizzazione di \textbf{DIPReader}, un’applicazione in grado di permettere la \textbf{consultazione} e la \textbf{ricerca} dei \textbf{Distribution Information Package (DIP)} provenienti da sistemi di conservazione documentale.
        L’obiettivo è fornire uno strumento che consenta agli utenti di esplorare in modo intuitivo e immediato i pacchetti di archiviazione prodotti, anche \textbf{offline}, rendendo possibile la ricerca testuale, la visualizzazione dei documenti e la selezione di sottoinsiemi da esportare.
        Il progetto si colloca nel contesto della \textbf{gestione documentale aziendale} e della \textbf{conservazione digitale}, ambiti fondamentali per la corretta amministrazione e tracciabilità dei dati.
    

    \subsection{Obiettivo}
        Il progetto ha come scopo la realizzazione di un prodotto software che permetta di:
        \begin{itemize}
            \item Visualizzare in modo strutturato le informazioni contenute nei pacchetti di archiviazione (DIP);
            \item Effettuare ricerche e filtraggi efficienti tra i documenti esportati, anche in assenza di connessione;
            \item Mostrare in anteprima i formati di file più comuni (come PDF, immagini o XML);
            \item Selezionare e salvare un sottoinsieme dei documenti sul dispositivo locale;
            \item Garantire un’interfaccia utente chiara, navigabile e coerente con i principi di accessibilità e usabilità.
        \end{itemize}
        L’applicazione dovrà dunque migliorare la \textbf{fruibilità dei dati conservati}, fornendo strumenti moderni e performanti per la consultazione.
    

    \subsection{Tecnologie richieste}
        Le tecnologie indicate dal proponente comprendono:
        \begin{itemize}
            \item \textbf{SQLite} come database relazionale per la memorizzazione locale dei metadati e dei riferimenti ai file;
            \item \textbf{FAISS} come libreria per l’indicizzazione e la ricerca efficiente su grandi quantità di dati testuali;
            \item Framework per il frontend quali \textbf{React} o \textbf{Angular} per la realizzazione di un’interfaccia interattiva e moderna;
            \item Eventuale integrazione di strumenti per la \textbf{visualizzazione dei documenti} e la \textbf{stampa} diretta dei risultati di ricerca.
        \end{itemize}
        L’architettura complessiva deve essere progettata per operare anche in modalità \textbf{standalone}, senza dipendenze da servizi esterni.
    

    \subsection{Pro e Contro}
        \begin{center}
            \begin{tabular}{|p{9cm}|p{5cm}|}
                \hline
                \textbf{Pro} & \textbf{Contro} \\
                \hline
                Progetto concreto e ben definito, con applicazione diretta nel campo della \textbf{gestione e consultazione documentale}. & Tecnologie (es. FAISS) poco note al gruppo e con curva di apprendimento significativa. \\
                \hline
                Permette di approfondire tematiche di \textbf{indicizzazione}, \textbf{ricerca semantica} e \textbf{usabilità}. & Il dominio applicativo, legato alla conservazione digitale, ha suscitato minor interesse rispetto ad altri ambiti. \\
                \hline
            \end{tabular}
        \end{center}
    

    \subsection{Conclusioni}
        Il capitolato \textbf{C3 - DIPReader} rappresenta un progetto ben strutturato e tecnicamente solido, con un chiaro valore applicativo nel settore della gestione documentale.
        Tuttavia, la natura specifica del dominio e la necessità di acquisire familiarità con tecnologie nuove per il gruppo hanno portato a preferire altri capitolati più affini alle competenze e agli interessi del team.
        Il gruppo ha pertanto deciso di non selezionare questo progetto come principale, pur riconoscendone la qualità e la rilevanza industriale.
    


\newpage

\section{C4 - L'app che Protegge e Trasforma}
    Proponente: \textbf{Miriade S.r.l.}

    \subsection{Descrizione Capitolato}
        Il capitolato proposto da Miriade S.r.l. ha come obiettivo la progettazione e lo sviluppo di un’applicazione mobile denominata \textbf{“L’app che Protegge e Trasforma”}, pensata per la \textbf{prevenzione} e il \textbf{supporto} alle vittime di violenza di genere.
        L’app mira a fornire uno strumento digitale sicuro, accessibile e intuitivo che consenta all’utente di ricevere aiuto in situazioni di emergenza, accedere a informazioni e servizi di assistenza, e promuovere la sensibilizzazione sul tema della violenza di genere.
        Il progetto intende unire aspetti tecnologici e sociali, ponendo particolare attenzione alla \textbf{privacy}, alla \textbf{protezione dei dati personali} e alla \textbf{usabilità} in condizioni critiche.
    

    \subsection{Obiettivo}
        L’obiettivo principale è realizzare un’applicazione per dispositivi \textbf{Android} e \textbf{iOS} che includa:
        \begin{itemize}
            \item Funzionalità di \textbf{emergenza immediata}, per consentire all’utente di inviare richieste di aiuto in modo rapido e discreto;
            \item Accesso a \textbf{contenuti informativi} e strumenti di prevenzione e sensibilizzazione;
            \item Canali sicuri di comunicazione con strutture o enti di supporto;
            \item Meccanismi di autenticazione e protezione dei dati sensibili;
            \item Una struttura modulare che consenta la futura integrazione di nuove funzionalità.
        \end{itemize}
        L’applicazione deve risultare semplice da utilizzare, affidabile e orientata all’accessibilità, in modo da garantire un’esperienza d’uso positiva anche in situazioni di difficoltà.
    

    \subsection{Tecnologie richieste}
        Le tecnologie e gli strumenti indicati nel capitolato comprendono:
        \begin{itemize}
            \item \textbf{Flutter} (o framework mobile equivalente) per lo sviluppo multipiattaforma;
            \item \textbf{AWS} per la gestione del backend e l’hosting dei servizi cloud (non obbligatorio ma consigliato dal proponente);
            \item \textbf{Amazon API Gateway} per la comunicazione tra frontend e backend;
            \item \textbf{PostgreSQL} per la gestione di dati strutturati e \textbf{Amazon DynamoDB} per quelli non strutturati;
            \item Meccanismi di \textbf{crittografia} e \textbf{autenticazione sicura} per la tutela dei dati personali.
        \end{itemize}
        Il sistema dovrà inoltre prevedere politiche di sicurezza e conformità al GDPR per la gestione delle informazioni sensibili.
    

    \subsection{Pro e Contro}
        \begin{center}
            \begin{tabular}{|p{9cm}|p{5cm}|}
                \hline
                \textbf{Pro} & \textbf{Contro} \\
                \hline
                Progetto dal forte impatto sociale e umano, che coniuga tecnologia e responsabilità etica. & Alcune tecnologie richieste, come AWS e DynamoDB, non sono note al gruppo. \\
                \hline
                Opportunità di lavorare su un'applicazione reale con elevati requisiti di sicurezza e privacy. & La complessità architetturale e la necessità di garantire elevati standard di protezione aumentano la difficoltà del progetto. \\
                \hline
                & Interesse limitato da parte del gruppo verso il dominio applicativo specifico. \\
                \hline
            \end{tabular}
        \end{center}
    

    \subsection{Conclusioni}
        Il capitolato \textbf{C4 - L’app che Protegge e Trasforma} rappresenta un progetto tecnologicamente valido e con un importante valore sociale, che unisce sicurezza, accessibilità e innovazione.
        Tuttavia, il gruppo ha scelto di non selezionare questa proposta a causa della complessità tecnica e organizzativa del progetto, nonché della distanza tra il dominio applicativo e gli interessi principali del team.
    

\newpage

\section{C5 - NEXUM}
    Proponente: \textbf{Eggon S.r.l.}

    \subsection{Descrizione Capitolato}{
        Il capitolato proposto da Eggon S.r.l. riguarda lo sviluppo e l’evoluzione della piattaforma \textbf{NEXUM}, un ecosistema digitale avanzato per la gestione delle \textbf{risorse umane} (HR) e la collaborazione con gli \textbf{studi dei Consulenti del Lavoro}.
        Il progetto nasce con l’obiettivo di ottimizzare e digitalizzare i processi HR aziendali, favorendo una comunicazione fluida tra aziende e consulenti, e offrendo strumenti intelligenti per l’analisi e la gestione dei dati dei dipendenti.
        L’iniziativa punta a migliorare l’esperienza digitale del personale e la trasparenza delle attività amministrative, integrando funzionalità \textbf{AI-driven} e \textbf{data-driven} per l’automazione di flussi di lavoro e processi decisionali.
    

    \subsection{Obiettivo}
        L’obiettivo del progetto è realizzare una piattaforma digitale modulare, scalabile e intelligente che consenta di:
        \begin{itemize}
            \item Automatizzare la gestione dei processi HR, riducendo tempi e costi operativi;
            \item Migliorare l’efficienza nella comunicazione e nello scambio di documenti tra aziende e consulenti;
            \item Integrare dati provenienti da fonti eterogenee per abilitare analisi predittive e decisionali;
            \item Fornire strumenti basati su \textbf{Intelligenza Artificiale} per l’elaborazione, la classificazione e la validazione delle informazioni;
            \item Assicurare un’esperienza utente fluida tramite un’interfaccia moderna e personalizzabile.
        \end{itemize}
        L’approccio proposto mira a creare un sistema innovativo e sostenibile, in grado di evolversi nel tempo e adattarsi a differenti contesti aziendali.
    

    \subsection{Tecnologie richieste}
        Le tecnologie indicate dal proponente includono:
        \begin{itemize}
            \item \textbf{Angular} per la realizzazione della dashboard amministrativa e \textbf{Next.js} per l’interfaccia destinata agli utenti finali;
            \item \textbf{Ruby on Rails} per lo sviluppo del backend e l’implementazione delle logiche applicative;
            \item \textbf{PostgreSQL} come database relazionale per la memorizzazione e gestione dei dati;
            \item Integrazione con servizi e API esterne per la sincronizzazione dei dati HR e la generazione di analisi;
            \item Architettura progettata per essere \textbf{scalabile}, modulare e predisposta a futuri ampliamenti funzionali.
        \end{itemize}
        L’utilizzo combinato di questi strumenti garantisce un ecosistema coerente e orientato alla crescita continua del prodotto.
    

    \subsection{Pro e Contro}
        \begin{center}
            \begin{tabular}{|p{9cm}|p{5cm}|}
                \hline
                \textbf{Pro} & \textbf{Contro} \\
                \hline
                Forte orientamento verso l’\textbf{Intelligenza Artificiale} e l’automazione dei processi aziendali, temi di grande interesse per il gruppo. & Richiede la padronanza di numerose tecnologie e un’architettura complessa da gestire. \\
                \hline
                Progetto con chiara applicabilità nel mondo del lavoro e potenziale impatto reale sulla digitalizzazione HR. & Presenza di molte funzionalità obbligatorie che aumentano i tempi di sviluppo. \\
                \hline
                Architettura modulare e scalabile, favorevole a una crescita evolutiva del sistema. & Complessità organizzativa nella gestione e integrazione dei vari componenti software. \\
                \hline
            \end{tabular}
        \end{center}
    

    \subsection{Conclusioni}
        Il capitolato \textbf{C5 - NEXUM} è stato valutato come un progetto moderno, ambizioso e altamente innovativo, con un chiaro orientamento alla trasformazione digitale dei processi HR.
        Nonostante la qualità e la rilevanza del tema, la complessità tecnica e la necessità di gestire numerosi moduli e tecnologie hanno portato il gruppo a non selezionare questo capitolato come principale.
        Il gruppo riconosce comunque il valore formativo e professionale dell’iniziativa proposta da Eggon S.r.l., apprezzandone la visione tecnologica e la solidità architetturale.


\newpage

\section{C6 - Second Brain}
    Proponente: \textbf{Zucchetti S.p.A.}

    \subsection{Descrizione Capitolato}
        Il capitolato proposto da Zucchetti S.p.A. prevede la realizzazione di un’applicazione web innovativa, denominata \textbf{Second Brain}, progettata per esplorare e valutare le potenzialità dei \textbf{Large Language Models (LLM)} nel supportare gli utenti in attività legate alla creazione di contenuti testuali e alla generazione di idee.
        Il progetto si colloca nel contesto dell’\textbf{intelligenza artificiale generativa}, con l’obiettivo di costruire un sistema capace di fungere da assistente virtuale per la scrittura, la revisione di testi e la stimolazione della creatività.
        L’applicazione dovrà offrire un’interfaccia intuitiva e funzionale che consenta all’utente di interagire con il modello linguistico per migliorare il contenuto prodotto o ricevere suggerimenti contestuali.
    

    \subsection{Obiettivo}
        L’obiettivo del progetto è realizzare una piattaforma web composta da due aree principali:
        \begin{itemize}
            \item \textbf{Editor Markdown}: un ambiente di scrittura che permetta di redigere, formattare e modificare testi in modo semplice e immediato;
            \item \textbf{Area di risultato}: uno spazio in cui visualizzare i suggerimenti, le riformulazioni e le risposte generate dal modello linguistico.
        \end{itemize}
        L’applicazione deve integrare le capacità di uno o più \textbf{modelli di linguaggio (LLM)} accessibili tramite API, per fornire funzionalità avanzate di:
        \begin{itemize}
            \item correzione ortografica e grammaticale;
            \item miglioramento stilistico e semantico;
            \item generazione di nuove idee e varianti di testo;
            \item sintesi o espansione dei contenuti.
        \end{itemize}
        Il sistema dovrà consentire inoltre di gestire lo storico delle sessioni di scrittura e mantenere un livello di interattività adeguato all’esperienza di un assistente personale virtuale.
    

    \subsection{Tecnologie richieste}
        Le tecnologie suggerite dal proponente includono:
        \begin{itemize}
            \item \textbf{API} compatibili con modelli linguistici di tipo \textbf{LLM} (ad esempio OpenAI o equivalenti);
            \item Strumenti per la creazione dell’interfaccia web, scelti liberamente dal gruppo, purché documentati e mantenibili;
            \item Possibile integrazione con framework di backend leggeri per la gestione delle chiamate API e delle sessioni utente;
            \item Utilizzo di strumenti di versionamento del codice e testing automatico per garantire qualità e riproducibilità.
        \end{itemize}
        Il capitolato lascia libertà di scelta per lo stack tecnologico, ponendo maggiore attenzione all’esperienza utente e alla qualità dell’interazione con il modello linguistico.
    

    \subsection{Pro e Contro}
        \begin{center}
            \begin{tabular}{|p{9cm}|p{5cm}|}
                \hline
                \textbf{Pro} & \textbf{Contro} \\
                \hline
                Elevata libertà tecnologica e possibilità di sperimentare con diversi modelli di intelligenza artificiale. & Supporto tecnico da parte dell’azienda proponente limitato rispetto ad altri capitolati. \\
                \hline
                Tema moderno e stimolante, con ampio margine di creatività e ricerca. & Complessità nell’interazione con modelli linguistici e gestione delle API. \\
                \hline
                Opportunità di approfondire tecniche di \textbf{prompt engineering} e uso pratico di LLM. & Rischio di dipendenza da servizi esterni e costi associati alle API. \\
                \hline
            \end{tabular}
        \end{center}
    

    \subsection{Conclusioni}
        Il capitolato \textbf{C6 - Second Brain} rappresenta un progetto di grande interesse, incentrato sull’uso dell’intelligenza artificiale generativa e sulla sperimentazione di nuovi paradigmi di scrittura assistita.
        Tuttavia, la ridotta disponibilità di supporto diretto da parte del proponente e la necessità di acquisire competenze specifiche nella gestione dei modelli LLM hanno portato il gruppo a non selezionare questo capitolato come progetto principale.
        Il gruppo ha comunque valutato positivamente la proposta, riconoscendone il potenziale didattico e la pertinenza con le tematiche attuali dell’AI generativa.
    


\newpage

\section{C7 - Sistema di acquisizione dati da sensori}
    Proponente: \textbf{M31 S.r.l.}

    \subsection{Descrizione Capitolato}
        Il capitolato proposto da M31 S.r.l. ha come obiettivo la progettazione e lo sviluppo di un \textbf{sistema distribuito per l’acquisizione e la gestione di dati provenienti da sensori Bluetooth Low Energy (BLE)}.
        Il sistema deve essere in grado di raccogliere dati da dispositivi periferici, trasmetterli a nodi intermedi e inviarli successivamente a un’infrastruttura \textbf{cloud}, dove i dati vengono elaborati, bufferizzati e resi disponibili tramite \textbf{API}.
        L’architettura proposta si articola in tre livelli principali:
        \begin{itemize}
            \item \textbf{Livello sensori}: dispositivi periferici BLE che raccolgono i dati dal campo;
            \item \textbf{Livello gateway}: nodi intermedi che aggregano e formattano i dati ricevuti dai sensori, inoltrandoli al cloud;
            \item \textbf{Livello cloud}: responsabile della memorizzazione, elaborazione e fornitura dei dati tramite interfacce di accesso.
        \end{itemize}
        Il progetto rientra nel dominio dell’\textbf{Internet of Things (IoT)} e punta alla creazione di un’infrastruttura scalabile e sicura per la gestione di dati sensibili provenienti da sensori distribuiti.
    

    \subsection{Obiettivo}
        Gli obiettivi principali del progetto sono:
        \begin{itemize}
            \item Progettare un’infrastruttura \textbf{scalabile e sicura} per l’acquisizione, l’elaborazione e la gestione dei dati sensoriali;
            \item Garantire la \textbf{segregazione dei dati} tra diversi tenant e la protezione delle informazioni scambiate;
            \item Implementare meccanismi di \textbf{comunicazione efficiente} tra i vari livelli (sensori, gateway e cloud);
            \item Fornire strumenti di \textbf{monitoraggio e visualizzazione} dei dati per amministratori e utenti finali;
            \item Simulare il comportamento dei gateway per consentire attività di test e validazione del sistema.
        \end{itemize}
        Il sistema dovrà inoltre supportare tecniche di autenticazione e crittografia avanzate per garantire l’affidabilità e la sicurezza delle comunicazioni.
    

    \subsection{Tecnologie richieste}
        Le tecnologie e gli strumenti indicati nel capitolato comprendono:
        \begin{itemize}
            \item \textbf{Node.js} e \textbf{Nest.js} per lo sviluppo dei microservizi del backend;
            \item \textbf{Go} per componenti ad alte prestazioni, come moduli di sincronizzazione e comunicazione;
            \item \textbf{NATS} o \textbf{Apache Kafka} per la messaggistica e la comunicazione asincrona tra microservizi;
            \item \textbf{Google Cloud Platform} per l’orchestrazione e la gestione dell’infrastruttura distribuita;
            \item \textbf{PostgreSQL} e \textbf{MongoDB} per la persistenza dei dati strutturati e non strutturati;
            \item \textbf{Angular} per la realizzazione dell’interfaccia utente;
            \item Strumenti di sicurezza come \textbf{TLS}, \textbf{JWT}, \textbf{OAuth2} e \textbf{mTLS} per la protezione dei dati e delle comunicazioni.
        \end{itemize}
        Il sistema dovrà includere inoltre una dashboard per la visualizzazione dei dati e il monitoraggio in tempo reale delle attività del sistema.
    

    \subsection{Pro e Contro}
        \begin{center}
            \begin{tabular}{|p{9cm}|p{5cm}|}
                \hline
                \textbf{Pro} & \textbf{Contro} \\
                \hline
                Progetto tecnicamente avanzato e formativo, che consente di acquisire esperienza su \textbf{architetture distribuite}, \textbf{microservizi} e \textbf{IoT}. & Elevata complessità architetturale e numero significativo di tecnologie da apprendere. \\
                \hline
                Opportunità di approfondire la gestione della sicurezza, della comunicazione e della scalabilità dei sistemi cloud. & Rischio di un carico di lavoro elevato nella fase di integrazione e test. \\
                \hline
                Buon potenziale formativo e applicabilità industriale nel settore dei sistemi embedded e sensoristici. & Richiede competenze iniziali non ancora consolidate dal gruppo. \\
                \hline
            \end{tabular}
        \end{center}
    

    \subsection{Conclusioni}
        Il capitolato \textbf{C7 - Sistema di acquisizione dati da sensori} rappresenta un progetto completo e altamente formativo nel campo dell’Internet of Things.
        Nonostante l’interesse tecnico suscitato e la rilevanza del tema, la complessità complessiva e la necessità di gestire molteplici componenti e tecnologie hanno portato il gruppo a non selezionare questo progetto.
        Il gruppo riconosce comunque l’alto valore didattico e professionale della proposta, apprezzandone la qualità architetturale e la chiarezza con cui è stato presentato dal proponente M31 S.r.l.
    



\newpage

\section{C8 - Smart Order}
    Proponente: \textbf{Ergon S.r.l.}

    \subsection{Descrizione Capitolato}
        Il capitolato proposto da Ergon S.r.l. ha come obiettivo la realizzazione di un sistema intelligente denominato \textbf{Smart Order}, in grado di automatizzare il processo di ricezione e interpretazione degli \textbf{ordini cliente}.
        Il sistema dovrà essere capace di analizzare \textbf{input multimodali} testi, immagini o registrazioni vocali e convertirli in \textbf{ordini strutturati} pronti per l’inserimento nei sistemi informativi aziendali, come gli \textbf{ERP}.
        Il progetto si inserisce nel dominio dell’\textbf{automazione dei processi aziendali} e dell’\textbf{intelligenza artificiale} applicata alla comprensione del linguaggio naturale, con l’obiettivo di ridurre gli errori umani e velocizzare le procedure di gestione degli ordini.
    

    \subsection{Obiettivo}
        L’obiettivo principale del progetto è creare una piattaforma che consenta di:
        \begin{itemize}
            \item Analizzare ordini cliente non strutturati provenienti da diverse fonti (e-mail, immagini, audio);
            \item Estrarre automaticamente i dati rilevanti, come codici articolo, quantità e informazioni logistiche;
            \item Effettuare la validazione e normalizzazione dei dati prima dell’inserimento nel sistema ERP;
            \item Gestire una base di conoscenza per migliorare progressivamente la precisione del riconoscimento;
            \item Offrire un’interfaccia web che permetta di visualizzare, verificare e confermare gli ordini generati.
        \end{itemize}
        Il sistema dovrà garantire accuratezza, affidabilità e un alto grado di automazione, riducendo al minimo l’intervento manuale.
    

    \subsection{Tecnologie richieste}
        Le tecnologie indicate nel capitolato comprendono:
        \begin{itemize}
            \item \textbf{BERT} e \textbf{GPT} per la comprensione del linguaggio naturale e il riconoscimento del contesto semantico;
            \item \textbf{Tesseract OCR} per l’estrazione di testo da immagini e documenti;
            \item \textbf{Whisper (OpenAI)} per la conversione di contenuti audio in testo (speech-to-text);
            \item Framework web come \textbf{.NET Blazor}, \textbf{React.js} o \textbf{Angular} per la realizzazione dell’interfaccia utente;
            \item Database relazionali a scelta per la memorizzazione degli ordini e dei metadati associati;
            \item Integrazione con \textbf{API ERP} per l’inserimento automatico dei dati strutturati.
        \end{itemize}
        Il sistema dovrà inoltre essere progettato per garantire la \textbf{scalabilità} e la \textbf{modularità}, così da poter integrare facilmente nuovi modelli e componenti in futuro.
    

    \subsection{Pro e Contro}
        \begin{center}
            \begin{tabular}{|p{9cm}|p{5cm}|}
                \hline
                \textbf{Pro} & \textbf{Contro} \\
                \hline
                Progetto fortemente innovativo che combina \textbf{AI}, \textbf{visione artificiale} e \textbf{NLP}, con applicazioni concrete nel mondo aziendale. & Elevata complessità architetturale dovuta all’integrazione di moduli eterogenei (AI, database, frontend, API ERP). \\
                \hline
                Possibilità di acquisire competenze avanzate su modelli linguistici e framework di machine learning. & Richiede un ampio lavoro di addestramento e validazione dei modelli per garantire risultati accurati. \\
                \hline
                Il proponente fornisce un set di dati reali per test e validazione, favorendo lo sviluppo del PoC. & Necessità di gestire la coerenza tra formati e sorgenti di dati differenti. \\
                \hline
            \end{tabular}
        \end{center}
    

    \subsection{Conclusioni}
        Il capitolato \textbf{C8 - Smart Order} è stato valutato positivamente per l’originalità e il potenziale di innovazione, grazie all’integrazione di tecniche avanzate di intelligenza artificiale e automazione.
        Tuttavia, la complessità derivante dalla gestione di input multimodali e dall’interconnessione con sistemi ERP ha portato il gruppo a considerare il progetto troppo impegnativo rispetto alle risorse disponibili.
    

\newpage

\section{C9 - View4Life}
    Proponente: \textbf{Vimar S.p.A.}

    \subsection{Descrizione Capitolato}
        Il capitolato proposto da Vimar S.p.A. riguarda lo sviluppo di una piattaforma denominata \textbf{View4Life}, progettata per la \textbf{gestione e il monitoraggio di impianti smart} installati all’interno di residenze protette.
        Il sistema dovrà consentire la supervisione e il controllo remoto di dispositivi intelligenti come attuatori, sensori, termostati e comandi per tapparelle, al fine di migliorare la sicurezza e il comfort degli utenti finali.
        L’applicazione dovrà offrire un’interfaccia web responsive, destinata principalmente al personale sanitario e agli operatori delle strutture, per consentire la gestione efficiente degli ambienti e il monitoraggio in tempo reale dello stato dei dispositivi.
        Il progetto si inserisce nel contesto della \textbf{domotica intelligente} e dell’\textbf{Internet of Things (IoT)}, con un focus particolare sull’usabilità, l’affidabilità e la sicurezza dei sistemi interconnessi.
    

    \subsection{Obiettivo}
        Gli obiettivi principali del progetto sono:
        \begin{itemize}
            \item Progettare un’applicazione web responsive per la gestione centralizzata degli impianti e dei dispositivi connessi;
            \item Implementare un’infrastruttura \textbf{cloud} per l’elaborazione e l’archiviazione dei dati generati dai dispositivi smart;
            \item Fornire strumenti per il \textbf{monitoraggio in tempo reale}, il \textbf{controllo remoto} e la \textbf{notifica di eventi o anomalie};
            \item Garantire semplicità d’uso e accessibilità da diversi dispositivi (smartphone, tablet, PC);
            \item Integrare un sistema di autenticazione e autorizzazione basato su ruoli, per differenziare i livelli di accesso degli utenti.
        \end{itemize}
        Il sistema dovrà favorire una gestione efficiente e sicura delle informazioni, offrendo una visione unificata degli impianti presenti nella struttura.
    

    \subsection{Tecnologie richieste}
        Le tecnologie e gli strumenti indicati dal proponente comprendono:
        \begin{itemize}
            \item \textbf{Docker} per la gestione e il deploy dell’infrastruttura cloud;
            \item \textbf{Git} come sistema di versionamento del codice;
            \item \textbf{KNX IoT 3rd-party API} per la comunicazione con gli impianti View Wireless e i dispositivi KNX compatibili;
            \item Framework per il frontend come \textbf{Angular}, \textbf{React} o \textbf{Flask};
            \item Linguaggi per il backend come \textbf{Node.js}, \textbf{Java} o \textbf{Python};
            \item Integrazione con protocolli di sicurezza e autenticazione quali \textbf{OAuth2} e meccanismi di \textbf{notifica push} per la gestione degli eventi in tempo reale.
        \end{itemize}
        L’architettura complessiva deve garantire \textbf{scalabilità}, \textbf{affidabilità} e un’elevata interoperabilità tra i diversi dispositivi connessi.
    

    \subsection{Pro e Contro}
        \begin{center}
            \begin{tabular}{|p{9cm}|p{5cm}|}
                \hline
                \textbf{Pro} & \textbf{Contro} \\
                \hline
                Progetto strutturato in modo chiaro, con requisiti tecnici ben definiti e supporto costante da parte dell’azienda proponente. & Richiede l’apprendimento e l’integrazione di molte tecnologie e API specifiche (es. KNX IoT). \\
                \hline
                Opportunità di lavorare in un contesto reale e professionale, con un impatto concreto nel campo della domotica e dell’assistenza. & Complessità elevata dovuta all’interazione tra più livelli applicativi e dispositivi eterogenei. \\
                \hline
                Focus su sicurezza, interoperabilità e architettura cloud, coerente con le tendenze tecnologiche attuali. & Maggiore rigidità nei vincoli tecnologici rispetto ad altri capitolati. \\
                \hline
            \end{tabular}
        \end{center}
    

    \subsection{Conclusioni}
        Il capitolato \textbf{C9 - View4Life} rappresenta un progetto completo e tecnicamente avanzato, con un impianto architetturale solido e un forte orientamento alla sicurezza e all’affidabilità.
        Tuttavia, la notevole complessità tecnica e l’obbligo di utilizzare tecnologie e protocolli specifici hanno portato il gruppo a non selezionare questa proposta come principale.
        Il gruppo riconosce comunque l’elevato valore tecnologico e la professionalità del proponente Vimar S.p.A., che ha presentato un progetto chiaro, ben documentato e di grande rilevanza nel settore della domotica intelligente.
    

\end{document}
