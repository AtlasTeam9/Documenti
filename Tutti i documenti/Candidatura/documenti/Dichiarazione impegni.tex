\documentclass[a4paper,12pt]{article}

% ----------------------------
% Pacchetti utili
% ----------------------------
\usepackage[utf8]{inputenc}
\usepackage[T1]{fontenc}
\usepackage[italian]{babel}
\usepackage{graphicx}
\usepackage{xcolor}
\usepackage{colortbl}
\usepackage{geometry}
\usepackage{setspace}
\usepackage{fancyhdr}
\usepackage{tikz}
\usepackage[colorlinks=true, linkcolor=blue, urlcolor=blue, citecolor=blue]{hyperref}
% ----------------------------
% Impostazioni pagina
% ----------------------------
\geometry{
    top=2cm,
    bottom=2cm,
    left=2cm,
    right=2cm
}

\setstretch{1.2}

% ----------------------------
% Dati personalizzabili
% ----------------------------
\newcommand{\Gruppo}{Atlas}
\newcommand{\Email}{\href{mailto:team9.atlas@gmail.com}{\textcolor{blue}{\underline{team9.atlas@gmail.com}}}}
\newcommand{\TitoloUno}{Dichiarazione degli impegni}
\newcommand{\DataModifica}{28/10/2025}
\newcommand{\LogoGruppo}{img/AtlasLogo.png} % Inserisci il file del logo

% --- Nuove variabili aggiunte ---
\newcommand{\VersioneDocumento}{v0.2} % <-- modifica qui la versione o ID
\newcommand{\Interno}{Interno} 

\pagestyle{fancy}
\fancyhf{}
\fancyhead[L]{\Gruppo}
\fancyhead[R]{Documento: \Interno \space - \space \DataModifica}
\fancyfoot[C]{\thepage}

% ----------------------------
% Inizio documento
% ----------------------------
\begin{document}

% ----------------------------
% Prima pagina
% ----------------------------
\begin{titlepage}
    \centering

    % Logo in alto
    \vspace*{0cm}
    \begin{tikzpicture}
        \clip (0,-0.1) circle (4.6cm); % raggio = metà della larghezza desiderata
        \node at (0,0) {\includegraphics[width=10cm]{\LogoGruppo}};
    \end{tikzpicture}\\[0.8cm]

    % Nome gruppo ed email sotto al logo
    {\LARGE \textbf{\Gruppo}}\\[0.1cm]
    {\large \Email}\\[1.2cm]

    % Sottotitolo del progetto
    {\Large \textit{Progetto di ingegneria del software A.A. 2025/2026}}\\[1.5cm]

    % Titolo principale
    {\Huge \textbf{\TitoloUno}}\\[.5cm]

    % Dati riunione
    \begin{tabular}{rl}
        \textbf{Data ultima modifica:} & \DataModifica \\
        \textbf{Versione:} & \VersioneDocumento \\
    \end{tabular}

\end{titlepage}


\section*{Tabella delle revisioni}{
    \begin{center} 
        \begin{tabular}{|l|l|l|l|l|}
            \hline
            \textbf{Versione} & \textbf{Data} & \textbf{Autore} & \textbf{Verificatore} & \textbf{Descrizione} \\
            \hline
            \VersioneDocumento & 2025/10/29 & Giacomo Giora & Francesco Marcolongo & Aggiunta contenuti \\
            \hline
            v0.1 & 2025/10/28 & Andrea Difino & & Prima Stesura \\
            \hline
        \end{tabular}
    \end{center}
}

\newpage

\tableofcontents

\newpage

\section{Introduzione}{
    In questo documento è presente il preventivo del progetto, la stima delle ore previste e la ripartizione di quest'ultime tra i membri del gruppo, definiti in base a un ragionamento sui ruoli necessari a svolgere il progetto riguardante il capitolato \textbf{C1 - Automated EN18031 Compliance Verification} proposto dall'azienda \textbf{Bluewind S.r.l.}.
}

\section{Impegni Orari}{
    Ogni componente del gruppo \textit{Atlas} si impegna a dedicare 91 ore produttive per il progetto, ottenendo un monte ore totale di 637 ore. 
    \newline La seguente tabella illustra la divisione delle ore per ogni ruolo con relativi costi:
    \begin{center}
        \begin{tabular}{|l|c|c|c|c|c|}
            \hline
            \rowcolor{gray!20}
            \textbf{Ruolo} & \textbf{Costo/h (€)} & \textbf{Stima Ore} & \textbf{Ore/Membro} & \textbf{Costo (€)} & \textbf{\%} \\
            \hline
            Responsabile & 30 & 56 & 8 & 1680 & 8.8 \\
            \hline
            Amministratore & 20 & 70 & 10 & 1400 & 11.0 \\
            \hline
            Analista & 25 & 84 & 12 & 2100 & 13.2 \\
            \hline
            Progettista & 25 & 112 & 16 & 2800 & 17.5 \\
            \hline
            Verificatore & 15 & 168 & 24 & 2520 & 26.4 \\
            \hline
            Programmatore & 15 & 147 & 21 & 2205 & 23.1 \\
            \hline
            \rowcolor{gray!20}
            \multicolumn{2}{|c|}{\textbf{Totale}} & \textbf{637} & \textbf{91} & \textbf{12705} & \textbf{100} \\
            \hline
        \end{tabular}

        \begin{figure}[h]
            \centering
            \begin{tikzpicture}
                \node at (0,0) {\includegraphics[width=15cm]{img/SuddivisioneRuoli.png}};
            \end{tikzpicture}
            \caption{Suddivisione ruoli}
            \label{fig:suddivisione-ruoli}
        \end{figure}
    \end{center}
}

\newpage
\section{Ruoli e motivazioni delle ore dedicate}{
    \subsection{Funzione ruoli}{
        Il calcolo del monte ore per ruolo è avvenuto tenendo in considerazione le seguenti funzioni per i ruoli da ricoprire:
        
        \begin{itemize}
            \item \textbf{Responsabile}: Punto di riferimento per le comunicazioni con il committente. Deve riuscire ad anticipare l'evoluzione del progetto in modo da pianificare le attività, gestire il team e tenere sotto controllo i progressi. Ha responsabilità di scelta e approvazione, per cui è essenziale e partecipa per tutta la durata del progetto;
            \item \textbf{Amministratore}: Si occupa dell'efficienza e dell'operatività dell'ambiente di sviluppo, deve assicurarsi che in ogni istante le risorse siano presenti e operanti. Ha il compito di gestione del controllo della configurazione del prodotto, del versionamento e della documentazione di progetto e definisce procedure, strumenti e convenzioni;
            \item \textbf{Analista}: Si occupa di capire e formalizzare il problema in maniera chiara. Il suo lavoro ha grande impatto sulla riuscita del progetto, in quanto raccoglie i requisiti dal cliente, li analizza e li specifica in modo chiaro e verificabile;
            \item \textbf{Progettista}: Si occupa dello sviluppo della soluzione al problema presentato tramite le attività di progettazione. Traduce i requisiti in architettura e soluzioni tecniche, definisce la struttura del sistema e sceglie tecnologie, pattern, protocolli di comunicazione;
            \item \textbf{Programmatore}: Si occupa di implementare la soluzione trovata dal progettista in codice in test di ausilio alla verifica, realizzando concretamente il software;
            \item \textbf{Verificatore}: Deve essere a conoscenza delle norme e deve avere capacità di giudizio e relazione. Si occupa di attività di verifica e validazione, partecipando all'intero ciclo di vita e controllando che il prodotto rispetti i requisiti e sia di qualità;
        \end{itemize}
    }
    
    \subsection{Distribuzione ore}{
        La distribuzione delle ore tra i vari ruoli è stata definita in funzione delle responsabilità specifiche e delle necessità progettuali relative al capitolato \textbf{C1 - Automated EN18031 Compliance Verification}.
        \vspace{0.3cm}
        \newline
        La figura del responsabile non è coinvolta in tutte le attività operative, infatti il suo ruolo è principalmente di coordinamento, pianificazione e controllo. Con 56 ore, dedicherà il tempo a definire piani, supervisionare l'avanzamento e approvare le consegne.
        \vspace{0.3cm}
        \newline
        L'amministratore deve impostare l'ambiente di lavoro e mantenerlo efficiente per tutto il progetto. Avendo un ruolo tecnico-organizzativo, le sue attività saranno periodiche ma non costanti: intense all'inizio e in corrispondenza dei rilasci, per un totale di 70 ore.
        \vspace{0.3cm}
        \newline
        La figura dell'analista invece impiega tempo significativo nella fase inziale per l'analisi dei requisiti, che richiede confronto con il committente e formalizzazione accurata, portando il suo monte ore a 84.
        \vspace{0.3cm}
        \newline
        Il progettista, con 112 ore, richiede tempo per la traduzione dei requisiti e per produrre scelte tecnologiche, ed è quindi una fase complessa che richiede precisione e revisione continua.
        \vspace{0.3cm}
        \newline
        Il programmatore con una stima di 147 ore, costituisce la parte sostanziale del lavoro di implementazione del prodotto, richiedendo tempo per implementare e correggere funzionalità.
        \vspace{0.3cm}
        \newline
        Il verificatore infine, con 168 ore, è la voce con più ore in assoluto, essendo necessario in tutte le fasi di progetto per garantire la qualità del prodotto attraverso test accurati.
        \vspace{0.3cm}
        \newline
        Questa distribuzione mira a bilanciare le esigenze del progetto, essendo esso complesso dal punto di vista dell'analisi e della verifica, garantendo che ogni ruolo abbia il tempo necessario per adempiere alle proprie responsabilità in modo efficace.
    }
}

\section{Rotazione dei ruoli}{
    Le ore di lavoro verranno suddivise equamente tra i membri del gruppo, in modo da garantire che ciascuno possa ricoprire tutti i ruoli nel corso del progetto. È previsto un periodo di assegnazione del ruolo di due settimane, al termine del quale avverrà una rotazione e ciascun membro assumerà un ruolo diverso rispetto a quello svolto in precedenza, rispettando i vincoli di progetto. In questo modo, ogni membro del gruppo avrà l'opportunità di acquisire esperienza in ogni fase di ciclo di vita del software, migliorando le proprie competenze e contribuendo in modo più completo al successo del progetto.
}

\section{Preventivo e stima di consegna}{
    A seguito delle valutazioni del gruppo Atlas, l'impegno orario totale è di nr. 637 ore, il preventivo calcolato relativo al capitolato \textit{C1 - Automated EN18031 Compliance Verification} equivale a \textbf{12705€} e la scadenza stiamata di consegna è prevista entro e non oltre la data \textbf{2026-03-20}.
}

\end{document}