\documentclass[a4paper,12pt]{article}

% ----------------------------
% Pacchetti utili
% ----------------------------
\usepackage[utf8]{inputenc}
\usepackage[T1]{fontenc}
\usepackage[italian]{babel}
\usepackage{graphicx}
\usepackage{xcolor}
\usepackage{colortbl}
\usepackage{geometry}
\usepackage{setspace}
\usepackage{fancyhdr}
\usepackage{tikz}
\usepackage[colorlinks=true, linkcolor=blue, urlcolor=blue, citecolor=blue]{hyperref}
% ----------------------------
% Impostazioni pagina
% ----------------------------
\geometry{
    top=2cm,
    bottom=2cm,
    left=2cm,
    right=2cm
}

\setstretch{1.2}

% ----------------------------
% Dati personalizzabili
% ----------------------------
\newcommand{\Gruppo}{Atlas}
\newcommand{\Email}{\href{mailto:team9.atlas@gmail.com}{\textcolor{blue}{\underline{team9.atlas@gmail.com}}}}
\newcommand{\TitoloUno}{Lettera di Presentazione}
\newcommand{\DataModifica}{2025/10/30}
\newcommand{\LogoGruppo}{img/AtlasLogo.png} % Inserisci il file del logo

% --- Nuove variabili aggiunte ---
\newcommand{\VersioneDocumento}{v0.1.0} % <-- modifica qui la versione o ID

\pagestyle{fancy}
\fancyhf{}
\fancyhead[L]{\Gruppo}
\fancyhead[R]{Lettera di Presentazione \space - \space \DataModifica}
\fancyfoot[C]{\thepage}

% ----------------------------
% Inizio documento
% ----------------------------
\begin{document}

% ----------------------------
% Prima pagina
% ----------------------------
\begin{titlepage}
    \centering

    % Logo in alto
    \vspace*{0cm}
    \begin{tikzpicture}
        \clip (0,-0.1) circle (4.6cm); % raggio = metà della larghezza desiderata
        \node at (0,0) {\includegraphics[width=10cm]{\LogoGruppo}};
    \end{tikzpicture}\\[0.8cm]

    % Nome gruppo ed email sotto al logo
    {\LARGE \textbf{\Gruppo}}\\[0.1cm]
    {\large \Email}\\[1.2cm]

    % Sottotitolo del progetto
    {\Large \textit{Progetto di ingegneria del software A.A. 2025/2026}}\\[1.5cm]

    % Titolo principale
    {\Huge \textbf{\TitoloUno}}\\[.5cm]

    % Dati riunione
    \begin{tabular}{rl}
        \textbf{Data ultima modifica:} & \DataModifica \\
        \textbf{Versione:} & \VersioneDocumento \\
    \end{tabular}

    \hspace{1cm}

     {\large \textbf{Componenti}}\\[0.5cm]
    \begin{tabular}{l|l}
        \textbf{Nome} & \textbf{matricola} \\
        \hline
        Andrea Difino & 2101072 \\
        Federico Simonetto & 2113180 \\
        Riccardo Valerio & 2075517 \\
        Francesco Marcolongo & 2101093 \\
        Michele Tesser & 2111012 \\
        Giacomo Giora & 2101094 \\
        Bilal Sabic & 2111022 \\
    \end{tabular}

\end{titlepage}


\section*{Tabella delle revisioni}
    \begin{center} 
        \begin{tabular}{|l|l|l|l|l|}
            \hline
            \textbf{Versione} & \textbf{Data} & \textbf{Autore} & \textbf{Verificatore} & \textbf{Descrizione} \\
            \hline
            v0.1.0 & 2025/10/30 & Michele Tesser & & Prima Stesura \\
            \hline
        \end{tabular}
    \end{center}


\newpage

\section*{Presentazione gruppo Atlas}

Il gruppo Atlas è lieto di manifestare il proprio interesse nel prendere parte allo sviluppo del capitolato \textbf{C1 – Automated EN18031 Compliance Verification}, proposto dall'azienda \textbf{Bluewind S.r.l.}.

Abbiamo scelto questo progetto poiché il progetto offre l'opportunità di approfondire tematiche avanzate quali automazione dei processi di verifica, validazione dei sistemi software e gestione dei requisiti secondo standard internazionali. Il gruppo ritiene che tale esperienza possa contribuire in modo significativo alla nostra crescita tecnica e professionale, grazie al confronto con una realtà aziendale strutturata e con solide competenze nel settore embedded e nel software industriale.

\section*{Piano di lavoro}

Il gruppo Atlas dichiara piena disponibilità a collaborare attivamente con l’azienda ospitante e con i referenti accademici per garantire il corretto avanzamento delle attività.
La data prevista di completamento del progetto è fissata al: \textbf{20 Marzo 2026}.
Il costo stimato dell'attività del gruppo è pari a: \textbf{12 705\texteuro}.
Ulteriori informazioni sul gruppo e sul progetto sono disponibili sul repository di riferimento:

\begin{center}
    \url{https://github.com/AtlasTeam9/Documenti} \\
    \url{https://atlasteam9.github.io/Atlas/}
\end{center}

Dove sono presenti:
\begin{itemize}
    \item Lettera di presentazione
    \item Dichiarazione d'Impegni
    \item Analisi dei capitolati e motivazione della scelta
    \item Verbali interni
    \item Verbali esterni
\end{itemize}

\section*{Conclusioni}
Vi ringraziamo per la vostra disponibilità e cordiali saluti, \\
\textbf{Gruppo Atlas}

\end{document}
