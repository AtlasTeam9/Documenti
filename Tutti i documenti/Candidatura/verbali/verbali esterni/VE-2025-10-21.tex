\documentclass[a4paper,12pt]{article}

% ----------------------------
% Pacchetti utili
% ----------------------------
\usepackage[utf8]{inputenc}
\usepackage[T1]{fontenc}
\usepackage[italian]{babel}
\usepackage{graphicx}
\usepackage{xcolor}
\usepackage{geometry}
\usepackage{setspace}
\usepackage{fancyhdr}
\usepackage{tikz}
\usepackage[colorlinks=true, linkcolor=blue, urlcolor=blue, citecolor=blue]{hyperref}
% ----------------------------
% Impostazioni pagina
% ----------------------------
\geometry{
    top=2cm,
    bottom=2cm,
    left=2cm,
    right=2cm
}

\setstretch{1.2}

% ----------------------------
% Dati personalizzabili
% ----------------------------
\newcommand{\Gruppo}{Atlas}
\newcommand{\Email}{\href{mailto:team9.atlas@gmail.com}{\textcolor{blue}{\underline{team9.atlas@gmail.com}}}}
\newcommand{\TitoloVerbale}{Verbale della Riunione}
\newcommand{\DataVerbale}{2025/10/21}
\newcommand{\OraInizio}{14:30}
\newcommand{\OraFine}{14:55}
\newcommand{\LuogoVerbale}{Chiamata Zoom}
\newcommand{\LogoGruppo}{img/AtlasLogo.png} % Inserisci il file del logo
\newcommand{\AbstractVerbale}{%
In questo verbale vengono riportate le domande poste e le risposte ricevute durante il primo meeting effettuato dal team \Gruppo \space con l'azienda \textbf{Zucchetti Spa}, riguardante il capitolato C6 del corso di Ingegneria del Sofware. L'incontro si è tenuto da remoto nel giorno \DataVerbale \space dalle \OraInizio \space alle \OraFine.
}

% --- Nuove variabili aggiunte ---
\newcommand{\VersioneVerbale}{v1.1} % <-- modifica qui la versione o ID
\newcommand{\VerbaleEsterno}{Esterno} 

\pagestyle{fancy}
\fancyhf{}
\fancyhead[L]{\Gruppo}
\fancyhead[R]{Verbale: \VerbaleEsterno \space - \space \DataVerbale}
\fancyfoot[C]{\thepage}

% ----------------------------
% Inizio documento
% ----------------------------
\begin{document}

% ----------------------------
% Prima pagina
% ----------------------------
\begin{titlepage}
    \centering

    % Logo
    \vspace*{0cm}
    %\includegraphics[width=10cm]{\LogoGruppo}\\[.5cm]
    \begin{tikzpicture}
        \clip (0,-0.1) circle (4.6cm); % raggio = metà della larghezza desiderata
        \node at (0,0) {\includegraphics[width=10cm]{\LogoGruppo}};
    \end{tikzpicture}\\
    [.5cm]
    % Titolo
    {\Huge \textbf{\TitoloVerbale}}\\[0.8cm]
    {\LARGE \Gruppo}\\[0.1cm]
    {\Email}\\[1.2cm]

    % Dati riunione
    \begin{tabular}{rl}
        \textbf{Data:} & \DataVerbale \\
        \textbf{Luogo:} & \LuogoVerbale \\
        \textbf{Versione:} & \VersioneVerbale \\
        \textbf{Tipo:} & \VerbaleEsterno \\
    \end{tabular}

    \vspace{1.2cm}

    % Componenti e ruoli
    {\large \textbf{Partecipanti}}\\[0.5cm]
    \begin{tabular}{l|l}
        \textbf{Nome} & \textbf{Presenza} \\
        \hline
        \textbf{Gregorio Piccoli} & SI (\textbf{Zucchetti Spa})\\
        Andrea Difino & SI \\
        Federico Simonetto & SI \\
        Riccardo Valerio & SI \\
        Francesco Marcolongo & SI \\
        Michele Tesser & SI \\
        Giacomo Giora & SI \\
        Bilal Sabic & SI \\
    \end{tabular}

\end{titlepage}

\section*{Registro delle modifiche}{
    \begin{center} 
        \begin{tabular}{|l|l|l|l|l|}
            \hline
            \textbf{Versione} & \textbf{Data} & \textbf{Autore} & \textbf{Verificatore} & \textbf{Descrizione} \\
            \hline
            \VersioneVerbale & 2025/10/23 & Giacomo Giora & Tesser Michele & Correzione grammatica\\ 
            \hline
            v1.0 & 2025/10/21 & Giacomo Giora & Tesser Michele & Sistemazione contenuto\\ 
            \hline
            v0.1 & 2025/10/21 & Giacomo Giora &  & Prima stesura\\ 
            \hline
       \end{tabular}
    \end{center}
}

\newpage

\tableofcontents

\newpage
% ----------------------------
% Inizio contenuto verbale
% ----------------------------
\section{Abstract}{
    % Abstract
    \begin{minipage}{0.9\textwidth}
        \small
        \AbstractVerbale
    \end{minipage}
}


\section{Ordine del Giorno}{
    \begin{itemize}
        \item Chiarimenti sul capitolato
        \item Esposizione domande riguardanti il progetto
    \end{itemize}
}


\section{Domande e risposte}{

    \subsection{Qualità del supporto LLM}
    \textbf{Domanda:} Qual è il livello di accuratezza o qualità atteso per le funzioni di sintesi, traduzione e riscrittura? Si devono solo dimostrare le funzionalità o deve esserci una valutazione qualitativa sui risultati?

    \textbf{Risposta:} Non è richiesto un livello minimo di accuratezza o qualità per le funzioni basate su LLM. È sufficiente dimostrare il funzionamento e le potenzialità del sistema. L'azienda lascia libertà di implementare diverse strategie per la gestione delle funzioni base, ad esempio, oltre al metodo dei sei cappelli, è stato menzionato anche il metodo PARA.

    \subsection{Gestione UI delle funzionalità intelligenti}
    \textbf{Domanda:} Come devono essere gestite le selezioni di testo su cui applicare le operazioni dell'LLM (riassunto, traduzione, critica)? L'utente seleziona con il mouse e poi sceglie l'operazione? O inserisce un comando in un campo separato?

    \textbf{Risposta:} L'interfaccia dovrà permettere di applicare le operazioni sia alla pagina corrente sia a una porzione selezionata di testo. Le funzioni potranno essere accessibili anche tramite un menu contestuale, ad esempio con il click destro del mouse.

    \subsection{Modalità salvataggio files}
    \textbf{Domanda:} Quando salviamo il file, verrà salvato sia il file markdown che il riassunto come un file di testo oppure andiamo a salvare solo il riassunto?

    \textbf{Risposta:} Al momento del salvataggio dovrà essere memorizzato il documento originale. Eventuali elaborazioni o riassunti potranno essere gestiti come informazioni aggiuntive, ma non sostitutive del file principale.

    \subsection{Visione del rendering}
    \textbf{Domanda:} Il rendering viene fatto in tempo reale oppure l'utente dovrà esplicitamente richiederlo tramite un tasto ad esempio?

    \textbf{Risposta:} Il rendering verrà eseguito su richiesta esplicita dell'utente, ad esempio tramite un pulsante o una specifica opzione dell'interfaccia.

    \subsection{API tokens}
    \textbf{Domanda:} Visto che è possibile lavorare con l'API di chatGPT, dovremo tenere in considerazione anche il limite di token? Se magari usiamo tutti i token, l'utente dovrà aspettare un tempo predeterminato oppure esiste qualche metodo per evitare questo problema di token?

    \textbf{Risposta:} L'utente non dovrà essere consapevole della gestione dei token. Il monitoraggio e l'eventuale limitazione avverranno lato server. L'azienda utilizza già un sistema interno, basato su LiteLLM, per gestire costi, accessi e modelli. Poiché Zucchetti dispone di AI ospitate localmente, l'impatto economico dell'utilizzo dei modelli è ridotto.

    \subsection{Comunicazione}
    \textbf{Domanda:} Come avverrà la comunicazione con l'azienda? Con che frequenze e in che modalità?

    \textbf{Risposta:} L'azienda si è resa disponibile per fornire chiarimenti e feedback in qualsiasi momento, sia a distanza che, se necessario, di persona presso la sede Zucchetti, nel rispetto degli impegni lavorativi del personale. È incoraggiato il contatto diretto in caso di dubbi o per mostrare progressi rilevanti.

    \subsection{Consigli}
    \textbf{Domanda:} Considerata la pluriennale esperienza della Zucchetti chiediamo qualche libero consiglio.

    \textbf{Risposta:} L'azienda ha evidenziato come il capitolato sia volutamente aperto, lasciando ampio spazio alla creatività del team. È stato suggerito di concentrarsi su interfacce chiare e intuitive, ad esempio integrando viste grafiche simili a quelle dell'applicazione Obsidian. È stata inoltre sottolineata l'importanza di limitare l'uso dei LLM alla rielaborazione del testo, evitando la generazione completa dei contenuti per mantenere coerenza e qualità.

}
\newpage

\section{Conclusioni}{
  Durante l'incontro con i rappresentanti di Zucchetti S.p.A., il team ha ricevuto risposte chiare e dettagliate alle domande relative al capitolato C6 - Second Brain.
  Sono stati approfonditi diversi aspetti tecnici e funzionali del progetto, tra cui il ruolo dei LLM, la gestione dell'interfaccia utente, le modalità di salvataggio e visualizzazione dei documenti, e le linee guida per l'interazione con l'azienda.
  L'incontro ha confermato la disponibilità di Zucchetti a collaborare e fornire supporto durante lo sviluppo. Le risposte ottenute hanno contribuito a delineare con maggiore precisione gli obiettivi del progetto e le libertà di implementazione concesse al team.
}

\vspace{2cm}
\begin{flushright}
    \textbf{Approvazione dell'azienda} \\
    Il proponente,\\[0.5cm]
    \rule{6cm}{0.4pt}\\
\end{flushright}



\end{document}
