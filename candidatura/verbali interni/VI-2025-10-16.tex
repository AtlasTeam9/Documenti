\documentclass[a4paper,12pt]{article}

% ----------------------------
% Pacchetti utili
% ----------------------------
\usepackage[utf8]{inputenc}
\usepackage[T1]{fontenc}
\usepackage[italian]{babel}
\usepackage{graphicx}
\usepackage{xcolor}
\usepackage{geometry}
\usepackage{setspace}
\usepackage{fancyhdr}
\usepackage{tikz}
\usepackage[colorlinks=true, linkcolor=blue, urlcolor=blue, citecolor=blue]{hyperref}
% ----------------------------
% Impostazioni pagina
% ----------------------------
\geometry{
    top=2cm,
    bottom=2cm,
    left=2cm,
    right=2cm
}

\setstretch{1.2}

% ----------------------------
% Dati personalizzabili
% ----------------------------
\newcommand{\Gruppo}{Atlas}
\newcommand{\Email}{\href{mailto:team9.atlas@gmail.com}{\textcolor{blue}{\underline{team9.atlas@gmail.com}}}}
\newcommand{\TitoloVerbale}{Verbale della Riunione}
\newcommand{\DataVerbale}{2025/10/16}
\newcommand{\OraInizio}{17:30}
\newcommand{\OraFine}{19:25}
\newcommand{\LuogoVerbale}{chiamata Discord}
\newcommand{\LogoGruppo}{img/AtlasLogo.png} % Inserisci il file del logo
\newcommand{\AbstractVerbale}{%
In questo verbale vengono riportati i principali argomenti discussi, le decisioni prese e le attività pianificate durante la riunione del gruppo nel giorno \DataVerbale \space dalle \OraInizio \space alle \space \OraFine .
}

% --- Nuove variabili aggiunte ---
\newcommand{\VersioneVerbale}{v0.1} % <-- modifica qui la versione o ID
\newcommand{\VerbaleInterno}{Interno} 

\pagestyle{fancy}
\fancyhf{}
\fancyhead[L]{\Gruppo}
\fancyhead[R]{Verbale: \VerbaleInterno - \space \DataVerbale}
\fancyfoot[C]{\thepage}

% ----------------------------
% Inizio documento
% ----------------------------
\begin{document}

% ----------------------------
% Prima pagina
% ----------------------------
\begin{titlepage}
    \centering

    % Logo
    \vspace*{0cm}
    %\includegraphics[width=10cm]{\LogoGruppo}\\[.5cm]
    \begin{tikzpicture}
        \clip (0,-0.1) circle (4.6cm); % raggio = metà della larghezza desiderata
        \node at (0,0) {\includegraphics[width=10cm]{\LogoGruppo}};
    \end{tikzpicture}\\
    [.5cm]
    % Titolo
    {\Huge \textbf{\TitoloVerbale}}\\[0.8cm]
    {\LARGE \Gruppo}\\[0.1cm]
    {\Email}\\[1.2cm]

    % Dati riunione
    \begin{tabular}{rl}
        \textbf{Data:} & \DataVerbale \\
        \textbf{Luogo:} & \LuogoVerbale \\
        \textbf{Versione:} & \VersioneVerbale \\
        \textbf{Tipo:} & \VerbaleInterno \\
    \end{tabular}

    \vspace{1.2cm}

    % Componenti e ruoli
    {\large \textbf{Partecipanti}}\\[0.5cm]
    \begin{tabular}{l|l}
        \textbf{Nome} & \textbf{Presenza} \\
        \hline
        Andrea Difino & SI \\
        Federico Simonetto & SI \\
        Riccardo Valerio & SI \\
        Francesco Marcolongo & SI \\
        Michele Tesser & SI \\
        Giacomo Giora & SI \\
        Bilal Sabic & SI \\
    \end{tabular}

\end{titlepage}

\section*{Registro delle modifiche}{
    \begin{center} 
        \begin{tabular}{|l|l|l|l|l|}
            \hline
            \textbf{Versione} & \textbf{Data} & \textbf{Autore} & \textbf{Verificatore} & \textbf{Descrizione} \\
            \hline
            \VersioneVerbale & \DataVerbale & Andrea Difino & & Prima stesura\\
            \hline
        \end{tabular}
    \end{center}
}

\newpage

\tableofcontents

\newpage
% ----------------------------
% Inizio contenuto verbale
% ----------------------------
\section{Abstract}{
    % Abstract
    \begin{minipage}{0.9\textwidth}
        \small
        \AbstractVerbale
    \end{minipage}
}


\section{Ordine del Giorno}{
    \begin{enumerate}
        \item Creazione e definizione del team (Nome, Logo, Email).
        \item Creazione organizzazione su GitHub e repository per documenti.
        \item Definizione struttura dei verbali interni.
        \item Discussione per la preferenza dei capitolati disponibili.
        \item Pianificazione delle comunicazioni con le aziende (mail e raccolta domande).
        \item Assegnazione dei ruoli e suddivisione dei capitolati per lo studio.
    \end{enumerate}
}

\section{Discussione}{
    \subsection{Creazione e definizione del team}{
        All'inizio della riunione, il team ha deciso di chiamarsi \textbf{ATLAS}, derivante da \emph{Atlante}.  
        Un nome che rappresenta forza e che raffigura, come si vede anche dal logo, un grande peso - il peso dei nuovi doveri e delle nuove conoscenze che sorreggeremo.

        Il logo scelto per il team è:

        \begin{center}  
            \begin{tikzpicture}
                \clip (0,-0.1) circle (4.6cm); % raggio = metà della larghezza desiderata
                \node at (0,0) {\includegraphics[width=10cm]{\LogoGruppo}};
            \end{tikzpicture}\\
        \end{center}

        Infine, per consolidare il gruppo, è stata creata l'email ufficiale: \texttt{\Email}.
    }

    \subsection{Creazione organizzazione su GitHub}{
        È stata creata l'organizzazione su GitHub per la gestione \textbf{efficace} delle repository del team. 
    }

    \subsection{Definizione struttura dei verbali}{
        Si è discusso della struttura dei verbali, distinguendo tra \textbf{interni} ed \textbf{esterni}
    }

    \subsection{Discussione per la preferenza dei capitolati disponibili}{
        Sono stati analizzati i capitolati disponibili: C1 è risultato il preferito, C2 la seconda scelta, e C6 come alternativa.  
    }

    \subsection{Pianificazione delle comunicazioni con le aziende}{
        Sono state inviate email alle aziende di interesse per richiedere un incontro per la prossima settimana e creato un file Google Docs per raccogliere le domande da porre durante le call e filtrare quelle migliori.  
        
    }

    \subsection{Assegnazione dei ruoli e suddivisione dei capitolati per lo studio}{
        Infine, è stato definito il responsabile per le prime due settimane e si è proceduto alla suddivisione dei capitolati tra i membri per lo studio e la preparazione delle domande.
    }

}


\section{Decisioni Prese}{
    \begin{center}
    \begin{tabular}{|c|p{11cm}|}
        \hline
        \textbf{ID} & \textbf{Decisione} \\
        \hline
        D1-16/10/2025 & Definito Nome ATLAS, Logo ed email ufficiale del team. \\
        \hline
        D2-16/10/2025 & Creata organizzazione su GitHub con repository privata per documenti del gruppo. \\
        \hline
        D3-16/10/2025 & Definita struttura dei verbali interni ed esterni. \\
        \hline
        D4-16/10/2025 & Capitolo C1 scelto come preferito, C2 come seconda scelta, C6 alternativa. \\
        \hline
        D5-16/10/2025 & Creato file Google Docs per raccogliere le domande da porre alle aziende. \\
        \hline
        D6-16/10/2025 & Definito \textbf{Andrea Difino} responsabile per le prime due settimane. \\
        \hline
        D7-16/10/2025 & Suddivisione dei capitolati tra i membri per lo studio delle domande. \\
        \hline
        D8-16/10/2025 & Pianificata riunione per il 2025/10/17 \\
        \hline
    \end{tabular}
    \end{center}
}


\end{document}
